% !TEX TS-program = xelatex
% !TEX encoding = UTF-8

\documentclass[11pt, a4paper]{article} % use larger type; default would be 10pt

\usepackage{fontspec} % Font selection for XeLaTeX; see fontspec.pdf for documentation
\defaultfontfeatures{Mapping=tex-text} % to support TeX conventions like ``---''
\usepackage{xunicode} % Unicode support for LaTeX character names (accents, European chars, etc)
\usepackage{xltxtra} % Extra customizations for XeLaTeX
\usepackage{tikz}

\setmainfont[Ligatures=TeX]{Garamond} % set the main body font (\textrm), assumes Charis SIL is installed
%\setsansfont{Deja Vu Sans}
\setmonofont[Ligatures=TeX]{Fira Code}

% other LaTeX packages.....
\usepackage{fullpage}
\usepackage[top=2cm, bottom=4.5cm, left=2.5cm, right=2.5cm]{geometry}
\usepackage{amsmath,amsthm,amsfonts,amssymb,amscd,systeme}
\geometry{a4paper} 
%\usepackage[parfill]{parskip} % Activate to begin paragraphs with an empty line rather than an indent
\usepackage{fancyhdr}
\usepackage{listings}
\usepackage{graphicx}
\usepackage{hyperref}
\usepackage{multicol}

\renewcommand\lstlistingname{Algorithm}
\renewcommand\lstlistlistingname{Algorithms}
\def\lstlistingautorefname{Alg.}
\lstdefinestyle{mystyle}{
    % backgroundcolor=\color{backcolour},   
    % commentstyle=\color{codegreen},
    % keywordstyle=\color{magenta},
    % numberstyle=\tiny\color{codegray},
    % stringstyle=\color{codepurple},
    basicstyle=\ttfamily\footnotesize,
    breakatwhitespace=false,         
    breaklines=true,                 
    captionpos=b,                    
    keepspaces=true,                 
    numbers=left,                    
    numbersep=5pt,                  
    showspaces=false,                
    showstringspaces=false,
    showtabs=false,                  
    tabsize=2
}
\lstset{style=mystyle}

\newcommand\course{5 - Теорія складності}
\newcommand\hwnumber{ДЗ №3}                   % <-- homework number
\newcommand\idgroup{ФІ-91}                
\newcommand\idname{Михайло Корешков}  

\usepackage[framemethod=TikZ]{mdframed}
\mdfsetup{%
	backgroundcolor = black!5,
}
\mdfdefinestyle{ans}{%
    backgroundcolor = green!5,
    linecolor = green!50,
    linewidth = 1pt,
}

\pagestyle{fancyplain}
\headheight 35pt
\lhead{\idgroup \\ \idname}
\chead{\textbf{\Large \hwnumber}}
\rhead{\course \\ \today}
\lfoot{}
\cfoot{}
\rfoot{\small\thepage}
\headsep 1.5em

\linespread{1.2}

\begin{document}

\section*{№ 3.2}
Let $N = \overline{1,1000}$.
Let $A = \{n\in N : n \text{ has digit '3'}\}$.

Let $A_0 = \{n \in N : n \% 10 = 3\}$ - all numbers with $3$ as the first digit.
$$A_0 = \{3,13,23,...,983,993\} = \{3 + 10k : k=\overline{0,99}\}$$
$$|A_0| = 100$$

Let $A_1 = \{n \in N : n \text{ has '3' as the second digit, but not the first}\}$.
$$A_1 = \{30, 31, ..., 39, 130, 131, ..., 139, 230, ..., 930, ...\} =$$
$$= \{100k + 30 + p : k=\overline{0,9}, p\in\{0,1,2,4,5,6,7,8,9\}\}$$
$$|A_1| = 10\cdot 9 = 90$$

Let $A_2 = \{n \in N : n \text{ has '3' as the third digit, but not 1st nor 2nd}\}$.
$$A_2 = \{300 + 10k + p : k\in\{0,1,2,4,5,6,7,8,9\}, p\in\{0,1,2,4,5,6,7,8,9\}\}$$
$$|A_2| = 9\cdot 9 = 81$$

Note that $A_0 \cap A_1 = A_0 \cap A_2 = A_1 \cap A_2 = \varnothing$. So,
$$A = A_0 \sqcup A_1 \sqcup A_2$$
$$|A| = \sum |A_i| = 100 + 90 + 81 = 271$$

\section*{№ 3.3}
\begin{mdframed}
    Let $\mathcal N$ - нумерація Геделя унарних обчислювальних функцій.\\
    That is, $\Phi_{\mathcal N} = \{\varphi_i\}$ - set of computable unary functions.

    Prove:
    $$\forall \mathcal N,\; f \in \Phi_\mathcal N,\; k\in \mathbb N:\; \exists n>k:\; \varphi_n \simeq \varphi_{f(n)}$$
\end{mdframed}


\begin{mdframed}[backgroundcolor=violet!25]
    \textbf{\Large Теорема про нерухому точку}

    Для довільної нумерацій Геделя унарних обчислювальних функцій $\Phi = \{\varphi_i\}$,
    довільної \textbf{всюди визначеної} унарної обчислювальної функції $f \in \Phi^{tot} \subset \Phi$, існує $n \in \mathbb N$ таке, що
    $$\varphi_n \simeq \varphi_{f(n)}$$
\end{mdframed}

Тобто різниця між теоремами в тому, що ми доводимо існування нескінченної кількості таких нерухомих точок.
\pagebreak

\begin{proof}
Let $k' = k+1$

Розглянемо функцію $\varphi_{f(\varphi_x(x)+k')}(y)$.
\begin{equation}    
\varphi_{f(\varphi_x(x)+k')}(y) \simeq g'(f(\varphi_x(x)+k'),y) \simeq g(x,y)
\end{equation}

З $s_n^m$ теореми:
\begin{equation}
    \exists h\in\Phi^{tot}:\; \varphi_{f(\varphi_x(x)+k')}(y) \simeq \varphi_{h(x)}(y)
\end{equation}

$h$ - обчислювальна та всюди визначена. Розглянемо
% $$v(x) = h(x) + k'$$

\begin{equation}
    \Phi^{tot} \ni \varphi_m(x) \simeq h(x)
\end{equation}
% $v$ - обчислювальна та всюди визначена.

З всюди визначеності можемо розглянути наступне значення:
\begin{equation}
    \begin{cases}
        \varphi_m(m) = t;%\\
        % t = h(m) + k + 1 > k
    \end{cases}
\end{equation} 

З рівняння 2:
$$\varphi_{f(\varphi_m(m)+k')} \simeq \varphi_{h(m)} $$%\simeq \varphi_{v(m)-k'}$$

Тобто маємо
$$\varphi_{f(\varphi_m(m)+k')} \simeq \varphi_{\varphi_{m}(m)}$$
Підставивши $t$:
$$\varphi_{f(t)} \simeq \varphi_{t}$$

Отже, знайшли нерухому точку - функція $\varphi_n$ з номером $n=t$. 

TODO: АЛЕ $t$ МАЄ БУТИ БІЛЬШЕ $k$!!

\end{proof}
\end{document}

