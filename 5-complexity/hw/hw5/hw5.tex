    % !TEX TS-program = xelatex
% !TEX encoding = UTF-8

\documentclass[11pt, a4paper]{article} % use larger type; default would be 10pt

\usepackage{fontspec} % Font selection for XeLaTeX; see fontspec.pdf for documentation
\defaultfontfeatures{Mapping=tex-text} % to support TeX conventions like ``---''
\usepackage{xunicode} % Unicode support for LaTeX character names (accents, European chars, etc)
\usepackage{xltxtra} % Extra customizations for XeLaTeX
\usepackage{tikz}
\usetikzlibrary{arrows,calc,patterns}
\usetikzlibrary{decorations.pathreplacing,calligraphy}

\setmainfont[Ligatures=TeX]{Times New Roman} % set the main body font (\textrm), assumes Charis SIL is installed
%\setsansfont{Deja Vu Sans}
\setmonofont[Ligatures=TeX]{Fira Code}

% other LaTeX packages.....
\usepackage{fullpage}
\usepackage[top=2cm, bottom=4.5cm, left=2.5cm, right=2.5cm]{geometry}
\usepackage{amsmath,amsthm,amsfonts,amssymb,amscd,systeme}
\usepackage{cancel}
\geometry{a4paper} 
%\usepackage[parfill]{parskip} % Activate to begin paragraphs with an empty line rather than an indent
\usepackage{fancyhdr}
\usepackage{listings}
\usepackage{graphicx}
\usepackage{hyperref}
\usepackage{multicol}

\renewcommand\lstlistingname{Algorithm}
\renewcommand\lstlistlistingname{Algorithms}
\def\lstlistingautorefname{Alg.}
\lstdefinestyle{mystyle}{
    % backgroundcolor=\color{backcolour},   
    % commentstyle=\color{codegreen},
    % keywordstyle=\color{magenta},
    % numberstyle=\tiny\color{codegray},
    % stringstyle=\color{codepurple},
    basicstyle=\ttfamily\footnotesize,
    breakatwhitespace=false,         
    breaklines=true,                 
    captionpos=b,                    
    keepspaces=true,                 
    numbers=left,                    
    numbersep=5pt,                  
    showspaces=false,                
    showstringspaces=false,
    showtabs=false,                  
    tabsize=2
}
\lstset{style=mystyle}

\newcommand\course{5 - Complexity}
\newcommand\hwnumber{ДЗ №5}                   % <-- homework number
\newcommand\idgroup{ФІ-91}                
\newcommand\idname{Михайло Корешков}  

\usepackage[framemethod=TikZ]{mdframed}
\mdfsetup{%
	backgroundcolor = black!5,
}
\mdfdefinestyle{ans}{%
    backgroundcolor = green!5,
    linecolor = green!50,
    linewidth = 1pt,
}

\pagestyle{fancyplain}
\headheight 35pt
\lhead{\idgroup \\ \idname}
\chead{\textbf{\Large \hwnumber}}
\rhead{\course \\ \today}
\lfoot{}
\cfoot{}
\rfoot{\small\thepage}
\headsep 1.5em

\linespread{1.2}

\begin{document}
\section*{№ 5.1}
\subsection*{$L_1 \cap L_2$}
Запускаємо два вирішувача. Якщо обидва повернули 1, то повератємо 1. 
Максимальний час: $T(n) = O(T_1(n) + T_2(n))$ - також поліном.


\subsection*{$L_1 \cup L_2$}
Запускаємо два вирішувача. Якщо хоча б один повернув 1, то повератємо 1. 
Максимальний час (якщо обидва повернуть 0): $T(n) = O(T_1(n) + T_2(n))$ - також поліном.


\subsection*{$L_1 \cdot L_2$}
Запускаємо два вирішувача для кожного розбиття слова на два підслова. Якщо хоча б один повернув 1, то повератємо 1. 
Таких розбиттів скінченна кількість.

Оцінка згори: $T(n) = |w| \cdot O(T_1(n) + T_2(n))$ - також поліном.

\subsection*{$L_1^R$}
Просто записати у зворотньому порядку. Той самий час розпізнавання, тобто поліномний

\subsection*{$L_1^*$}
Замкнено відносно конкатенації $\implies$ замкнено відносно ступеня мови. 
Запускаємо розпізнавачі $L_1, L_1^2, ... , L_1^{|w|}$. Якщо хоча б один поверне 1, то повертаємо 1.

Час роботи буде не більше, ніж на поліном, більше за сумму часів розпізнавання кожного степеня. Маємо суму поліномів - поліном

\subsection*{№ 5.2}
\subsection*{a)}
$$\forall m: m \vdots m$$
А нам потрібно, щоб k було не менше, ніж максимальний дільник. Просто перевіряємо $k \ge m$.
Це аналогічно відніманню - $P$

\subsection*{b)}
Весь граф можна представити як $0,1-$ матрицю суміжності $A$. Підносимо $A$ в степінь $|V|$ - маємо матрицю досяжності.
Піднесення до степеня - $P$.


\subsection*{c)}
Весь граф можна представити як $0,1-$ матрицю суміжності $A$. Підносимо $A$ в степінь $|V|$ - маємо матрицю досяжності.
Піднесення до степеня - $P$.
Тепер просто дивимося, чи є в матриці $A^{|V|}$ нулі. Якщо є - деякі вершини недосяжні. Якщо всі одиниці - маємо лише одну компоненту зв'язності.

\subsection*{№ 5.5}
\begin{mdframed}
    Чи iснує така мова $L_1 \subseteq \{0,1\}^*$, що $L_1 \notin P$, але $L_1^* \in P$?
\end{mdframed}

let $L_1 = HALT \cup \{0,1\}$.
$L_1$ - невирішувана, тобто точно не в $P$.

$\{0,1\}^* \supset L_1^* = (HALT \cup \{0,1\})^* = HALT^* (\{0,1\}(HALT^*))^* \supseteq \{0,1\}^*$

Тобто $L_1^* = \{0,1\}^*$ - розпізнається за $O(1)$.

\subsection*{№ 5.6}
\begin{mdframed}
    Prove: $$P \subseteq NP \cap coNP$$
\end{mdframed}

Тобто prove $$L_1 \in P \implies L_1 \in NP \wedge L_1 \in coNP$$

\begin{proof}
    $$L_1 \in P \implies coL_1 \in P$$
    $$P \subseteq NP$$
    Тобто 
    $$coL_1 \in NP$$
    $$L_1 \in NP$$
    Тобто
    $$L_1 \in NP \cap coNP$$
\end{proof}

\subsection*{№ 5.7}
\begin{mdframed}
    Доведiть, що класи складностi $NP$ та $coNP$ є замкненими вiдносно по-
лiномiального зведення, а клас складностi $NP \cap coNP$ є замкненим вiдносно
зведення за Куком.
\end{mdframed}

\subsection*{1.}
\begin{proof}
    Нехай $L_1 \in NP$, МТ $M_1: M_1(w) = M_P(M_{L1}(w))$, де $M_P$ - деяка МТ, що працює за $P$ час.
    Тоді $M_1$ також буде в $NP$.

    Тобто будь-яке поліноміальне зведення не може вивести з $NP$. Аналогічно для $coNP$.
\end{proof}

\subsection*{2.}

\subsection*{№ 5.8}
\begin{proof}
    Нехай $$\begin{cases}
        L_1 \in coNP \\
        L_1 \in NP \\
        \forall L' \in coNP: L' \le_p L_1 
    \end{cases}$$

    Тоді
    $$\forall L' \in coNP: L' \le_p L_1 \in NP$$
    Тобто $\forall L' \in coNP: L' \in NP$
    \begin{mdframed}
        $$coNP \subseteq NP$$
    \end{mdframed}

    Відомо, що $$L_1 \le L_2 \implies coL_1 \le coL_2$$

    % $$\forall L' \in NP: \exists L''\in coNP: L' = coL''$$
    $$\forall L' \in coNP: L''=coL' \le_p coL_1 \in NP, coNP$$
    $$\forall L'' \in NP: L'' \le_p coL_1 \in NP, coNP$$
    Тобто $\forall L'' \in NP: L'' \in coNP$
    \begin{mdframed}
        $$NP \subseteq coNP$$
    \end{mdframed}
\end{proof}

\section*{№ 5.10}
\begin{mdframed}
    $$L_1 \in NPC$$
    Prove:
    $$L_2 = 1\cdot L_1 \in NPC$$
\end{mdframed}
\begin{proof}
    Для цього побудую P-зведення $L_1$ до $L_2$, бо цього достатньо.

    Нехай $M_{L2}$ розпізнає $L_2$.
    
    Тоді $M_1(w) = M_{L2}(1w)$ буде розпізнавати $L_1$.
\end{proof}

\end{document}