\section*{Ex 1.9}
\begin{mdframed}
	\textbf{Prove or disprove:}
	\begin{enumerate}
        \item $2n = \mathcal{O}(n)$;
        \item $n^2 = O(n)$;
        \item $n^2 = O(n \log^2 n)$;
        \item $2^{2^n} = O(2^{2^n})$;
        \item $20 = O(1)$;
        \item $n = o(2n)$;
        \item $2n = o(n^2)$;
        \item $1 = o(1/n)$;
        \item $20 = o(1)$;
    \end{enumerate}
\end{mdframed}

\subsection*{Definitions}
\begin{enumerate}
    \item $f = O(g) \wedge g = \Omega(f)  \iff \exists c :\; f(n) \le c \cdot g(n)$ after some $n$
    \item $f = \Theta(g) \iff f = O(g) \wedge g = O(f)$
    \item $f = o(g) \wedge g = \omega(f) \iff \forall \varepsilon > 0 \; f(n) \le \varepsilon \cdot g(n)$ after some $n$
\end{enumerate} 

\subsection*{Problems}
\begin{enumerate}
    \item $2n = \mathcal{O}(n)$ ?
        let $c = 2$. $2n \le 2 \cdot n. $. \\ \qedsymbol \quad \textbf{True}

    \item $n^2 = O(n)$;
        let $\exists c :\; n^2 \le c \cdot n$. 
        Then $n^2 / n = n \le c$. Contradiction - for every $c$ we can make it false with sufficiently large $n$. \\
        \qedsymbol \quad \textbf{False}

    \item $n^2 = O(n \log^2 n)$ ?
        \begin{flalign*}
            n^2 & \le C \cdot n \ln^2 n \\
            n  & \le C \cdot \ln^2 n \quad(\text{Let}\; n = e^{m}) \\
            e^m  & \le C \cdot m^2
        \end{flalign*}
        That is just false. $e^m > C m^2$ for $m > 4$
        \qedsymbol \quad \textbf{False}

    \item $2^{2^n} = O(2^{2^n})$ ?
        I'll prove more general statement.
        \[\forall n:\; f(n) \le 1 \cdot f(n)\]
        So, $f = O(f)$ with $n_0 = 1, C = 1$. Consequently, $2^{2^n} = O(2^{2^n})$ \\
        \qedsymbol \quad \textbf{True}

    \item $20 = O(1)$ ?
        \[20 \le 20 \cdot 1\]
        \qedsymbol \quad \textbf{True}

    \item $n = o(2n)$ ?
        \[n \le 2 \cdot n\]
        \qedsymbol \quad \textbf{True}
    
    \item $2n = o(n^2)$ ?
        \[2n \le n^2\]
        \[n^2 - 2n = n(n-2) \ge 0 \implies \forall n > 2:\; 2n \le n^2\]
        \qedsymbol \quad \textbf{True}

    \item $1 = o(1/n)$ ?
        \[\lim_{n\to\infty} \frac{1}{1/n} = \lim_{n\to \infty}n = \infty \ne 0\]
        So, $1 \ne o(1/n)$

        From the other perspective, we need to check 
        \[\forall \varepsilon > 0, n > n_0: 1 \le \varepsilon \cdot 1/n\]
        That is, $n \le \varepsilon$. 
        In fact, $\forall \varepsilon \exists n:\; n > \varepsilon$ \\
        \qedsymbol \quad \textbf{False}

    \item $20 = o(1)$ ?
        \[\forall \varepsilon > 0, n > n_0: 20 \le \varepsilon \cdot 1 = \varepsilon\]
        Actually, $\exists \varepsilon > 0 :\; \forall n:\; 20 > \varepsilon $. For example, $\varepsilon = 19 < 20$.
        From the other side, $\lim_{n\to\infty}\frac{20}{1} = 20 \ne 0$
        \qedsymbol \quad \textbf{False}
        
\end{enumerate}
