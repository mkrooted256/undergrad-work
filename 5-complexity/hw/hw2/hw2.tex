% !TEX TS-program = xelatex
% !TEX encoding = UTF-8

\documentclass[11pt, a4paper]{article} % use larger type; default would be 10pt

\usepackage{fontspec} % Font selection for XeLaTeX; see fontspec.pdf for documentation
\defaultfontfeatures{Mapping=tex-text} % to support TeX conventions like ``---''
\usepackage{xunicode} % Unicode support for LaTeX character names (accents, European chars, etc)
\usepackage{xltxtra} % Extra customizations for XeLaTeX
\usepackage{tikz}

\setmainfont[Ligatures=TeX]{Garamond} % set the main body font (\textrm), assumes Charis SIL is installed
%\setsansfont{Deja Vu Sans}
\setmonofont[Ligatures=TeX]{Fira Code}

% other LaTeX packages.....
\usepackage{fullpage}
\usepackage[top=2cm, bottom=4.5cm, left=2.5cm, right=2.5cm]{geometry}
\usepackage{amsmath,amsthm,amsfonts,amssymb,amscd,systeme}
\geometry{a4paper} 
%\usepackage[parfill]{parskip} % Activate to begin paragraphs with an empty line rather than an indent
\usepackage{fancyhdr}
\usepackage{listings}
\usepackage{graphicx}
\usepackage{hyperref}
\usepackage{multicol}

\renewcommand\lstlistingname{Algorithm}
\renewcommand\lstlistlistingname{Algorithms}
\def\lstlistingautorefname{Alg.}
\lstdefinestyle{mystyle}{
    % backgroundcolor=\color{backcolour},   
    % commentstyle=\color{codegreen},
    % keywordstyle=\color{magenta},
    % numberstyle=\tiny\color{codegray},
    % stringstyle=\color{codepurple},
    basicstyle=\ttfamily\footnotesize,
    breakatwhitespace=false,         
    breaklines=true,                 
    captionpos=b,                    
    keepspaces=true,                 
    numbers=left,                    
    numbersep=5pt,                  
    showspaces=false,                
    showstringspaces=false,
    showtabs=false,                  
    tabsize=2
}
\lstset{style=mystyle}

\newcommand\course{5 - Теорія складності}
\newcommand\hwnumber{ДЗ №2}                   % <-- homework number
\newcommand\idgroup{ФІ-91}                
\newcommand\idname{Михайло Корешков}  

\usepackage[framemethod=TikZ]{mdframed}
\mdfsetup{%
	backgroundcolor = black!5,
}
\mdfdefinestyle{ans}{%
    backgroundcolor = green!5,
    linecolor = green!50,
    linewidth = 1pt,
}

\pagestyle{fancyplain}
\headheight 35pt
\lhead{\idgroup \\ \idname}
\chead{\textbf{\Large \hwnumber}}
\rhead{\course \\ \today}
\lfoot{}
\cfoot{}
\rfoot{\small\thepage}
\headsep 1.5em

\linespread{1.2}

\begin{document}

\section*{№ 2.1}
Спочатку погрався з операціями, подивився що взагалі можливо.
Згодом відчув, що краще всього робити три послідовні операції
$$10 = f(v_1,g(v_2, h(v_3,v_4)))$$
бо інакше можливо використати лише дві
$$10 = f(g(v_1, v_2),h(v_3,v_4))$$
а цього, недостатньо ні для чого, схожого на 10

Отож, після ще 5 хвилин роздумів, я вирішив перебрати Python'ом всі
підходящі нам вирази вигляду 
$$f(v_1,g(v_2, h(v_3,v_4)))$$

Таким чином знайшов одну відповідь:
\begin{mdframed}[style=ans]
    $$10 = \frac{8}{1-\frac{1}{5}}$$    
\end{mdframed}

\section*{№ 2.2}
\begin{align*}
    & A = |\overline{\text{Black}}| = 14 \\  
    & B = |\overline{\text{White}}| = 16 \\  
    & C = |\overline{\text{Red}}| = 24 \\  
    & D = |\overline{\text{Green}}| = 12
\end{align*}

$|A \cup B \cup C \cup D|$ - кількість всіх кульок.

\begin{align*}
    |A \cup B \cup C \cup D| &= |A| + |B| + |C| + |D| - \\
        &- |A\cap B| - |A \cap C| - |A \cap D| - |B \cap C| - |B \cap D| - |C \cap D| + \\
        &+ |A \cap B \cap C| + |A \cap B \cap D| + |A \cap C \cap D| + |B \cap C \cap D| -\\
        &- |A\cap B \cap C\cap D| 
\end{align*}

\begin{itemize}
    \item $|A\cap B \cap C\cap D| = 0$, бо там не може бути кулька не червоного, не чорного, не білого та не зеленого кольору одночасно
    \item $|A \cap B \cap C| = |\text{Green}|$
    \item $|A \cap B \cap D| = |\text{Red}|$
    \item $|A \cap C \cap D| = |\text{White}|$
    \item $|B \cap C \cap D| = |\text{Black}|$
    \item $|A\cap B| = |\text{Red}| + |\text{Green}|$
    \item $|A\cap C| = |\text{White}| + |\text{Green}|$
    \item $|A\cap D| = |\text{White}| + |\text{Red}|$
    \item $|B\cap C| = |\text{Black}| + |\text{Green}|$
    \item $|B\cap D| = |\text{Black}| + |\text{Red}|$
    \item $|C\cap D| = |\text{Black}| + |\text{White}|$
\end{itemize}

$$|U| = |\text{White}| + |\text{Green}| + |\text{Black}| + |\text{Red}|$$
$$U = A \cup B \cup C \cup D$$


\begin{align*}
    |A \cup B \cup C \cup D| &= |A| + |B| + |C| + |D| - \\
        &- |\text{Red}| - |\text{Green}| - |\text{White}| - |\text{Green}| -\\
        &- |\text{White}| - |\text{Red}| - |\text{Black}| - |\text{Green}| - |\text{Black}| - |\text{Red}| - |\text{Black}| - |\text{White}| + \\
        &+ |\text{Green}| + |\text{Red}| + |\text{White}| + |\text{Black}| -\\
        &- 0
\end{align*}

\begin{align*}
    |U| &= |A| + |B| + |C| + |D| - \\
        &- 3|U| + \\
        &+ |U| -\\
        &- 0
\end{align*}

$$3|U| = |A| + |B| + |C| + |D|$$
\begin{mdframed}[style=ans]
    $$|U| = \frac{|A| + |B| + |C| + |D|}{3} = \frac{14 + 16 + 24 + 12}{3} = \frac{66}{3} = 22$$
\end{mdframed}


\section*{№ 2.3}



\end{document}

