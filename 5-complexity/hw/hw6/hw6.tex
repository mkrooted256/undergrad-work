    % !TEX TS-program = xelatex
% !TEX encoding = UTF-8

\documentclass[11pt, a4paper]{article} % use larger type; default would be 10pt

\usepackage{fontspec} % Font selection for XeLaTeX; see fontspec.pdf for documentation
\defaultfontfeatures{Mapping=tex-text} % to support TeX conventions like ``---''
\usepackage{xunicode} % Unicode support for LaTeX character names (accents, European chars, etc)
\usepackage{xltxtra} % Extra customizations for XeLaTeX
\usepackage{tikz}
\usetikzlibrary{arrows,calc,patterns}
\usetikzlibrary{decorations.pathreplacing,calligraphy}

\setmainfont[Ligatures=TeX]{Times New Roman} % set the main body font (\textrm), assumes Charis SIL is installed
%\setsansfont{Deja Vu Sans}
\setmonofont[Ligatures=TeX]{Fira Code}

% other LaTeX packages.....
\usepackage{fullpage}
\usepackage[top=2cm, bottom=4.5cm, left=2.5cm, right=2.5cm]{geometry}
\usepackage{amsmath,amsthm,amsfonts,amssymb,amscd,systeme}
\usepackage{cancel}
\geometry{a4paper} 
%\usepackage[parfill]{parskip} % Activate to begin paragraphs with an empty line rather than an indent
\usepackage{fancyhdr}
\usepackage{listings}
\usepackage{graphicx}
\usepackage{hyperref}
\usepackage{multicol}

\renewcommand\lstlistingname{Algorithm}
\renewcommand\lstlistlistingname{Algorithms}
\def\lstlistingautorefname{Alg.}
\lstdefinestyle{mystyle}{
    % backgroundcolor=\color{backcolour},   
    % commentstyle=\color{codegreen},
    % keywordstyle=\color{magenta},
    % numberstyle=\tiny\color{codegray},
    % stringstyle=\color{codepurple},
    basicstyle=\ttfamily\footnotesize,
    breakatwhitespace=false,         
    breaklines=true,                 
    captionpos=b,                    
    keepspaces=true,                 
    numbers=left,                    
    numbersep=5pt,                  
    showspaces=false,                
    showstringspaces=false,
    showtabs=false,                  
    tabsize=2
}
\lstset{style=mystyle}

\newcommand\course{5 - Complexity}
\newcommand\hwnumber{ДЗ №6}                   % <-- homework number
\newcommand\idgroup{ФІ-91}                
\newcommand\idname{Михайло Корешков}  

\usepackage[framemethod=TikZ]{mdframed}
\mdfsetup{%
	backgroundcolor = black!5,
}
\mdfdefinestyle{ans}{%
    backgroundcolor = green!5,
    linecolor = green!50,
    linewidth = 1pt,
}

\pagestyle{fancyplain}
\headheight 35pt
\lhead{\idgroup \\ \idname}
\chead{\textbf{\Large \hwnumber}}
\rhead{\course \\ \today}
\lfoot{}
\cfoot{}
\rfoot{\small\thepage}
\headsep 1.5em

\linespread{1.2}

\begin{document}
\section*{№ 6.1}
Виконуємо $M_1$, але додатково запишемо в двійковому вигляді довжину вхідного слова. 

\section*{№ 6.2}
\begin{itemize}
    \item Додавання та віднімання використовує log n пам'яті + накладні ресурси того ж порядку (наприклад на біти переносу).
    \item Множення можна представити через додавання |w| разів + біти переносу. Не більше c log n памяті.
    \item Два рядки однакові - ведемо бінарні лічильники номерів символа, який порівнюємо. 
    Просто ходимо туди-сюди по вхідній стрічці, станами запам'ятовуємо символ, ведемо в лічильниках номер поточного символа та номер символа, до якого треба дійти.
    \item substring - також ведемо лічильники, як в попередньому пункті. просто перевіряємо всі можливі місця, де рядки можуть бути однакові
    \item 
\end{itemize}

\section*{№ 6.4}
\begin{itemize}
    \item Якщо зустрічаємо 0 після першої одиниці - reject
    \item Якщо пустий вхід - reject
    \item Рахуємо кількість нулів бінарним лічильником
    \item Після першої одиницi зменшуємо лічильник на 1 на кожній одиниці
    \item Якщо в результаті лічильник = 0, то все ок, accept
    \item Інакше reject
\end{itemize}

\section*{№ 6.5}
\begin{itemize}
    \item Суть в тому, щоб слідкувати за тим, щоб кількість закритих дужок була не більша за кількість відкритих в кожному префіксі.
    \item Ведемо двійковий лічильник. +1 на $($, -1 на $)$
    \item Якщо в якийсь момент менше нуля - reject
    \item Якщо в результаті 0 - accept
    \item Інакше - reject
\end{itemize}

\section*{№ 6.6}
Те саме, що й в 6.5, але два лічильники.

Єдине що, додались заборонені комбінації дужок. Наприклад $[)$ та $(]$





\end{document}