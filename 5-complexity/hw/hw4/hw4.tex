    % !TEX TS-program = xelatex
% !TEX encoding = UTF-8

\documentclass[11pt, a4paper]{article} % use larger type; default would be 10pt

\usepackage{fontspec} % Font selection for XeLaTeX; see fontspec.pdf for documentation
\defaultfontfeatures{Mapping=tex-text} % to support TeX conventions like ``---''
\usepackage{xunicode} % Unicode support for LaTeX character names (accents, European chars, etc)
\usepackage{xltxtra} % Extra customizations for XeLaTeX
\usepackage{tikz}
\usetikzlibrary{arrows,calc,patterns}
\usetikzlibrary{decorations.pathreplacing,calligraphy}

\setmainfont[Ligatures=TeX]{Garamond} % set the main body font (\textrm), assumes Charis SIL is installed
%\setsansfont{Deja Vu Sans}
\setmonofont[Ligatures=TeX]{Fira Code}

% other LaTeX packages.....
\usepackage{fullpage}
\usepackage[top=2cm, bottom=4.5cm, left=2.5cm, right=2.5cm]{geometry}
\usepackage{amsmath,amsthm,amsfonts,amssymb,amscd,systeme}
\usepackage{cancel}
\geometry{a4paper} 
%\usepackage[parfill]{parskip} % Activate to begin paragraphs with an empty line rather than an indent
\usepackage{fancyhdr}
\usepackage{listings}
\usepackage{graphicx}
\usepackage{hyperref}
\usepackage{multicol}

\renewcommand\lstlistingname{Algorithm}
\renewcommand\lstlistlistingname{Algorithms}
\def\lstlistingautorefname{Alg.}
\lstdefinestyle{mystyle}{
    % backgroundcolor=\color{backcolour},   
    % commentstyle=\color{codegreen},
    % keywordstyle=\color{magenta},
    % numberstyle=\tiny\color{codegray},
    % stringstyle=\color{codepurple},
    basicstyle=\ttfamily\footnotesize,
    breakatwhitespace=false,         
    breaklines=true,                 
    captionpos=b,                    
    keepspaces=true,                 
    numbers=left,                    
    numbersep=5pt,                  
    showspaces=false,                
    showstringspaces=false,
    showtabs=false,                  
    tabsize=2
}
\lstset{style=mystyle}

\newcommand\course{5 - Complexity}
\newcommand\hwnumber{ДЗ №4}                   % <-- homework number
\newcommand\idgroup{ФІ-91}                
\newcommand\idname{Михайло Корешков}  

\usepackage[framemethod=TikZ]{mdframed}
\mdfsetup{%
	backgroundcolor = black!5,
}
\mdfdefinestyle{ans}{%
    backgroundcolor = green!5,
    linecolor = green!50,
    linewidth = 1pt,
}

\pagestyle{fancyplain}
\headheight 35pt
\lhead{\idgroup \\ \idname}
\chead{\textbf{\Large \hwnumber}}
\rhead{\course \\ \today}
\lfoot{}
\cfoot{}
\rfoot{\small\thepage}
\headsep 1.5em

\linespread{1.2}

\begin{document}

\section*{№ 4.2}
\begin{mdframed}
    \begin{multicols*}{2}
        $$L_1,L_2,L_3 \subseteq \{0,1\}*$$
        $$L_1 \in R$$
        $$L_2 \in RE \setminus R$$
        $$L_3 \in NRNC = ALL \setminus (RE \cup coRE)$$
    \columnbreak
        $$L'_1, L'_2, L'_3 \subseteq \{0,1\}*$$
        $$L'_1 = 0 \cdot L_1$$
        $$L'_2 = 0 \cdot L_2$$
        $$L'_3 = 0 \cdot L_3$$
    \end{multicols*}
$$L'_1, L'_2, L'_3 \in \; ?$$
\end{mdframed}

$$L'_1 = \{0x : x\in L_1\} \wedge L_1 \in R$$
Перевірка $x \in L_1$ - вирішувана. Перевірка, чи є перший символ 0 - вирішувана.
Значить, перевірка $w \in L'_1$ - вирішувана.\\
$L'_1 \in R$

З іншого боку:
$$\text{Let } L_0 = \{0x : x \in \{0,1\}*\};\quad L_0 \in R$$
Клас $R$ замкнений відносно перетину.
$$\begin{cases}
    L'_1 = L_0 \cap L_1\\
    L_0 \in R\\
    L_1 \in R
\end{cases} \implies L'_1 \in R$$

$$L'_2 = \{0x : x\in L_2\} \wedge L_2 \in RE\setminus R$$
$$L_0 \in R \implies L_0 \in RE$$
$$\begin{cases}
    L'_2 = L_0 \cap L_2\\
    L_0 \in RE\\
    L_2 \in RE
\end{cases} \implies L'_2 \in RE$$

$L'_2 \notin R$, бо інакше машиною $M_{L_2}(x) = M_{L'_2}(0x)$ можна вирішити мову $L_2$ (протиріччя).


$$L'_3 = \{0x : x\in L_3\} \wedge L_3 \in NRNC$$
Припустимо, що $L'_3 \in RE$.
Тоді можна побудувати $M_{L_3}(x) = M_{L'_3}(0x)$ - розпізнавач $L_3$. Протиріччя. 

Припустимо, що $L'_3 \in coRE$.
Тоді можна побудувати $M_{\overline{L_3}}(x) = 1 - M_{\overline{L'_3}}(0x)$ - розпізнавач $\overline{L_3}$. 
Але $L_3 \notin coRE$ Протиріччя.

Значить, $L'_3 \in NRNC$

\begin{mdframed}[style=ans]
    $$L'_1 \in R$$
    $$L'_2 \in RE$$
    $$L'_3 \in NRNC$$
\end{mdframed}

\section*{№ 4.3}
\subsection*{1)}
\begin{mdframed}
    $$\exists? L_1, L_2 \in NRNC;\quad L_1\cup L_2 \in NRNC$$
\end{mdframed}
Нехай $L_{N} \in NRNC$.
Нехай $L_1 = 0\cdot L_N, L_2 = 1\cdot L_N$.
В попередньому завданні доведено, що ці мови в $NRNC$.

$$L_1 \cup L_2 = \{xy : x \in \{0,1\}, y \in L_{N}\} \in NRNC$$
Бо інакше видаливши перший символ слова можна вирішити \ розв'язати $L_N$.
\begin{mdframed}[style=ans]
    Існує
\end{mdframed}

\subsection*{2)}
\begin{mdframed}
    $$\exists? L_1, L_2 \in RE \setminus R;\quad L_x = L_1\cap L_2 \in R$$
\end{mdframed}
% Нехай 
% $$L_1, L_2 \in RE \setminus R \wedge L_x = L_1\cap L_2 \in R$$
% $$\overline{L_x} = \overline{L_1} \cup \overline{L_2} \in R \text{ з замкненості }R$$

Нехай $L_{1} = \{<M> : |L(M)| = 1\}$.
$L_{1} \in RE \setminus R$.

Нехай $L_{2} = \{<M> : |L(M)| = 2\}$.
$L_{2} \in RE \setminus R$.

$$L_1 \cap L_2 = \varnothing = L_x$$
$$L_x \in R$$
бо $L_x$ вирішує МТ, що відхиляє всі слова.
\begin{mdframed}[style=ans]
    Існує
\end{mdframed}

\section*{№ 4.4}
\subsection*{a)}
\begin{mdframed}
    $$A, B \subseteq 2^{\{0,1\}^*}$$
    $$A\cup B = A \vee B$$
\end{mdframed}

$$A = \{0 \cdot L_0 : L_0 \in ALL \setminus \{\varnothing\}\}$$
$$B = \{1 \cdot L_0 : L_1 \in ALL \setminus \{\varnothing\}\}$$
$$C = \{\{\varepsilon\}, \varnothing\}$$

$$A \cup B = ALL \setminus C$$
$$A \vee B = \{L_1\cup L_2 : \forall x \in \{0,1\}^*: 0x \in L_1 \wedge 1x \in L_2\} = $$
$$= \{L_0 : \forall x \in \{0,1\}^*: 0x \in L_0 \vee 1x \in L_0\} = $$
$$= ALL \setminus C = A\cup B$$


\subsection*{б)}
\begin{mdframed}
    $$A, B \subseteq 2^{\{0,1\}^*}$$
    $$A\cap B = A \wedge B$$
\end{mdframed}

$$A = ALL$$
$$B = \{\varnothing\}$$
$$A \cap B = \{\varnothing\}$$
$$A \wedge B = \{L_1\cap L_2 : L_1\in ALL, L_2 = \varnothing\} = \{\varnothing\}$$

% \subsection*{в)}
% \begin{mdframed}
%     $$A \subseteq 2^{\{0,1\}^*}$$
%     $$\overline{A} = coA$$
% \end{mdframed}

% $$\overline{coA} = A$$
% $$ALL \setminus \{2^{\{0,1\}^*} \setminus L_0 : L_0 \in A\} = A$$

\section*{№ 4.5}
\begin{mdframed}
    $$C_1 = \{L_1 : \text{другий символ кожного слова мови дорiвнює 1}\}$$
    $$C_2 = \{L_1 : \text{третій символ кожного слова мови дорiвнює 0}\}$$
\end{mdframed}

$$C_1 \cap C_2 = \{L_1 : \text{другий символ = 1, третій символ = 0, довжина >= 3}\}$$

$$C_1 \wedge C_2 = C_1 \cap C_2 \cap \{\varnothing\}$$
- бо $\{01, 11\} \cap \{010, 110\} = \varnothing$

$$C_1 \cup C_2 = \{L_1 : \text{другий символ = 1 АБО третій символ = 0}\}$$
$$C_1 \vee C_2 = \{L_1 : \text{другий символ = 1 АБО третій символ = 0}\}$$

$$coC_1 = \{L_1 : \text{другий символ кожного слова мови \textbf{не} дорiвнює 1}\}$$
$$coC_2 = \{L_1 : \text{третій символ кожного слова мови \textbf{не} дорiвнює 0}\}$$

$$\overline{C_1} = 2^{\{\varepsilon, 1, 0\}} \cup coC_1$$
$$\overline{C_2} = 2^{\{\varepsilon, 1, 0, 00, 01, 10, 11\}} \cup coC_2$$

\section*{№ 4.6}
\begin{mdframed}
    Доведiть, що мова $HALT_{TM}$ є невирiшуваною, використовуючи
необчислюванiсть функцiй Радо
\end{mdframed}

Нехай $HALT_{TM}$ є вирiшуваною. 
Будуємо МТ $M_{1}(M)$ таку, що повертає, скільки кроків виконає машина М на порожньому вхідному слові
або повертає 0 якщо M зациклюється (ця перевірка можлива, бо $HALT_{TM}$ є вирiшуваною).

Будуємо МТ $M_{2,n,m}$ таку, що перевіряє, чи є МТ така, що зробить більше ніж n кроків.
Вона просто дає на вхід $M_1$ послідовно всі МТ з $m$ станами. 
Якщо знайшли таку МТ - повертаємо 1 та завершуємо роботу. Інакше природьо зациклюємося.

$M_{3,m}(n) = (1-M_{HALT}(M_{2,n,m}, n))$ - перевіряє, чи є число $n$ максимальною кількістю кроків для класу $K_{BB}(m)$.
Якщо $M_{2,n,m}$ зациклюється, то не існує МТ з більшою кількістю кроків на $\varepsilon$.

Запускаємо $M_3$ послідовно на всіх числах від 1 до нескінченності.
Рано чи пізно знайдемо значення функції.

Тобто, функція Радо - обчислювана. протиріччя. Значить, $HALT_{TM}$ є невирiшуваною.

\section*{№ 4.7}

\section*{№ 4.8}
% \begin{mdframed}
%     Довести:

%     Нехай $L_1$ - повна відносно зведення $r$ в классі $C_1$.
    
%     Тоді $\overline{L_1}$ - повна відносно зведення $r$ в классі $coC_1$
% \end{mdframed}

% $$r \subseteq 2^{\{0,1\}^*} \times 2^{\{0,1\}^*}$$
% $$\forall L_2 \in 2^{\{0,1\}^*}: (L_2,L_2) \in 2^{\{0,1\}^*}$$
% $$r - \text{транзитивне}$$

% $$L_1 \in C_1 \wedge \forall L_2 \in C_1: L_2 \le_r L_1$$

% $$coC_1 = \{\overline{L_2} : L_2 \in C_1\}$$

% $$\forall L_2 \in coC_1: \exists L_3 \in C_1: L_2 = \overline{L_3}$$

\section*{№ 4.9}


\end{document}