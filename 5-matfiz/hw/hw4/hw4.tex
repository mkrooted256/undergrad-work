% !TEX TS-program = xelatex
% !TEX encoding = UTF-8

\documentclass[11pt, a4paper]{article} % use larger type; default would be 10pt

\usepackage{fontspec} % Font selection for XeLaTeX; see fontspec.pdf for documentation
\defaultfontfeatures{Mapping=tex-text} % to support TeX conventions like ``---''
\usepackage{xunicode} % Unicode support for LaTeX character names (accents, European chars, etc)
\usepackage{xltxtra} % Extra customizations for XeLaTeX
\usepackage{tikz}
\usetikzlibrary{arrows,calc,patterns}

\setmainfont[Ligatures=TeX]{Garamond} % set the main body font (\textrm), assumes Charis SIL is installed
%\setsansfont{Deja Vu Sans}
\setmonofont[Ligatures=TeX]{Fira Code}

% other LaTeX packages.....
\usepackage{fullpage}
\usepackage[top=2cm, bottom=4.5cm, left=2.5cm, right=2.5cm]{geometry}
\usepackage{amsmath,amsthm,amsfonts,amssymb,amscd,systeme}
\usepackage{cancel}
\geometry{a4paper} 
%\usepackage[parfill]{parskip} % Activate to begin paragraphs with an empty line rather than an indent
\usepackage{fancyhdr}
\usepackage{listings}
\usepackage{graphicx}
\usepackage{hyperref}
\usepackage{multicol}

\renewcommand\lstlistingname{Algorithm}
\renewcommand\lstlistlistingname{Algorithms}
\def\lstlistingautorefname{Alg.}
\lstdefinestyle{mystyle}{
    % backgroundcolor=\color{backcolour},   
    % commentstyle=\color{codegreen},
    % keywordstyle=\color{magenta},
    % numberstyle=\tiny\color{codegray},
    % stringstyle=\color{codepurple},
    basicstyle=\ttfamily\footnotesize,
    breakatwhitespace=false,         
    breaklines=true,                 
    captionpos=b,                    
    keepspaces=true,                 
    numbers=left,                    
    numbersep=5pt,                  
    showspaces=false,                
    showstringspaces=false,
    showtabs=false,                  
    tabsize=2
}
\lstset{style=mystyle}

\newcommand\course{5 - Матфіз}
\newcommand\hwnumber{ДЗ №4}                   % <-- homework number
\newcommand\idgroup{ФІ-91}                
\newcommand\idname{Михайло Корешков}  

\usepackage[framemethod=TikZ]{mdframed}
\mdfsetup{%
	backgroundcolor = black!5,
}
\mdfdefinestyle{ans}{%
    backgroundcolor = green!5,
    linecolor = green!50,
    linewidth = 1pt,
}

\pagestyle{fancyplain}
\headheight 35pt
\lhead{\idgroup \\ \idname}
\chead{\textbf{\Large \hwnumber}}
\rhead{\course \\ \today}
\lfoot{}
\cfoot{}
\rfoot{\small\thepage}
\headsep 1.5em

\linespread{1.2}

\begin{document}

%  N24

\title{Метод розділення змінних для одновимірного рівняння теплопровідності}
\date{}
\maketitle


\section*{№ 24}

\begin{mdframed}
    \begin{equation}
        \begin{cases}
            u_t=a^2 u_{xx} + tx -1, \quad 0<x<1,\; t>0;\\
            u_x(0,t) = t^2;\\
            u(1,t) = 1;\\
            u(x,0) = 1 - \cos \frac{\pi}{2}x
        \end{cases}    
    \end{equation}
\end{mdframed}

\section*{Загальний розв'язок однорідного рівняння}
\begin{mdframed}
    $$\begin{cases}
        u_t=a^2 u_{xx}, \quad 0<x<1,\; t>0;\\
        u_x(0,t) = t^2;\\
        u(1,t) = 1;\\
        u(x,0) = 1 - \cos \frac{\pi}{2}x
    \end{cases}$$
\end{mdframed}

Шукаємо розв'язок у вигляді
$$u(x,t) = X(x)T(t)$$

$$X(x)T'(t) = a^2 X''(x)T(t)\; | \cdot \frac{1}{a^2X(x)T(t)}$$
$$\frac{T'(t)}{a^2T(t)} = \frac{X''(x)}{X(x)} = -\lambda$$

Розв'язуємо рівняння для $T$:
$$T'(t)=-\lambda a^2 T(t)$$
$$T(t) = e^{-\lambda a^2 t}$$
$$T(0) = 1$$

Розв'язуємо рівняння для $X$:
$$\begin{cases}
    X''(x) + \lambda X(x) = 0;\\
    u_x(0,t) = t^2; \; | \cdot \frac{d}{dx}\\
    u(1,t) = 1;\; | \cdot \frac{d}{dx}\\
    u(x,0) = 1 - \cos \frac{\pi}{2}x\;
\end{cases}$$
$$\begin{cases}
    u_{xx}(0,t) &= X''(0)T(t) = 0;\\
    u_x(1,t) &= X'(1)T(t) = 0;\\
    u(x,0) &= X(x) = 1 - \cos \frac{\pi}{2}x\;\\
    u_x(x,0) &= X'(x) = \frac{\pi}{2} \sin \frac{\pi}{2}x
\end{cases}$$

$$X''(0)T(t) = 0 \implies X''(0) = 0 \implies -\lambda X(0) = 0 \implies X(0) = 0$$
$$X'(1)T(t) = 0 \implies X'(1) = 0$$

\begin{mdframed}
    Нехай $\lambda = -r^2 < 0$
\end{mdframed}
$$X''(x) = r^2 X(x)$$
$$X(x) = A\cosh rx + B\sinh rx$$
$$X(0) = 0 \implies A = 0$$
$$X'(1) = 0 \implies rB\cosh r = 0 \implies B = 0$$
Тобто, лише тривіальний власний вектор

\begin{mdframed}
    Нехай $\lambda = 0$
\end{mdframed}
$$X''(x) = 0$$
$$X(x) = cx + b$$
$$X(0) = b = 0$$
$$X'(1) = c = 0$$
Тобто, лише тривіальний власний вектор


\begin{mdframed}
    $\lambda = r^2 > 0$
\end{mdframed}
$$X''(x) + r^2 X(x) = 0$$
$$X(x) = A\cos rx + B\sin rx$$
$$X(0) = A = 0$$
$$X'(1) = rB\cos r = 0 \implies \cos r = 0$$

Маємо:
$$\lambda_n = r_n^2 = \left(\frac{\pi}{2} (2n+1)\right)^2 = \frac{\pi^2}{4}(2n+1)^2$$

Тобто,
$$u(x,t) = \sum_{n=0}^\infty c_n e^{-\frac{\pi^2}{4}(2n+1)^2 a^2 t}\cos \frac{\pi}{2} (2n+1) x$$

Умова $$u(x,0) = 1 - \cos \frac{\pi}{2} x;$$
$$u(x,0) = \sum_{n=0}^\infty c_n \cos \frac{\pi}{2} (2n+1) x = 1 - \cos \frac{\pi}{2} x$$

\end{document}
 
