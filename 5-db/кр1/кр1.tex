% !TEX TS-program = xelatex
% !TEX encoding = UTF-8

\documentclass[11pt, a4paper]{article} % use larger type; default would be 10pt

\usepackage{fontspec} % Font selection for XeLaTeX; see fontspec.pdf for documentation
\defaultfontfeatures{Mapping=tex-text} % to support TeX conventions like ``---''
\usepackage{xunicode} % Unicode support for LaTeX character names (accents, European chars, etc)
\usepackage{xltxtra} % Extra customizations for XeLaTeX
\usepackage{tikz}
\usetikzlibrary{arrows,calc,patterns}

\setmainfont[Ligatures=TeX]{Garamond} % set the main body font (\textrm), assumes Charis SIL is installed
%\setsansfont{Deja Vu Sans}
\setmonofont[Ligatures=TeX]{Fira Code}

% other LaTeX packages.....
\usepackage{fullpage}
\usepackage[top=2cm, bottom=4.5cm, left=2.5cm, right=2.5cm]{geometry}
\usepackage{amsmath,amsthm,amsfonts,amssymb,amscd,systeme}
\usepackage{cancel}
\geometry{a4paper} 
%\usepackage[parfill]{parskip} % Activate to begin paragraphs with an empty line rather than an indent
\usepackage{fancyhdr}
\usepackage{listings}
\usepackage{graphicx}
\usepackage{hyperref}
\usepackage{multicol}

\renewcommand\lstlistingname{Algorithm}
\renewcommand\lstlistlistingname{Algorithms}
\def\lstlistingautorefname{Alg.}
\lstdefinestyle{mystyle}{
    % backgroundcolor=\color{backcolour},   
    % commentstyle=\color{codegreen},
    % keywordstyle=\color{magenta},
    % numberstyle=\tiny\color{codegray},
    % stringstyle=\color{codepurple},
    basicstyle=\ttfamily\footnotesize,
    breakatwhitespace=false,         
    breaklines=true,                 
    captionpos=b,                    
    keepspaces=true,                 
    numbers=left,                    
    numbersep=5pt,                  
    showspaces=false,                
    showstringspaces=false,
    showtabs=false,                  
    tabsize=2
}
\lstset{style=mystyle}

\newcommand\course{5 - Бази Даних}
\newcommand\hwnumber{КР 1}                   % <-- homework number
\newcommand\idgroup{ФІ-91}                
\newcommand\idname{Михайло Корешков}  

\usepackage[framemethod=TikZ]{mdframed}
\mdfsetup{%
	backgroundcolor = black!5,
}
\mdfdefinestyle{ans}{%
    backgroundcolor = green!5,
    linecolor = green!50,
    linewidth = 1pt,
}

\pagestyle{fancyplain}
\headheight 35pt
\lhead{\idgroup \\ \idname}
\chead{\textbf{\Large \hwnumber}}
\rhead{\course \\ \today}
\lfoot{}
\cfoot{}
\rfoot{\small\thepage}
\headsep 1.5em

\linespread{1.2}

\begin{document}

\section*{Опис предметної області}
Відношення моделюють роботу міжнародної фірми, що має кілька філій. Філії фірми можуть бути розташовані в різних країнах, це відображено в відношенні R1. Клієнти фірми також можуть бути з різних країн, і це відображено в відношенні R4. По кожному конкретному замовленню клієнт міг замовити кілька різних товарів. 

\begin{mdframed}
    $$R_1 = (\text{Філія}, \text{Країна})$$
    $$R_2 = (\text{Філія}, \text{Замовник}, \text{№ Замовлення})$$
    $$R_3 = (\text{№ Замовлення}, \text{Товар}, \text{Кількість})$$
    $$R_4 = (\text{Замовник}, \text{Країна})$$
\end{mdframed}

\section*{1) Філіали фірми, які торгують всіма товарами.}

Щоб дізнатися, якими товарами торгує філія, дивимося на замовлення.
Обираємо всі можливі \\ замовлення та зв'язані з ними філії:
$$R_{11} = \left(R_2[\;R_2[\text{№ Замовлення}] = R_3[\text{№ Замовлення}]\;]R_3\right)[\text{Філія},\;\text{Товар}]$$
Тепер отримаємо список всіх філій:
$$R_{12} = R_1[\text{Філія}]$$
Залишається застосувати ділення відношень, щоб отримати всі філії, що мають всі товари:
$$A_1 = R_{11} \div R_{12}$$

\section*{2) Філії, з якими не працює жоден замовник. }

$R_2$ містить інфу про те, з якими замовниками працюють філії.
В $R_1$ список всіх філій. \\
Знайдемо філії з хоча б одним замовником:
$$R_{21} = R_2[\text{Філія}]$$
Віднімемо зі списку всіх філій $R_{21}$. Отримаємо філії без замовників:
$$A_2 = R_1[\text{Філія}] \setminus R_{21}$$

\section*{3) Замовників, які працюють з усіма філіями фірми, але купують тільки один товар.}
Спочатку знайдемо замовників, що працюють з усіма філіями:
$$R_{31} = R_2[\text{Філія}, \text{Замовник}] \div R_1[\text{Філія}] $$
Тепер шукаємо замовників, що купують лише 1 товар. 
Спочатку зробімо список товарів для кожного замовника:
$$R_{32} = (R_2[\;R_2[\text{№ Замовлення}] = R_3[\text{№ Замовлення}]\;]R_3)[\text{Замовник},\text{Товар}]$$
Тепер дивимося, які замовники купують більше 1 товару:
$$R_{33} = (R_{32}[\;R_{32}[\text{Замовник}]=R'_{32}[\text{Замовник}]\;\wedge\;
R_{32}[\text{Товар}]<>R'_{32}[\text{Товар}]\;]R'_{32})[\text{Замовник}]$$

Все готово для відповіді:
$$A_3 = R_{31} \cap (R_2[\text{Замовник}] \setminus R_{33})$$

\section*{4) Замовників, які працюють з філіями фірми, які розташовані тільки в одній країні.}
Знайдемо з якими країнами працюють замовники:
$$R_{41} = R_1[\;R_1[\text{Філія}]=R_2[\text{Філія}]\;]R_2[\text{Замовник},\text{Країна}]$$

Шукаємо всіх замовників, що працюють лише з філіями з однією країною 
(віднімемо від всіх замовників тих, хто працює більш ніж з 1 країною):
$$A_4 = R_2[\text{Замовник}] \;\setminus $$
$$\setminus \;(R_{41}[\;R_{41}[\text{Замовник}]=R'_{41}[\text{Замовник}]\;\wedge\;
R_{41}[\text{Країна}] <> R'_{41}[\text{Країна}]\;]R'_{41})[\text{Замовник}]$$

\section*{5) Замовників, які працюють тільки з філіями, розташованими в тій же країні, що і замовник.}
Спочатку знайдемо з якими філіями можуть працювати замовники, що нас цікавлять:
$$R_{51} = (R_1[\;R_1[\text{Країна}]=R_4[\text{Країна}]\;]R_4)[\text{Замовник}, \text{Філія}]$$
Тепер шукаємо всіх замовників, що працюють з філіями, з якими вони \textbf{не} можуть працювати:
$$R_{52} = (R_{41}\setminus R_{51})[\text{Замовник}]$$
Відповідь - всі інші замовники (ті, що не працюють з філіями, з якими не можна працювати):
$$A_5 = R_4[\text{Замовник}] \;\setminus\; R_{52}$$


\end{document}

