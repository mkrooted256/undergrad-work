% !TeX program = xelatex
% !TEX encoding = UTF-8

\documentclass[11pt, a4paper]{article} % use larger type; default would be 10pt

\usepackage{fontspec} % Font selection for XeLaTeX; see fontspec.pdf for documentation
\defaultfontfeatures{Mapping=tex-text} % to support TeX conventions like ``---''
\usepackage{xunicode} % Unicode support for LaTeX character names (accents, European chars, etc)
\usepackage{xltxtra} % Extra customizations for XeLaTeX
\usepackage{tikz}
\usetikzlibrary{arrows,calc,patterns}


% other LaTeX packages.....
\usepackage{fullpage}
\usepackage[top=2cm, bottom=4.5cm, left=2.5cm, right=2.5cm]{geometry}
\usepackage{amsmath,amsthm,amsfonts,amssymb,amscd,systeme}
\usepackage{unicode-math}
\usepackage{cancel}
\geometry{a4paper} 
%\usepackage[parfill]{parskip} % Activate to begin paragraphs with an empty line rather than an indent
\usepackage{fancyhdr}
\usepackage{listings}
\usepackage{graphicx}
\usepackage{hyperref}
\usepackage{multicol}

% FONTS
\setmainfont[Ligatures=TeX]{Cambria Math} % set the main body font (\textrm), assumes Charis SIL is installed
%\setsansfont{Deja Vu Sans}
\setmonofont[Ligatures=TeX]{Fira Code}
\setmathfont[Ligatures=TeX]{NewCMMath-Regular.otf}

\renewcommand\lstlistingname{Algorithm}
\renewcommand\lstlistlistingname{Algorithms}
\def\lstlistingautorefname{Alg.}
\lstdefinestyle{mystyle}{
    % backgroundcolor=\color{backcolour},   
    % commentstyle=\color{codegreen},
    % keywordstyle=\color{magenta},
    % numberstyle=\tiny\color{codegray},
    % stringstyle=\color{codepurple},
    basicstyle=\ttfamily\footnotesize,
    breakatwhitespace=false,         
    breaklines=true,                 
    captionpos=b,                    
    keepspaces=true,                 
    numbers=left,                    
    numbersep=5pt,                  
    showspaces=false,                
    showstringspaces=false,
    showtabs=false,                  
    tabsize=2
}
\lstset{style=mystyle}

\newcommand\course{6 - Функціональний аналіз}
\newcommand\hwnumber{ДЗ №2}                   % <-- homework number
\newcommand\idgroup{ФІ-91}                
\newcommand\idname{Михайло Корешков}  

\usepackage[framemethod=TikZ]{mdframed}
\mdfsetup{%
	backgroundcolor = black!5,
}
\mdfdefinestyle{ans}{%
    backgroundcolor = green!5,
    linecolor = green!50,
    linewidth = 1pt,
}

\pagestyle{fancyplain}
\headheight 35pt
\lhead{\idgroup \\ \idname}
\chead{\textbf{\Large \hwnumber}}
\rhead{\course \\ \today}
\lfoot{}
\cfoot{}
\rfoot{\small\thepage}
\headsep 1.5em

\linespread{1.2}

\begin{document}

% № 6.11 ( д), е), и), к) ). Треногин
% № 21 ( 1), 2), 3), 4) ). Антонович

\section*{№ 6.11}
\begin{mdframed}
    $x,y \in S$ - простір всіх послідовностей з натуральних чисел
    $$d(x,y) = \begin{cases}
        0, & x=y \\
        \frac{1}{k(x,y)}
    \end{cases}$$
    $k(x,y)$ - номер першого елемента, що відрізняється
\end{mdframed}

\subsection*{д) }
\begin{mdframed}
    Довести, що замкнена куля $\bar B(r,x)$ є також відкритою.  

    Довести, що $\forall y \in \bar B(r,x): \bar B(r,y) = \bar B(r,x)$
\end{mdframed}

\begin{proof}
    Множина відкрита якщо кожен елемент входить з деяким відкритим околом.
    
    Нехай $y \in \bar B(r,x)$. Тобто 
    $$\frac{1}{k(x,y)} = d(x,y) \le r $$
    $$\iff k(x,y) \ge \left\lceil \frac{1}{r} \right\rceil $$

    Відомо що $\forall r: \exists n\in \mathbb N: \frac{1}{n+1} < r \le \frac{1}{n}$.

    Нехай $z \in B(r, y)$. Тоді $k(z,y) > n $.

    \begin{multline*}
        \begin{cases}
            t = k(z,y) > n\\
            l = k(x,y) \ge n
        \end{cases} \implies \begin{cases}
            z = (y_1, y_2, ..., y_t, z_{t+1}, ...), t > n \\
            y = (x_1, x_2, ..., x_l, y_{l+1}, ...), l \ge n \\
            x = (x_1, x_2, ...)
        \end{cases} \implies \\ \implies k(z,x) \ge \min(t,l) \ge n \implies d(z, x) \le \frac{1}{n} \implies z \in \bar B(r,x) 
    \end{multline*}

\end{proof}

\begin{proof}
    Нехай $y \in \bar B(r,x)$. Тобто 
    $$\frac{1}{k(x,y)} = d(x,y) \le r $$
    $$\iff k(x,y) \ge \left\lceil \frac{1}{r} \right\rceil $$
    
    Нехай $z \in \bar B(r,y)$. $k(z,y) \ge \left\lceil \frac{1}{r} \right\rceil$.

    Тоді $k(x,z) \ge \min(k(x,y), k(y,z)) \ge \left\lceil \frac{1}{r} \right\rceil$. Тобто $z \in \bar B(r,x)$.

    Нехай $z \in \bar B(r,x)$. $k(z,x) \ge \left\lceil \frac{1}{r} \right\rceil$.

    Тоді $k(y,z) \ge \min(k(x,y), k(z,x)) \ge \left\lceil \frac{1}{r} \right\rceil$. Тобто $z \in \bar B(r,y)$.

    Маємо $\forall y \in \bar B(r,x): \bar B(r,x) \subseteq \bar B(r,y) \wedge \bar B(r,y) \subseteq \bar B(r,x)$.

    $\forall y \in \bar B(r,x): \bar B(r,x) = \bar B(r,y)$
    
\end{proof}

\subsection*{е)}
\begin{mdframed}
    Якщо дві кулі мають спільну точку, то одна міститься в іншій.
\end{mdframed}

\begin{proof}
    Нехай $r_1 < r_2$. Нехай $y \in B(r_1,x_1), B(r_2, x_2)$. 
    Тоді $\begin{cases}
        d(y,x_1) < r_1 \\
        d(y,x_2) < r_2
    \end{cases}$. 
    Вже довели нерівність трикутника в посиленій формі (в тому числі неявно в попередньому пункті):
    $$d(x,y) \le \max(d(x,z), d(y,z))$$
    це еквівалентно
    $$k(x,y) \ge \min(k(x,z), k(y,z))$$
    
    $$d(x_1, x_2) \le \max(d(y,x_1), d(y,x_2)) < \max(r_1, r_2) = r_2$$

    Нехай $z \in B(r_1, x_1)$. Тоді
    $$d(z, x_2) \le \max(d(z,x_1), d(x_1,x_2)) < \max(r_1, r_2) = r_2$$
    Тобто $z \in B(r_2, x_2)$.
    Тобто $B(r_1, x_1) \subseteq B(r_2, x_2)$
\end{proof}

\subsection*{и) Повнота простору}

\begin{proof}
    Нехай послідовність (послідовностей) $(x^n)_{n\ge 1}$ - фундаментальна.

    Тобто $d(x^n, x^m) \to 0$. Тобто $k(x^n, x^m) \to \infty$.
    $$\forall D>0: \exists N: \forall n,m >N: k(x^n, x^m) > D$$

    Візьмемо $x_{(k)} = (x_k^1, x_k^2, ...)$ - послідовність значень одного елементу.
    З попереднього тверження видно, що $\forall k: x_{(k)}$ - з якогось моменту стала послідовність.
    Тобто $\forall k: \exists \lim_{t\to\infty} x_k^t =: x_k$.
    
    Будуємо з $x_k$ послідовність $x$. Вона, очевидно, складається з натуральних чисел та міститься у $X$.

    $\lim_{n\to\infty} d(x^n, x) = \lim_{n\to\infty} \frac{1}{k(x^n,x)} = 0$

    Маємо, що фундаментальна послідовність - збіжна.
\end{proof}

\subsection*{к) Сепарабельність}
\begin{proof}
    У якості зліченної всюди щільної множини можна спробувати взяти множину всіх можливих 
    натуральних \textbf{фінітних} послідовностей.
    $$S = \{s = (s_1, ..., s_t, 0, 0, ...) : s_i, t \in \mathbb N\}$$
    $$|S| = |\bigcup_{i=1}^\infty \{(s_1, ..., s_i, 0, 0, ...)\}| = \aleph_0$$

    Нехай $x \in X, \epsilon > 0$. Шукаємо $s\in S: d(x,s) < \epsilon$.
    
    По-перше, нехай $t = \left\lceil\frac{1}{\epsilon}\right\rceil + 1$.
    Беремо $s = (x_1, x_2, ... ,x_t, 0, 0, ...) \in S$.
    $$d(s, x) = \frac{1}{k(s,x)} \le \frac{1}{\left\lceil\frac{1}{\epsilon}\right\rceil + 1} < \epsilon$$

    Тобто, елементи зліченної $S$ знайдуться в будь-якому околі будь-якого елементу $X$. 
    That is, $X$ - сепарабельний
\end{proof}

\section*{№ 3.21 (Антонович)}
\begin{mdframed}
    $p$ - деяке просте, $n$ - натуральне, $x$ - раціональне.
    Нехай $V_p(n) : n = \prod_p p^{V_p(n)}$.

    $$x = \pm\frac{m}{n} \implies V_p(x) := V_p(m) - V_p(n)$$
    $$V_p(xy) = V_p(x) + V_p(y)$$

    Нехай $d(x,y) = \begin{cases}
        0, & x=y \\
        p^{-V_p(x-y)}& x\ne y
    \end{cases}$
\end{mdframed}

\subsection*{1)}
\begin{proof}
    1. $d(x,y) = 0$ за визначенням.

    2. $V_p(x) = V_p(m) - V_p(n) = V_p(-x)$. Тоді $d(x,y) = p^{-V_p(x-y)} = p^{-V_p(-(x-y))}= d(y,x)$

    3. $d(x,y) + d(y,z) = p^{-V_p(x-y)} + p^{-V_p(y-z)}$

    Нехай 
    $$x = \frac{m(x)}{n(x)} \cdot p^a, y = \frac{m(y)}{n(y)} \cdot p^b, z = \frac{m(z)}{n(z)} \cdot p^c$$
    $$x-y = \frac{m(x)}{n(x)} \cdot p^a - \frac{m(y)}{n(y)} \cdot p^b 
    \le p^{\max(a,b)} \cdot (\frac{m(x)}{n(x)} - \frac{m(y)}{n(y)}) \le 
    $$
    В правій дужці множник $p$ може з'явитись лише в чисельнику та лише з додатнім ступенем.
    
\end{proof}

\subsection*{1)}


\end{document}

