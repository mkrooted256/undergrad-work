% !TeX program = xelatex
% !TEX encoding = UTF-8

\documentclass[11pt, a4paper]{article} % use larger type; default would be 10pt

\usepackage{fontspec} % Font selection for XeLaTeX; see fontspec.pdf for documentation
\defaultfontfeatures{Mapping=tex-text} % to support TeX conventions like ``---''
\usepackage{xunicode} % Unicode support for LaTeX character names (accents, European chars, etc)
\usepackage{xltxtra} % Extra customizations for XeLaTeX
\usepackage{tikz}
\usetikzlibrary{arrows,calc,patterns}

\setmainfont[Ligatures=TeX]{Cambria Math} % set the main body font (\textrm), assumes Charis SIL is installed
%\setsansfont{Deja Vu Sans}
\setmonofont[Ligatures=TeX]{Fira Code}

% other LaTeX packages.....
\usepackage{fullpage}
\usepackage[top=2cm, bottom=4.5cm, left=2.5cm, right=2.5cm]{geometry}
\usepackage{amsmath,amsthm,amsfonts,amssymb,amscd,systeme}
\usepackage{unicode-math}
\usepackage{cancel}
\geometry{a4paper} 
%\usepackage[parfill]{parskip} % Activate to begin paragraphs with an empty line rather than an indent
\usepackage{fancyhdr}
\usepackage{listings}
\usepackage{graphicx}
\usepackage{hyperref}
\usepackage{multicol}

\renewcommand\lstlistingname{Algorithm}
\renewcommand\lstlistlistingname{Algorithms}
\def\lstlistingautorefname{Alg.}
\lstdefinestyle{mystyle}{
    % backgroundcolor=\color{backcolour},   
    % commentstyle=\color{codegreen},
    % keywordstyle=\color{magenta},
    % numberstyle=\tiny\color{codegray},
    % stringstyle=\color{codepurple},
    basicstyle=\ttfamily\footnotesize,
    breakatwhitespace=false,         
    breaklines=true,                 
    captionpos=b,                    
    keepspaces=true,                 
    numbers=left,                    
    numbersep=5pt,                  
    showspaces=false,                
    showstringspaces=false,
    showtabs=false,                  
    tabsize=2
}
\lstset{style=mystyle}

\newcommand\course{6 - Функціональний аналіз}
\newcommand\hwnumber{ДЗ №1}                   % <-- homework number
\newcommand\idgroup{ФІ-91}                
\newcommand\idname{Михайло Корешков}  

\usepackage[framemethod=TikZ]{mdframed}
\mdfsetup{%
	backgroundcolor = black!5,
}
\mdfdefinestyle{ans}{%
    backgroundcolor = green!5,
    linecolor = green!50,
    linewidth = 1pt,
}

\pagestyle{fancyplain}
\headheight 35pt
\lhead{\idgroup \\ \idname}
\chead{\textbf{\Large \hwnumber}}
\rhead{\course \\ \today}
\lfoot{}
\cfoot{}
\rfoot{\small\thepage}
\headsep 1.5em

\linespread{1.2}

\begin{document}

% [1] № 6.2 (2, 3), [2] № 23, 24, 39 (доробити 1, зробити 2, 3, 4)

\section*{№6.2}
\begin{mdframed}
    Довести, що $d_2,d_3$ - метрики, якщо $d$ - метрика.
\end{mdframed}

\begin{proof}
    \[d_2(x,y) = \ln(1 + d(x,y))\]
    \begin{enumerate}
        \item \(1 + d(x,y) \ge 1 \implies d_2(x,y) = \ln (1 + d(x,y)) \ge 0\)
        \item \(d_2(x,x) = \ln(1 + d(x,x)) = \ln(1 + 0) = 0\)
        \item \(d_2(x,y) = \ln(1 + d(x,y)) = \ln(1 + d(y,x)) = d_2(y,x)\)
        \item Знаємо, що \(d(x,y) + d(y,z) \ge d(x,z)\). Тоді
        \begin{align*}
            d_2(x,y) + d_2(y,z) &= \ln(1 + d(x,y)) + \ln(1 + d(y,z)) = \ln\left[(1 + d(x,y))(1 + d(y,z))\right] = \\
            &= \ln\left[1 + d(x,y) + d(y,z) + d(x,y)d(y,z)\right] \ge\\
            &\ge \ln\left[1 + d(x,z) + d(x,y)d(y,z)\right] \ge \\
            &\ge \ln(1 + d(x,z)) = d_2(x,z)
        \end{align*}
    \end{enumerate}
\end{proof}
    
\begin{proof}
    \[d_3(x,y) = \min(1, d(x,y))\]
    \begin{enumerate}
        \item \(1, d(x,y) \ge 0 \implies d_3(x,y) = \min(1, d(x,y)) \ge 0\)
        \item \(d_3(x,x) = \min(1, 0) = 0\)
        \item \(d_3(x,y) = \min(1, d(x,y)) = \min(1, d(y,x)) = d_3(y,x)\)
        \item Знаємо, що \(d(x,z) \le d(x,y) + d(y,z)\). Тоді
        \[d_3(x,z) = \min(1, d(x,z)) = \begin{cases}
            1, & d(x,z) > 1 \\
            d(x,z), & d(x,z) \le 1
        \end{cases}\]
        \begin{align*}
            d(x,z) \le 1 \implies d_3(x,z) &= d(x,z) \le \min(1, d(x,y) + d(y,z)) \le \\
            &\le \min(1, d(x,y)) + \min(1, d(y,z)) = \\
            &= d_3(x,y) + d_3(y,z)
        \end{align*}
    \end{enumerate}
    
\end{proof}

\section*{№ 23}
\begin{mdframed}
    Визначити операції \(\cup, \cap\) через \(\triangle, \setminus\) 
\end{mdframed}

\begin{mdframed}
    Загальноприйняті визначення:
    \[A \triangle B = (A\setminus B) \cup (B \setminus A) = (A \cup B) \setminus (A \cap B) \]
    \[A \setminus B = A \cap \overline B\]
\end{mdframed}

\begin{align*}
    A \cup B = \{x : x \in A \vee x \in B\} =   
\end{align*}


\end{document}

