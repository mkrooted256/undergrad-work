% !TeX program = xelatex
% !TEX encoding = UTF-8

\documentclass[11pt, a4paper]{article} % use larger type; default would be 10pt

\usepackage{fontspec} % Font selection for XeLaTeX; see fontspec.pdf for documentation
\defaultfontfeatures{Mapping=tex-text} % to support TeX conventions like ``---''
\usepackage{xunicode} % Unicode support for LaTeX character names (accents, European chars, etc)
\usepackage{xltxtra} % Extra customizations for XeLaTeX
\usepackage{tikz}
\usetikzlibrary{arrows,calc,patterns}


% other LaTeX packages.....
\usepackage{fullpage}
\usepackage[top=2cm, bottom=4.5cm, left=2.5cm, right=2.5cm]{geometry}
\usepackage{amsmath,amsthm,amsfonts,amssymb,amscd,systeme}
\usepackage{unicode-math}
\usepackage{cancel}
\geometry{a4paper} 
%\usepackage[parfill]{parskip} % Activate to begin paragraphs with an empty line rather than an indent
\usepackage{fancyhdr}
\usepackage{listings}
\usepackage{graphicx}
\usepackage{hyperref}
\usepackage{multicol}

% FONTS
\setmainfont[Ligatures=TeX]{Cambria Math} % set the main body font (\textrm), assumes Charis SIL is installed
%\setsansfont{Deja Vu Sans}
\setmonofont[Ligatures=TeX]{Fira Code}
\setmathfont[Ligatures=TeX]{NewCMMath-Regular.otf}

\renewcommand\lstlistingname{Algorithm}
\renewcommand\lstlistlistingname{Algorithms}
\def\lstlistingautorefname{Alg.}
\lstdefinestyle{mystyle}{
    % backgroundcolor=\color{backcolour},   
    % commentstyle=\color{codegreen},
    % keywordstyle=\color{magenta},
    % numberstyle=\tiny\color{codegray},
    % stringstyle=\color{codepurple},
    basicstyle=\ttfamily\footnotesize,
    breakatwhitespace=false,         
    breaklines=true,                 
    captionpos=b,                    
    keepspaces=true,                 
    numbers=left,                    
    numbersep=5pt,                  
    showspaces=false,                
    showstringspaces=false,
    showtabs=false,                  
    tabsize=2
}
\lstset{style=mystyle}

\newcommand\course{6 - Функціональний аналіз}
\newcommand\hwnumber{ДЗ №1}                   % <-- homework number
\newcommand\idgroup{ФІ-91}                
\newcommand\idname{Михайло Корешков}  

\usepackage[framemethod=TikZ]{mdframed}
\mdfsetup{%
	backgroundcolor = black!5,
}
\mdfdefinestyle{ans}{%
    backgroundcolor = green!5,
    linecolor = green!50,
    linewidth = 1pt,
}

\pagestyle{fancyplain}
\headheight 35pt
\lhead{\idgroup \\ \idname}
\chead{\textbf{\Large \hwnumber}}
\rhead{\course \\ \today}
\lfoot{}
\cfoot{}
\rfoot{\small\thepage}
\headsep 1.5em

\linespread{1.2}

\begin{document}

% [1] № 6.2 (2, 3), [2] № 23, 24, 39 (доробити 1, зробити 2, 3, 4)

\section*{№6.2}
\begin{mdframed}
    Довести, що $d_2,d_3$ - метрики, якщо $d$ - метрика.
\end{mdframed}

\begin{proof}
    \[d_2(x,y) = \ln(1 + d(x,y))\]
    \begin{enumerate}
        \item \(1 + d(x,y) \ge 1 \implies d_2(x,y) = \ln (1 + d(x,y)) \ge 0\)
        \item \(d_2(x,x) = \ln(1 + d(x,x)) = \ln(1 + 0) = 0\)
        \item \(d_2(x,y) = \ln(1 + d(x,y)) = \ln(1 + d(y,x)) = d_2(y,x)\)
        \item Знаємо, що \(d(x,y) + d(y,z) \ge d(x,z)\). Тоді
        \begin{align*}
            d_2(x,y) + d_2(y,z) &= \ln(1 + d(x,y)) + \ln(1 + d(y,z)) = \ln\left[(1 + d(x,y))(1 + d(y,z))\right] = \\
            &= \ln\left[1 + d(x,y) + d(y,z) + d(x,y)d(y,z)\right] \ge\\
            &\ge \ln\left[1 + d(x,z) + d(x,y)d(y,z)\right] \ge \\
            &\ge \ln(1 + d(x,z)) = d_2(x,z)
        \end{align*}
    \end{enumerate}
\end{proof}
    
\begin{proof}
    \[d_3(x,y) = \min(1, d(x,y))\]
    \begin{enumerate}
        \item \(1, d(x,y) \ge 0 \implies d_3(x,y) = \min(1, d(x,y)) \ge 0\)
        \item \(d_3(x,x) = \min(1, 0) = 0\)
        \item \(d_3(x,y) = \min(1, d(x,y)) = \min(1, d(y,x)) = d_3(y,x)\)
        \item Знаємо, що \(d(x,z) \le d(x,y) + d(y,z)\). Тоді
        \[d_3(x,z) = \min(1, d(x,z)) = \begin{cases}
            1, & d(x,z) > 1 \\
            d(x,z), & d(x,z) \le 1
        \end{cases}\]
        \begin{align*}
            d(x,z) \le 1 \implies d_3(x,z) &= d(x,z) \le \min(1, d(x,y) + d(y,z)) \le \\
            &\le \min(1, d(x,y)) + \min(1, d(y,z)) = \\
            &= d_3(x,y) + d_3(y,z)
        \end{align*}
    \end{enumerate}
    
\end{proof}
\pagebreak

\section*{№ 23}
\begin{mdframed}
    Чи є сепарабельним простір 
    \[l^1 = \{\{x_n\}_{n\ge1} : \sum_{k=1}^\infty |x_k| < +\infty\}\]
    з метрикою 
    \[d(x,y) = \sum_{k=1}^\infty |x_k - y_k|\]?
\end{mdframed}

\begin{mdframed}[backgroundcolor=purple!20]
    Простір $(X, d)$ сепарабельний, якщо містить всюди щільну не більш ніж зліченну множину.
    Тобто
    \[\exists S \subset X : (|X| \le \aleph_0) \wedge (\forall x \in X, \varepsilon > 0: S \cap B(x,\varepsilon) \ne \varnothing) \]
\end{mdframed}

Знаємо, що $\mathbb Q$ - всюди щільна в $\mathbb R$. 
Перевіримо потужність множини всіх раціональних послідовностей.

\[\text{let } Q - \text{ множина всіх раціональних послідовностей}\]
На жаль, за методом діагоналізації Кантора для 0/1-послідовностей можна сказати, що $|Q| > \aleph_0$.
Треба йти іншим шляхом.

Я можу як завгодно близько наблизити нескінченну послідовність $\{x_n\}_{n\ge1} = x \in l^1$ 
скінченною послідовністю \[(x^{(n)}_1, x^{(n)}_2, ..., x^{(n)}_n, 0, 0, ...) = x^{(n)} \in l^1\].
Це можлилво, бо \[\sum_{k=1}^\infty |x_k| < +\infty \implies \lim_{n\to \infty} \sum_{k=n}^\infty |x_k| = 0  \].
Тобто залишкова сума збіжного ряду має прямувати до 0. Тобто 
\[\forall \varepsilon>0: \exists N: \forall n>N: \sum_{k=n}^\infty |x_k| < \varepsilon\]

Таким чином
\[\forall \varepsilon>0, x \in l^1: \exists x^{(n)}: d(x,x^{(n)}) = \sum_{k=n}^\infty |x_k| < \varepsilon\]

Оскільки $\mathbb Q$ - всюди щільна в $\mathbb R$, то можемо обмежитись скінченними послідовностями з раціональними елементами.

З іншого боку, множина всіх скінченних послідовностей довжини $n$ над $\mathbb Q$ - зліченна.
\[|X^{(n)}|=|=|\{(x^{(n)}_1, x^{(n)}_2, ..., x^{(n)}_n, 0, 0, ...), x^{(n)}_i \in \mathbb Q\}| = |\mathbb Q^n| = \aleph_0\]
Більше того, 
\[|X| = \left|\bigcup_{i=1}^\infty X^{i}\right| = \aleph_0\]
як зліченне об'єднання зліченних множин.

Маємо,
\[\left(\forall \varepsilon>0, x \in l^1: \exists x^{(n)} \in X: d(x,x^{(n)}) < \varepsilon\right) \wedge \left(|X|=\aleph_0\right)\]

Отже, простір $(l^1, d)$ - сепарабельний, бо $X \subset l^1$ - всюди щільна скінченна множина.

\section*{№ 24}
\begin{mdframed}
    Чи є сепарабельним простір 
    \[l^2 = \{\{x_n\}_{n\ge1} : \sum_{k=1}^\infty |x_k|^2 < +\infty\}\]
    з метрикою 
    \[d_2(x,y) = \sqrt{\sum_{k=1}^\infty |x_k - y_k|^2}\]?
\end{mdframed}

$\sum_{k=1}^\infty |x_k|^2$ - збіжний ряд.
Тоді суму $S = \sum_{k=n}^\infty |x_k|^2$ можна зробити як завгодно малою (збільшуючи $n$).

Тоді довільну послідовність $x \in l^2$ можна наблизити скінченною послідовністю, ще й раціональних чисел.
Бо \[d_2(x,x^{(n)}) = \sqrt{\sum_{k=n}^\infty |x_k|^2} = \sqrt{S}\]

А далі аналогічно № 23. 
Множина всіх скінченних раціональних послідовностей - зліченна та всюди щільна в $(l^2, d_2)$

Отже, простір $(l^2, d_2)$ - сепарабельний
\pagebreak

\section*{№ 39}

\begin{mdframed}
    \[l^p = \{(x_1,x_2,...) : \left(\sum_{k=1}^\infty |x_k|^p\right)^{\frac{1}{p}} < +\infty\}, \quad d(x,y) = \left(\sum_{k=1}^\infty |x_k-y_k|^p\right)^{\frac{1}{p}}\]
    \[l^\infty = \{(x_1,x_2,...) : \max_{k\ge1} |x_k| < +\infty\}, \quad d(x,y) = \max_{k\ge1} |x_k-y_k|\]
    \[s = \{(x_1, x_2,...) - \text{довільна дійсна послідовність}\}, \quad d(x,y) = \sum_{k=1} \frac{1}{2^k} \frac{|x_k - y_k|}{1+|x_k - y_k|}\]
\end{mdframed}

\subsection*{1.}
\begin{mdframed}
    \[\{x^{(n)}\}, \; x^{(n)} = (1,2,...,n,0,0,...)\]
    \[\lim_{n\to\infty} x^{(n)} = (1,2,3,...,n-1,n,n+1,...)\]
\end{mdframed}
\begin{enumerate}
    \item $l^p$: \[\lim_{n\to\infty} \sum_{k=1}^\infty |x^{(n)}_k|^p = \sum_{k=1}^\infty k^p = +\infty\]
    тобто $\lim_{n\to\infty} x^{(n)} \notin l^p$. \begin{mdframed}[backgroundcolor=red!20]
    Робіжна
    \end{mdframed}

    \item $l^\infty$: \[\lim_{n\to\infty} \max_{k\ge1} x^{(n)}_k = \max_{k\ge1} k  = +\infty\]
    тобто $\lim_{n\to\infty} x^{(n)} \notin l^\infty$. \begin{mdframed}[backgroundcolor=red!20]
    Робіжна
    \end{mdframed}

    \item $s$: очевидно, $x = \lim_{n\to\infty} x^{(n)} \in s$
    Перевіримо збіжність. Нехай $d(x^{(n)}, x) = d_n$. Для збіжності достатньо щоб $\lim_{n\to\infty} d_n = 0$.
    
    \[d_n = d(x^{(n)}, x) = \sum_{k > n} \frac{k}{2^k} \cdot \frac{1}{1+1} = \frac{1}{2} \sum_{k > n} \frac{k}{2^k}\]
    \[d_0 = \frac{1}{2} \sum_{k \ge 1} \frac{k}{2^k} \]
    Досліджу на збіжність ряд $d_0$. Ознака д'Аламбера:
    \[\lim_{k\to\infty} \frac{\frac{k+1}{2^{k+1}}}{\frac{k}{2^k}} = \frac{1}{2}\lim_{k\to\infty} \frac{k+1}{k} = \frac{1}{2} \cdot 1 < 1\]
    Успішно. Тобто ряд збіжний. Тоді $\lim_{n\to\infty} d_n =0$ як залишок збіжного ряду.

    Отже, $x^{(n)} \to x$ у $(s,d)$ - \begin{mdframed}[style=ans]
    Збіжна
    \end{mdframed}
\end{enumerate}

\subsection*{2.}
\begin{mdframed}
    \[x^{(n)} = (\underset{n}{\underbrace{1,1,...,1}},0,...)\]
    \[x = (1,1,1,...,1,...)\]
\end{mdframed}
\begin{enumerate}
    \item $l^p$: \[\lim_{n\to\infty} \sum_{k=1}^\infty |x^{(n)}_k|^p = \sum_{k=1}^\infty 1 = +\infty\]
    тобто $\lim_{n\to\infty} x^{(n)} \notin l^p$. \begin{mdframed}[backgroundcolor=red!20]
    Робіжна
    \end{mdframed}

    \item $l^\infty$: 
    \[\lim_{n\to\infty} \max_{k\ge1} x^{(n)}_k = \max_{k\ge1} 1  = 1 < \infty\; \implies \; x\in l^\infty \]
    Нехай $d(x^{(n)}, x) = d_n$.
    \[\forall n: d_n = \max_{k\ge1} |x^{(n)}_k-1| = 1 = const\]
    $d_n \to 1 \ne 0$, отже $\lim_{n\to\infty}x^{(n)} \ne x$.

    Треба ще довести, що інших кандидатів на границю немає, але в мене не вийшло.

    \begin{mdframed}[backgroundcolor=red!20]
    Робіжна
    \end{mdframed}


    \item $s$: очевидно, $x = \lim_{n\to\infty} x^{(n)} \in s$
    \[d_n = d(x^{(n)}, x) = \sum_{k>n} \frac{1}{2^k} \cdot \frac{1}{1+1} = \frac{1}{2}\sum_{k>n} \frac{1}{2^k}\}\]
    $d_0 = \frac{1}{2}\sum_{k\ge 1} \frac{1}{2^k} = \frac{1}{2} \cdot \frac{1/2}{1-1/2} = \frac{1}{2}$ - збіжний ряд.
    Тоді $d_n \longrightarrow 0$ як залишок збіжного ряду.
    
    Отже, $x^{(n)} \longrightarrow x$ у $(s,d)$ - \begin{mdframed}[style=ans]
    Збіжна
    \end{mdframed}. 
\end{enumerate}
\pagebreak

\subsection*{3.}
\begin{mdframed}
    \[x^{(n)} = (\underset{n}{\underbrace{\frac{1}{n},\frac{1}{n},...,\frac{1}{n}}},0,...)\]
    \[x = (0,0,...,0,...)\]
\end{mdframed}
\begin{enumerate}
    \item $l^p$: очевидно $x\in l^p$.
    Нехай $d(x^{(n)}, x) = d_n$.
    \[d_n = \sum_{1 \le k \le n} \frac{1}{n} = n \cdot \frac{1}{n} = 1 = const\]
    \[d_n \longrightarrow 1 \ne 0\]
    Тобто, збіжності до $x$ немає. (очевидно, інших границь крім $x$ бути не може).

    \begin{mdframed}[backgroundcolor=red!20]
    Робіжна
    \end{mdframed}
    

    \item $l^\infty$: очевидно $x\in l^\infty$.
    \[d_n = d(x^{(n)}, x) = \max_{1 \le k \le n} \frac{1}{n} = \frac{1}{n} \underset{n\to\infty}{\longrightarrow} 0\]
    Тобто, $x^{(n)} \longrightarrow x$ в $(l^\infty, d)$! 
    
    \begin{mdframed}[style=ans]
    Збіжна
    \end{mdframed}

    \item $s$: очевидно, $x \in s$. 
    Нехай \[d'(x,y) = \sum_{k=1} \frac{1}{2^k} |x_k - y_k| \ge \sum_{k=1} \frac{1}{2^k} \frac{|x_k - y_k|}{1+|x_k - y_k|}\]
    \[d'_n = d'(x^{(n)}, x) = \sum_{1\le k \le n} \frac{1}{2^k} \cdot \frac{1}{n} = \frac{1}{n} \cdot \frac{1}{2} \cdot \frac{1-\frac{1}{2^n}}{1-\frac{1}{2}} = \frac{1}{n} \cdot \left(1-\frac{1}{2^n}\right)\]
    Тут була застосована формула суми скінченної геометричної прогресії.
    \[\lim_{n\to\infty} d'_n = \lim_{n\to\infty} \frac{1}{n} \cdot \left(1-\frac{1}{2^n}\right) = 0 \cdot 1 = 0\]
    \[0 \le \lim_{n\to\infty} d_n \le \lim_{n\to\infty} d'_n = 0\]
    Тобто, $x^{(n)} \longrightarrow x$ в $(s, d)$! 
    
    \begin{mdframed}[style=ans]
    Збіжна
    \end{mdframed}
\end{enumerate}
\pagebreak

\subsection*{4.}
\begin{mdframed}
    \[x^{(n)} = (\underset{n}{\underbrace{\frac{1}{n^a},\frac{1}{n^a},...,\frac{1}{n^a}}},0,...)\]
    \[x = (0,0,...,0,...)\]

    Одразу припускаю, що $a \ne 1$.
\end{mdframed}
\begin{enumerate}
    \item $l^p$: очевидно $x\in l^p$.
    \begin{multline*}
        d_n = d(x^{(n)}, x) = \sum_{1 \le k \le n} \frac{1}{n^a} = n \cdot \frac{1}{n^a} = n^{1-a} 
        \underset{n\to\infty}{\longrightarrow} \begin{cases}
            0, & a > 1\\
            \infty, & a < 1
        \end{cases} 
    \end{multline*}
    Тобто, збіжність є за $a > 1$.

    \begin{mdframed}[backgroundcolor=yellow!20]
    Збіжна за $a > 1$
    \end{mdframed}
    

    \item $l^\infty$: очевидно $x\in l^\infty$.
    \begin{multline*}
        d_n = d(x^{(n)}, x) = \max_{1 \le k \le n} \frac{1}{n^a} = \frac{1}{n^a} 
        \underset{n\to\infty}{\longrightarrow} \begin{cases}
            0, & a > 0\\
            \infty, & a < 0\\
            1, & a = 0
        \end{cases}
    \end{multline*} 
    Тобто, збіжність є за $a > 0$.

    \begin{mdframed}[backgroundcolor=yellow!20]
    Збіжна за $a > 0$
    \end{mdframed}

    \item $s$: очевидно, $x \in s$.
    \begin{multline*}
        d_n = d(x^{(n)}, x) = \sum_{1\le k \le n} \frac{1}{2^k} \cdot \frac{\frac{1}{n^a}}{1+\frac{1}{n^a}} = \frac{\frac{1}{n^a}}{1+\frac{1}{n^a}} \cdot \frac{1}{2} \cdot \frac{1-\frac{1}{2^n}}{1-\frac{1}{2}} = \frac{\frac{1}{n^a}}{1+\frac{1}{n^a}} \cdot \left(1-\frac{1}{2^n}\right)
    \end{multline*}
    Тут була застосована формула суми скінченної геометричної прогресії.
    
    \[\lim_{n\to\infty} d_n = \lim_{n\to\infty} \frac{\frac{1}{n^a}}{1+\frac{1}{n^a}} \cdot \left(1-\frac{1}{2^n}\right) \]
    \[\lim_{n\to\infty} \frac{1}{n^a} = \begin{cases}
        0, & a>0\\
        1, & a=0\\
        \infty, & a<0
    \end{cases}\]
    Тоді
    \[\lim_{n\to\infty} d_n = \begin{cases}
        0, & a>0\\
        \frac{1}{1+1} \cdot 1 = \frac{1}{2}, & a=0\\
        \infty, & a<0
    \end{cases}\]
    Тобто, збіжність є за $a > 0$.

    \begin{mdframed}[backgroundcolor=yellow!20]
    Збіжна за $a > 0$
    \end{mdframed}
\end{enumerate}



\end{document}

