% !TEX TS-program = xelatex
% !TEX encoding = UTF-8

\documentclass[11pt, a4paper]{article} % use larger type; default would be 10pt

\usepackage{fontspec} % Font selection for XeLaTeX; see fontspec.pdf for documentation
\defaultfontfeatures{Mapping=tex-text} % to support TeX conventions like ``---''
\usepackage{xunicode} % Unicode support for LaTeX character names (accents, European chars, etc)
\usepackage{xltxtra} % Extra customizations for XeLaTeX
\usepackage{tikz}
\usetikzlibrary{arrows,calc,patterns}

\setmainfont[Ligatures=TeX]{Garamond} % set the main body font (\textrm), assumes Charis SIL is installed
%\setsansfont{Deja Vu Sans}
\setmonofont[Ligatures=TeX]{Fira Code}

% other LaTeX packages.....
\usepackage{fullpage}
\usepackage[top=2cm, bottom=4.5cm, left=2.5cm, right=2.5cm]{geometry}
\usepackage{amsmath,amsthm,amsfonts,amssymb,amscd,systeme}
\usepackage{cancel}
\geometry{a4paper} 
%\usepackage[parfill]{parskip} % Activate to begin paragraphs with an empty line rather than an indent
\usepackage{fancyhdr}
\usepackage{listings}
\usepackage{graphicx}
\usepackage{hyperref}
\usepackage{multicol}

\renewcommand\lstlistingname{Algorithm}
\renewcommand\lstlistlistingname{Algorithms}
\def\lstlistingautorefname{Alg.}
\lstdefinestyle{mystyle}{
    % backgroundcolor=\color{backcolour},   
    % commentstyle=\color{codegreen},
    % keywordstyle=\color{magenta},
    % numberstyle=\tiny\color{codegray},
    % stringstyle=\color{codepurple},
    basicstyle=\ttfamily\footnotesize,
    breakatwhitespace=false,         
    breaklines=true,                 
    captionpos=b,                    
    keepspaces=true,                 
    numbers=left,                    
    numbersep=5pt,                  
    showspaces=false,                
    showstringspaces=false,
    showtabs=false,                  
    tabsize=2
}
\lstset{style=mystyle}

\newcommand\course{5 - Теорія ймовірності}
\newcommand\hwnumber{ДЗ №6}                   % <-- homework number
\newcommand\idgroup{ФІ-91}                
\newcommand\idname{Михайло Корешков}  

\usepackage[framemethod=TikZ]{mdframed}
\mdfsetup{%
	backgroundcolor = black!5,
}
\mdfdefinestyle{ans}{%
    backgroundcolor = green!5,
    linecolor = green!50,
    linewidth = 1pt,
}

\pagestyle{fancyplain}
\headheight 35pt
\lhead{\idgroup \\ \idname}
\chead{\textbf{\Large \hwnumber}}
\rhead{\course \\ \today}
\lfoot{}
\cfoot{}
\rfoot{\small\thepage}
\headsep 1.5em

\linespread{1.2}

\begin{document}

\section*{№ 6.14}
$$
H_{1w} = \{\text{з першої урни дістали білу кулю}\}; \quad P(H_{1w}) = \frac{1}{10}$$
$$
H_{2w} = \{\text{з другої урни дістали білу кулю}\}; \quad P(H_{2w}) = \frac{5}{6}
$$

Зауважу, що події $H_{1w}$ та $H_{2w}$ - незалежні.

Група подій $\{ H_{1w}^{a_1} H_{2w}^{a_2} \; | \; a_i \in {1, -1} \}$ є повною:
\begin{itemize}
    \item Події не перетинаються
    \item В об'єднанні дають всі можливі події
\end{itemize}

Застосовна формула повної ймовірності

\begin{align*}
    & A = \{\text{з 3 урни дістали білу кулю}\} \\
    & P(A|H_{1w}H_{2w}) = \frac{4}{4+10} = \frac{4}{14} \\
    & P(A|H_{1w}H_{2b}) = \frac{5}{5+9} = \frac{5}{14} \\
    & P(A|H_{1b}H_{2w}) = \frac{5}{5+9} = \frac{5}{14} \\
    & P(A|H_{1b}H_{2b}) = \frac{6}{6+8} = \frac{6}{14} \\
\end{align*}

\begin{align*}
    P(A) &=  
    P(A|H_{1w}H_{2w}) P(H_{1w}H_{2w}) +
    P(A|H_{1w}H_{2b}) P(H_{1w}H_{2b}) + \\
    &+ P(A|H_{1b}H_{2w}) P(H_{1b}H_{2w}) +
    P(A|H_{1b}H_{2b}) P(H_{1b}H_{2b}) = \\
    &= \frac{4}{14} \frac{1}{10} \frac{5}{6} + \frac{5}{14} \frac{1}{10} \frac{1}{6} 
    + \frac{5}{14} \frac{9}{10} \frac{5}{6} + \frac{6}{14} \frac{9}{10} \frac{5}{6} = \\
    &= \frac{20 + 5 + 225 + 270}{840} = \frac{13}{21}
\end{align*}

\begin{mdframed}[style=ans]
    $$P(A) = \frac{13}{21}$$
\end{mdframed}

\pagebreak

\section*{№ 6.15}

$$C = \{\text{хтось влучив в ціль}\}$$

\begin{align*}
    & H_A = \{\text{Спостерігач побачив вистріл A}\} \\
    & H_B = \{\text{Спостерігач побачив вистріл B}\} \\
    & P(A) = P(C|H_A) = \frac{5}{10} \\
    & P(B) = P(C|H_B) = \frac{8}{10} \\
    & P(H_A) = P(H_B) = \frac{1}{2}
\end{align*}

$$P(A|C) = ?$$

\begin{align*}
    & P(C) = P(C|H_A) P(H_A) + P(C|H_B) P(H_B) = \frac{13}{20}; \\
    & P(\overline{C}) = \frac{7}{20} = P(\{\text{хтось не влучив в ціль}\}); \\
    & P(H_A|C) = \frac{P(C|H_A)P(H_A)}{P(C)} = \frac{\frac{5}{10}\cdot \frac{1}{2}}{\frac{13}{20}} = \frac{5}{13}
\end{align*}

\begin{mdframed}[style=ans]
    $$P(A|C) = \frac{5}{13}$$
\end{mdframed}

\section*{№ 6.16}
$$A = \{\text{Перший студент витягнув щасливий білет}\}$$
$$B = \{\text{Другий студент витягнув щасливий білет}\}$$

$$P(A) = \frac{n}{N}$$

\begin{align*}
    P(B) &= P(B|A)P(A) + P(B|\overline A) P(\overline A) = \\
    &= \frac{n-1}{N-1} \cdot \frac{n}{N} + \frac{n}{N-1} \cdot \frac{N-n}{N} = \\
    &= \frac{n^2-n+Nn-n^2}{(N-1)N} = \frac{n(N-1)}{N(N-1)} = \\
    &= \frac{n}{N} = P(A)
\end{align*}

\begin{mdframed}[style=ans]
    Тобто, не має різниці, брати білет першим чи другим
\end{mdframed}

\begin{align*}
    P(A|B) &= \frac{P(B|A)P(A)}{P(B)} = \frac{\frac{n-1}{N-1} \cdot \frac{n}{N}}{\frac{n}{N}} = \\
    &= \frac{n-1}{N-1}
\end{align*}

\begin{mdframed}[style=ans]
    $$P(A|B) = \frac{n-1}{N-1}$$
\end{mdframed}

\section*{№ 6.17}

$$S = \{\text{signal + noise}\};\quad P(S) = 0.4$$
$$N = \{\text{only noise}\};\quad P(N) = 0.6$$
$$D = \{\text{signal detected}\};$$
$$P(D|S) = 0.7;\quad P(D|N) = 0.5$$
$$P(S|D) = ?$$

\begin{align*}
    P(S|D) &= \frac{P(D|S)P(S)}{P(D|S)P(S)+P(D|N)P(N)} = (0.7 \cdot 0.4) / (0.7 \cdot 0.4 + 0.5 \cdot 0.6) = \\
    &= 28 / (28+30) = \frac{14}{29}
\end{align*}

\begin{mdframed}[style=ans]
    Якщо пристрій зафіксував сигнал, то сигнал дійсно надійшов з ймовірністю $\frac{14}{29}$
\end{mdframed}

\section*{№ 6.18}
$$\Omega = \{(a,b)\;|\; a,b \in \{F,F,F,R,R,R,R\}\};\quad \text{F - fake, R - real}$$
$$|\Omega| = 7*6 = 42$$
$$F_0 = \{\text{0 fake dice}\};\quad P(F_0) = A_4^2 / |\Omega| = \frac{12}{42} = \frac{2}{7}$$
$$F_1 = \{\text{1 fake dice}\};\quad P(F_1) = (3*4*2) / |\Omega| = \frac{24}{42} = \frac{4}{7}$$
$$F_2 = \{\text{2 fake dice}\};\quad P(F_2) = A_3^2 / |\Omega| = \frac{6}{42} = \frac{1}{7}$$
$$D_6 = \{t(a)+t(b)=6\}$$
$$P(F_{>0} | D_6) = ?$$

$F_0, F_1, F_2$ - повна група гіпотез

$$P(F_0 | D_6) = \frac{P(D_6 | F_0) P(F_0)}{(P(D_6|F_0)P(F_0) + P(D_6|F_1)P(F_1) + P(D_6|F_2)P(F_2)}$$

\begin{mdframed}
    $$P(D_6 | F_0) = |\{(3,3), (4,2), (2,4), (5,1), (1,5)\}|/36 = \frac{5}{36}$$
    $$P(D_6 | F_1) = \frac{|\{(3,3)\}|}{|\{(3,k)\}|} = \frac{1}{6}$$
    $$P(D_6 | F_2) = 1$$
\end{mdframed}

\begin{align*}
    P(F_0 | D_6) &= \frac{5}{36} \frac{2}{7} / (\frac{5}{36}\frac{2}{7} + \frac{1}{6}\frac{4}{7} + 1 \cdot \frac{1}{7}) 
    = \frac{5\cdot 2}{5\cdot 2+ 6\cdot 4+36\cdot 1} = \frac{10}{10+24+36} =\\
    &= \frac{10}{70} = \frac{1}{7}
\end{align*}

$$P(F_{>0} | D_6) = 1 - P(F_0 | D_6) = 1 - \frac{1}{7}$$

\begin{mdframed}[style=ans]
    $$P(F_{>0} | D_6) = \frac{6}{7}$$
\end{mdframed}

\section*{№ 6.19}
$S$ - all people, $F$ - females, $M$ - males.
$$S = M \sqcup F;\quad |S| = 75$$
$$\frac{|M|}{|F|} = \frac{7}{8}$$
$$\implies |M| = 35;\quad |F| = 40$$

$$L = \{\text{left-handed people}\}$$
$$|L\cap F| = 2$$
$$|L\cap M| = 3$$

$$\text{That is, } P(L|F) = \frac{1}{20}; \quad P(L|M) = \frac{3}{35}$$

$$H_{ff} = \{\text{вибрано дві жінки}\}$$
$$H_{fm} = \{\text{вибрано жінку і чоловіка}\}$$
$$H_{mm} = \{\text{вибрано два чоловіки}\}$$
$H_{ff}, H_{mm}, H_{fm}$ - повна група гіпотез

$$A = \{\text{both chosen people are left-handed}\}$$
$$P(H_{ff}|A) = ?$$

$$P(H_{ff}|A) = \frac{P(A|H_{ff})\cdot P(H_{ff})}{P(A|H_{ff})\cdot P(H_{ff}) + P(A|H_{fm})\cdot P(H_{fm}) + P(A|H_{mm})\cdot P(H_{mm})}$$
\begin{align*}
    & P(H_{ff}) = \frac{C_{40}^2}{C_{75}^2} 
    & P(A|H_{ff}) = \left(\frac{1}{20}\right)^2 \\
    & P(H_{mf}) = \frac{40\cdot 35 \cdot 2}{C_{75}^2} 
    & P(A|H_{fm}) = \frac{1}{20}\cdot\frac{3}{35} \\
    & P(H_{mm}) = \frac{C_{35}^2}{C_{75}^2} 
    & P(A|H_{ff}) = \left(\frac{3}{35}\right)^2 \\
\end{align*}

\begin{mdframed}[style=ans]
    
\begin{align*}
    P(H_{ff}|A) &= \frac{40 \cdot 39 \cdot \frac{1}{20^2}}{40\cdot 39\cdot \frac{1}{20^2} + 40\cdot 35 \cdot \frac{1}{20\cdot 35} + 35\cdot 34\cdot \frac{1}{35^2}} = \\
    &=\frac{21}{37}
\end{align*}
\end{mdframed}

\section*{№ 6.20}
$$A = \{\text{Студент знає відповідь на питання}\}$$
$$P(A) = \frac{2}{3}$$
$$H_k = \{\text{на кубику випало k}\}$$
$$P(H_k) = \frac{1}{6};\quad k=\overline{1,6}$$
$H_1, \cdots, H_6$ - повна група гіпотез
$$F = \{\text{Студент не знає відповіді на жодне питання}\}$$
$$P(\overline{F}) = 1-P(F) = ?$$

\begin{align*}
    P(F) &= \sum_{k=1}^6 P(F|H_k)P(H_k) = \\
    &= \frac{1}{6}\cdot \sum_k \left(\frac{1}{3}\right)^k = \frac{1}{6}\cdot \frac{1-\frac{1^6}{3^6}}{1-\frac{1}{3}} = \\
    &= \frac{1}{6} \cdot \frac{364}{729}
\end{align*}

\begin{mdframed}[style=ans]
    $$P(\overline{F}) = 1-\frac{1}{6} \cdot \frac{364}{729} \approx 0.917$$
\end{mdframed}

\section*{№ 6.21}
\begin{align*}
    & R = \{\text{It is raining today}\}; & P(R) = \frac{1}{2} && \\
    & NR = \{\text{No rain today}\}; & P(NR) = \frac{1}{2} && \\
    & RF = \{\text{Rain forecasted}\} & P(R|RF) = \frac{2}{3} && \\
    & & P(NR|RF) = \frac{1}{3} &&\\
    & NRF = \{\text{No rain forecasted}\}& P(NR|NRF) = \frac{2}{3} &&\\
    & & P(R|NRF) = \frac{1}{3} &&\\
    & U = \{\text{Mr. took an umbrella}\} & P(U|RF) = 1 &&\\
    & & P(U|NRF) = \frac{1}{3} && \\
    & NU = \{\text{No umbrella}\} & &&
\end{align*}

$$P(NU|R) = ?;\quad P(NR|U) = ?$$

\begin{mdframed}
    $$P(R) = P(R|RF)P(RF) + P(R|NRF)P(NRF)$$
    $$\frac{1}{2} = \frac{2}{3} P(RF) + \frac{1}{3}(1-P(RF))$$
    $$\frac{1}{2} - \frac{1}{3} = \frac{1}{3} P(RF)$$

    $$P(RF) = \frac{3}{6} = \frac{1}{2};\quad P(NRF) = 1-\frac{1}{2} = \frac{1}{2}$$
\end{mdframed}

$$P(U) = P(U|RF)P(RF) + P(U|NRF)P(NRF) = \frac{1}{2} + \frac{1}{6} = \frac{2}{3}$$

Нехай $U$ та $R$ - незалежні (бо інакше нічого не працює).
$P(U\cap R) = P(U)P(R) + X$
Тоді 
\begin{align*}
    P(NU|R) &= 1-P(U|R) = 1-\frac{P(U\cap R)}{P(R)} =\\
    &= 1-\frac{P(U)P(R)}{P(R)} = 1-P(U) = \\ &= \frac{1}{3}
\end{align*}

\begin{align*}
    P(NR|U) &= 1-P(R|U) = 1-P(R) =\\ &= \frac{1}{2}
\end{align*}

\end{document}

