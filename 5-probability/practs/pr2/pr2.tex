% !TEX TS-program = xelatex
% !TEX encoding = UTF-8

% This is a simple template for a XeLaTeX document using the "article" class,
% with the fontspec package to easily select fonts.

\documentclass[11pt]{article} % use larger type; default would be 10pt

\usepackage{fontspec} % Font selection for XeLaTeX; see fontspec.pdf for documentation
\defaultfontfeatures{Mapping=tex-text} % to support TeX conventions like ``---''
\usepackage{xunicode} % Unicode support for LaTeX character names (accents, European chars, etc)
\usepackage{xltxtra} % Extra customizations for XeLaTeX

\setmainfont{Garamond} % set the main body font (\textrm), assumes Charis SIL is installed
%\setsansfont{Deja Vu Sans}
%\setmonofont{Deja Vu Mono}

% other LaTeX packages.....
\usepackage{geometry} % See geometry.pdf to learn the layout options. There are lots.
\geometry{a4paper} % or letterpaper (US) or a5paper or....
%\usepackage[parfill]{parskip} % Activate to begin paragraphs with an empty line rather than an indent

\usepackage{amsmath}
\usepackage{amssymb}
\usepackage{graphicx} % support the \includegraphics command and options

\title{Теорія ймовірності \\ Практика 2}
\author{ФІ-91 Михайло Корешков}
%\date{} % Activate to display a given date or no date (if empty),
         % otherwise the current date is printed 

\begin{document}
\maketitle
\pagebreak

\section{hw help}
\subsection{1.19b}
$
\begin{pmatrix}
    a1, b1, c1, d1, a2, d2 \\
\end{pmatrix} \\
\begin{pmatrix}
    a2, b1, c1, d1, a1, d2 \\
\end{pmatrix}
$ 

Два випадки: \\
1. Ax2, Bx2, Cx1, Dx1 \\
2. Ax3, Bx1, Cx1, Dx1 \\
\begin{align*} 
    N = C_4^1 \cdot C_13^3 \cdot C_13^1 \cdot C_13^1 \cdot C_13^1 + \\ 
    + C_4^2 \cdot C_13^2 \cdot C_13^2 \cdot C_13^2 \cdot C_13^1
\end{align*}
Краще всього взагалі уникати залежних виборів в перемноженні. Тут ми розбили на всі можливі неперетинні випадки.


\section*{# 3.3}

\section{# 3.4}
a) $A = \{$ Одна монета випала гербом $\}$ \\
$\Omega = \{HHH, HHT, HTH, THH, HTT, THT, TTH, TTT\}$ \\
$A = \Omega \setminus \{HHH\} \quad \implies \quad P(A) = \frac{7}{8}$
\\
b) $B = \{$ 2 герба $\}$ \\
$B = \{HHT, THH, HTH\} \implies P(B) = \frac{3}{8}$ \\
\\
c) $C = \{$ 2+ гебрів $\}$ \\
$C = \{HHH, HHT, THH, HTH\} \implies P(C) = \frac{4}{8}$
\\

\section{# 3.5}
$\Omega = \{(x_1, x_2, ..., x_{12}\:|\: x_i = \overline{1,6})\}$ \\
$|\Omega| = 6^{12};\quad P(\omega) = \frac{1}{6^{12}}$ \\
$A = \{$Кожне очко випало двічі$\}$ \\
$A = \{\omega \in \Omega \:|\: \#_i(\omega)=2 \text{for} i=\overline{1,6} \}$
1. Розставляємо 2 одиниці. Потім 2 двійки. ... Потім 2 шестірки: \\
$$|A| = \frac{12!}{(2!)^6} \implies P(A) = \frac{12!}{2^6 \cdot 6^{12}}$$

\section{# 3.7}
$\Omega = \{1, 2, ..., 100!\}; \quad |\Omega| = 100!$ \\
a) $A = \{2, 4, 6, ..., 100!\}; \quad P(A) = \frac{1}{2}$ \\
b) $B = \{k | 2\mid k and 3\mid k\} = \{6, 12, ..., 100!\} \implies P(A) = \frac{1}{6}$\\
c) $C = \{k | 2\mid k or 3\mid k or 5\mid k \}$
$$ P(C) = P(F \cup G \cup H) $$

\end{document}