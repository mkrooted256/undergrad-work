% !TEX TS-program = xelatex
% !TEX encoding = UTF-8

\documentclass[11pt, a4paper]{article} % use larger type; default would be 10pt

\usepackage{fontspec} % Font selection for XeLaTeX; see fontspec.pdf for documentation
\defaultfontfeatures{Mapping=tex-text} % to support TeX conventions like ``---''
\usepackage{xunicode} % Unicode support for LaTeX character names (accents, European chars, etc)
\usepackage{xltxtra} % Extra customizations for XeLaTeX
\usepackage{tikz}
\usetikzlibrary{arrows,calc,patterns}

\setmainfont[Ligatures=TeX]{Garamond} % set the main body font (\textrm), assumes Charis SIL is installed
%\setsansfont{Deja Vu Sans}
\setmonofont[Ligatures=TeX]{Fira Code}

% other LaTeX packages.....
\usepackage{fullpage}
\usepackage[top=2cm, bottom=4.5cm, left=2.5cm, right=2.5cm]{geometry}
\usepackage{amsmath,amsthm,amsfonts,amssymb,amscd,systeme}
\usepackage{cancel}
\geometry{a4paper} 
%\usepackage[parfill]{parskip} % Activate to begin paragraphs with an empty line rather than an indent
\usepackage{fancyhdr}
\usepackage{listings}
\usepackage{graphicx}
\usepackage{hyperref}
\usepackage{multicol}

\renewcommand\lstlistingname{Algorithm}
\renewcommand\lstlistlistingname{Algorithms}
\def\lstlistingautorefname{Alg.}
\lstdefinestyle{mystyle}{
    % backgroundcolor=\color{backcolour},   
    % commentstyle=\color{codegreen},
    % keywordstyle=\color{magenta},
    % numberstyle=\tiny\color{codegray},
    % stringstyle=\color{codepurple},
    basicstyle=\ttfamily\footnotesize,
    breakatwhitespace=false,         
    breaklines=true,                 
    captionpos=b,                    
    keepspaces=true,                 
    numbers=left,                    
    numbersep=5pt,                  
    showspaces=false,                
    showstringspaces=false,
    showtabs=false,                  
    tabsize=2
}
\lstset{style=mystyle}

\newcommand\course{5 - Теорія ймовірності}
\newcommand\hwnumber{ДЗ №7 - КР№1 - Варіант 1}                   % <-- homework number
\newcommand\idgroup{ФІ-91}                
\newcommand\idname{Михайло Корешков}  

\usepackage[framemethod=TikZ]{mdframed}
\mdfsetup{%
	backgroundcolor = black!5,
}
\mdfdefinestyle{ans}{%
    backgroundcolor = green!5,
    linecolor = green!50,
    linewidth = 1pt,
}

\pagestyle{fancyplain}
\headheight 35pt
\lhead{\idgroup \\ \idname}
\chead{\textbf{\Large \hwnumber}}
\rhead{\course \\ \today}
\lfoot{}
\cfoot{}
\rfoot{\small\thepage}
\headsep 1.5em

\linespread{1.2}

\begin{document}

\section*{№1}
$$\Omega = \{(w_1,w_2,w_3)\;|\; w_i \in \overline{1,6}\}$$
$$A = \{(w_1,w_2,w_3)\;|\; \sum_i w_i \in \{3,5,7,9,11,13,15,17\}\}$$

$$|\Omega| = 6^3;\;\quad |A| = ?$$

Зауваження: ($a$ - парне, $b$ - непарне)
\begin{itemize}
    \item $a + a + a = a$
    \item $a + a + b = b$
    \item $a + b + b = a$
    \item $b + b + b = b$
\end{itemize}

Нехай $B_i = \{w\in\Omega\;|\; w_i \text{ - непарне}\}$.
$$P(B_i) = \frac{1}{2}, \quad|B_i| = \frac{|\Omega|}{2}$$
Нехай $p = P(B_i)$

\begin{mdframed}[style=ans]
    $$P(A) = C_3^3 p^3 \cdot C_3^1 p(1-p)^2 = p^3 (1 + 3) = \frac{4}{8} = \frac{1}{2} $$
\end{mdframed}

\section*{№2}
\begin{mdframed}
    З колоди 36 карт навмання виймають чотири карти. Знайдіть умовну
ймовірність того, що серед них є принаймні три тузи, якщо відомо, що серед
них є два тузи.
\end{mdframed}

$$|\Omega| = C_36^4$$
$$A_i = \{i \text{ тузів}\}$$


$|A_0| = |$ обираємо наче тузів немає $| = C_{32}^4$ \\
$|A_1| = С_4^1\cdot C_{32}^3 = 4\cdot \frac{32!}{3!29!}$ - обираємо 1 з 4 тузів та 3 не-тузові карти\\ 
$|A_2| = c_4^2\cdot C_{32}^2 = 6\cdot \frac{32!}{2!30!}$ - обираємо 2 з 4 тузів та 2 не-тузові карти\\ 
$|A_3| = C_4^3\cdot C_{32}^1 = 4\cdot \frac{32!}{31!}$ - обираємо 3 з 4 тузів та 1 не-тузову карту\\ 
$|A_4| = C_4^4\cdot C_{32}^0 = 1$ - обираємо лише тузи\\

$$B = (A_3 \sqcup A_4) \cap (A_2 \sqcup A_3 \sqcup A_4) = A_3 \sqcup A_4$$
$$|B| = |A_3|+|A_4| = 1 + 4\cdot 32 = 129$$

$$|A_2 \sqcup A_3 \sqcup A_4| = 1 + 4\cdot 32 + 6 \cdot 32\cdot 16 = 3201$$

\begin{mdframed}[style=ans]
    $$P(B|A_2 \sqcup A_3 \sqcup A_4) = \frac{|B|}{|A_2 \sqcup A_3 \sqcup A_4|} = \frac{129}{3201} = 0.04$$
\end{mdframed}

\section*{№3}
\begin{mdframed}
    На відрізку [0,2] навмання обрано дві точки А та В. Знайдіть ймовірність
того, що довжина кола, побудованого на АВ як на діаметрі, є не меншою за
π/2 і не більшою за π
\end{mdframed}

$$\Omega = \{(a,b)\in [0,2]^2\}$$
$$A = \{(a,b)\in\Omega\;|\; \frac{\pi}{2} \le \pi \cdot |b-a| \le \pi \}$$
$$A = \{(a,b)\in\Omega\;|\; \frac{1}{2} \le |b-a| \le 1 \}$$

$$A:\; \begin{cases}
    \begin{cases}
        b \le a + 1\\
        b \ge a + \frac{1}{2}
    \end{cases},& b\ge a \\
    \begin{cases}
        b \ge a - 1\\
        b \le a - \frac{1}{2}
    \end{cases},& b<a
\end{cases}$$

$$
\begin{tikzpicture}
    \draw[->] (0,0) -- (2,0) node[right] {$a$};
    \draw[->] (0,0) -- (0,2) node[above] {$b$};
    \fill[green!25] (0,1) -- (1,2) -- (1.5,2) -- (0,0.5) -- cycle;
    \draw (0,1) -- (1,2);
    \draw (0,0.5) -- (1.5,2);
    \fill[green!25] (1,0) -- (2,1) -- (2,1.5) -- (0.5,0) -- cycle;
    \draw (0.5,0) node[below] {$\frac{1}{2}$} -- (2,1.5);
    \draw (1,0) node [below] {$1$} -- (2,1);
\end{tikzpicture}
$$

$$\mu(A) = 2 \cdot (\frac{1}{2}\cdot (2-\frac{1}{2})^2 - \frac{1}{2}\cdot (2-1)^2) = $$
$$= 1.5^2 - 1^2 = 1.25$$

\begin{mdframed}[style=ans]
    
    $$P(A) = \frac{\mu(A)}{\mu(\Omega)} = \frac{1.25}{4} = 0.3125$$
\end{mdframed}

\section*{№4}

\begin{mdframed}
    З 18 біатлоністів 5 влучають в ціль з ймовірністю 0.8, 7 – з ймовірністю 0.7, 4
- з ймовірністю 0.6 і 2 - з ймовірністю 0.5. Навмання обраний спортсмен зробив
два постріли, лише один з яких був влучним. До якої з груп найімовірніше
належав цей біатлоніст?
\end{mdframed}

$$A = \{\text{Влучив в ціль}\}$$

$$P(A|G_1) = 0.8$$
$$P(A|G_2) = 0.7$$
$$P(A|G_3) = 0.6$$
$$P(A|G_4) = 0.5$$
$$P(G_1) = \frac{5}{18}$$
$$P(G_2) = \frac{7}{18}$$
$$P(G_3) = \frac{4}{18}$$
$$P(G_4) = \frac{2}{18}$$

$$B = \{\text{влучив в ціль один раз з двох}\}$$

$$P(A) = \sum_i P(A|G_i)P(G_i) = 0.683(3) = \frac{205}{300}$$
$$P(B) = P(A)\cdot P(\overline{A}) = \frac{205 \cdot 95}{300^2} \approx 0.2164$$


$$P(G_i|B)P(B) = P(B|G_i)P(G_i)$$
$$P(G_i|B) = \frac{P(B|G_i)P(G_i)}{P(B)}$$

$$P(B|G_1) = 0.8\cdot 0.2 = 0.16$$
$$P(B|G_2) = 0.7\cdot 0.3 = 0.21$$
$$P(B|G_3) = 0.6\cdot 0.4 = 0.24$$
$$P(B|G_4) = 0.5\cdot 0.5 = 0.25$$
$$P(G_1|B) = \frac{0.16\cdot 5}{18 \cdot 0.2164} = 0.205$$
$$P(G_2|B) = \frac{0.21\cdot 7}{18 \cdot 0.2164} = 0.377$$
$$P(G_3|B) = \frac{0.24\cdot 4}{18 \cdot 0.2164} = 0.246$$
$$P(G_4|B) = \frac{0.25\cdot 2}{18 \cdot 0.2164} = 0.128$$

\begin{mdframed}[style=ans]
    Отже, найбільш ймовірно, що стріляв біатлоніст з \textbf{групи 2}
\end{mdframed}

\section*{№5}
\begin{mdframed}
    Два рівносильних шахіста А та В зіграли сім партій з рахунком 3 : 4. Знайдіть
ймовірність того, що А ніколи не відставав від В більше, ніж на одне очко.
\end{mdframed}

$$a=3;\quad b=4;$$
$$e = a-b = -1;\quad e_i = a_i-b_i;$$

Нехай $N$ - кількість всіх можливих турнірів.\\
Нехай $N'$ - кількість таких турнірів, в яких  А ніколи не відставав від В більше, ніж на одне очко.

$$N = L_{(0,0)}(7,-1) = C_7^3 = \frac{7\cdot 6 \cdot 5}{6} = 7\cdot 5 = 35$$
$$N' = L_{(0,0)}^{\ge -1}(7,-1) = L_{(0,0)}^{> -2}(7,-1) = L_{(0,0)}(7,-1) - L_{(0,0)}^{\times -2}(7,-1)$$

$$L_{(0,0)}(7,-1) = C_7^3 = 35$$
$$L_{(0,0)}^{\times -2}(7,-1) = L_{(0,-4)}(7,-1) = L_{(0,0)}(7,3) = C_7^2 = 21$$
$$N' = 35 - 21 = 14$$

\begin{mdframed}[style=ans]
    
    $$P = \frac{N'}{N} = \frac{14}{35} = 0.4$$
\end{mdframed}

\section*{№6}
\begin{mdframed}
    Серед 10 лотерейних білетів цінні виграші припадають на два білета. Яку
найменшу кількість білетів потрібно придбати, щоб ймовірність цінного
виграшу була не меншою за 0.5?
\end{mdframed}
$$W = \{\text{Виграш}\};\quad B_i = \{\text{Придбано $i$ білетів}\};\quad P(W|B_1) = \frac{2}{10}$$
$$P(W|B_i) = 1-\frac{C_8^i}{C_{10}^i} \text{ - віднімаємо від всіх випадків ті, де жодного виграшу}$$

$$P(W|B_2) = 1 - \frac{8\cdot 7}{10\cdot 9} \approx 1 - 0.622 \approx 0.378$$
$$P(W|B_3) = 1 - \frac{8\cdot 7 \cdot 6}{10\cdot 9 \cdot 8} = 1 - \frac{7 \cdot 6}{10 \cdot 9} \approx 1 - 0.467 \approx 0.533 > 0.5$$

$$P(W|B_i) = 1 - \frac{(10-i)(10-i-1)}{90} = 1 - \frac{i^2 - 19i + 90}{90} = \frac{i^2 - 19i}{90}$$

\begin{mdframed}[style=ans]
    Достатньо придбати 3 білета щоб ймовірність виграшу була $> 0.5$
\end{mdframed}


\end{document}

