% !TEX TS-program = xelatex
% !TEX encoding = UTF-8

\documentclass[11pt, a4paper]{article} % use larger type; default would be 10pt

\usepackage{fontspec} % Font selection for XeLaTeX; see fontspec.pdf for documentation
\defaultfontfeatures{Mapping=tex-text} % to support TeX conventions like ``---''
\usepackage{xunicode} % Unicode support for LaTeX character names (accents, European chars, etc)
\usepackage{xltxtra} % Extra customizations for XeLaTeX
\usepackage{tikz}
\usetikzlibrary{arrows,calc,patterns}

\setmainfont[Ligatures=TeX]{Garamond} % set the main body font (\textrm), assumes Charis SIL is installed
%\setsansfont{Deja Vu Sans}
\setmonofont[Ligatures=TeX]{Fira Code}

% other LaTeX packages.....
\usepackage{fullpage}
\usepackage[top=2cm, bottom=4.5cm, left=2.5cm, right=2.5cm]{geometry}
\usepackage{amsmath,amsthm,amsfonts,amssymb,amscd,systeme}
\usepackage{unicode-math}
\usepackage{cancel}
\geometry{a4paper} 
%\usepackage[parfill]{parskip} % Activate to begin paragraphs with an empty line rather than an indent
\usepackage{fancyhdr}
\usepackage{listings}
\usepackage{graphicx}
\usepackage{hyperref}
\usepackage{multicol}

\renewcommand\lstlistingname{Algorithm}
\renewcommand\lstlistlistingname{Algorithms}
\def\lstlistingautorefname{Alg.}
\lstdefinestyle{mystyle}{
    % backgroundcolor=\color{backcolour},   
    % commentstyle=\color{codegreen},
    % keywordstyle=\color{magenta},
    % numberstyle=\tiny\color{codegray},
    % stringstyle=\color{codepurple},
    basicstyle=\ttfamily\footnotesize,
    breakatwhitespace=false,         
    breaklines=true,                 
    captionpos=b,                    
    keepspaces=true,                 
    numbers=left,                    
    numbersep=5pt,                  
    showspaces=false,                
    showstringspaces=false,
    showtabs=false,                  
    tabsize=2
}
\lstset{style=mystyle}

\newcommand\course{5 - Теорія ймовірності}
\newcommand\hwnumber{ДЗ №11}                   % <-- homework number
\newcommand\idgroup{ФІ-91}                
\newcommand\idname{Михайло Корешков}  

\usepackage[framemethod=TikZ]{mdframed}
\mdfsetup{%
	backgroundcolor = black!5,
}
\mdfdefinestyle{ans}{%
    backgroundcolor = green!5,
    linecolor = green!50,
    linewidth = 1pt,
}

\pagestyle{fancyplain}
\headheight 35pt
\lhead{\idgroup \\ \idname}
\chead{\textbf{\Large \hwnumber}}
\rhead{\course \\ \today}
\lfoot{}
\cfoot{}
\rfoot{\small\thepage}
\headsep 1.5em

\linespread{1.2}

\begin{document}

% 9.25-9.28, 11.18

\section*{№ 9.25}
\begin{mdframed}
    $\xi$ - в.в із щільністю розподілу $p(x)$
    \begin{equation*}
        p(x) = \begin{cases}
            2e^{-2x},& x\ge 0\\
            0, x < 0
        \end{cases}
    \end{equation*}
\end{mdframed}



\subsection*{1. $P(\xi\in[1,2])$}
$$P(\xi\in[1,2]) = \int_1^2 p(x) dx = \int_1^2 2e^{-2x} = \left. -e^{-2x} \right|_1^2 = e^{-2}-e^{-4}$$


\subsection*{2. $P(\xi\in(1,2])$}
$$\xi$$ має щільність розподілу $\implies$ випадкова величина неперервна \\
$\implies F_\xi(x-o) = F_\xi(x) = F_\xi(x+o)$

$$P(\xi\in(1,2]) = \int_1^2 p(x) dx = P(\xi\in[1,2]) = e^{-2}-e^{-4}$$


\subsection*{3. $P(\xi\ge 2)$}
$$P(\xi\ge 2) = \int_2^\infty p(x) dx = \lim_{a\to +\infty} \left. -e^{-2x}\right|_2^a = e^{-4} - \cancelto{0}{e^{-\infty}} = e^{-4} $$

\subsection*{4. $P(\xi \le 3)$}
$$P(\xi\le 3) = \int_{-\infty}^3 p(x) dx = \int_{-\infty}^0 p(x) dx + \int_{0}^3 p(x) dx = $$
$$= 0 - e^{-2\cdot 3} + e^{-2\cdot 0} = 1-e^{-6} \approx 0.997522$$

Note: $P(\xi\le 3) = P(\xi \in [0,3])$.

Note: $$F_\xi (x) = \begin{cases}
    0, & x<0 \\
    1-e^{-2x}, & x \ge 0 
\end{cases}$$

\subsection*{5. $P(\xi^2-5\xi\ge -6)$}

$$\xi^2-5\xi\ge -6$$
$$\xi^2-5\xi + 6 \ge 0$$
$$(\xi - 2)(\xi - 3) \ge 0$$
$$\left[ \begin{matrix}
    \xi \le 2\\
    \xi \ge 3
\end{matrix} \right.$$

$$P(\xi^2-5\xi\ge -6) = P(\xi \in (-\infty;2]\cup [3,\infty)) 
= \int_{-\infty}^2 p(x)dx + \int_3^{+\infty} p(x) dx 
= 1-\int_2^3 p(x) dx = $$
$$= 1 + e^{-6} - e^{-4} \approx 0.9842$$

\subsection*{6. $\eta = \ln (2\xi)$}

$$P(\eta\le x) = P(\ln (2\xi) \le x)$$
$$F_\eta(x) = P(\xi \le \frac{1}{2} e^x) = F_\xi(\frac{1}{2} e^x)$$
$$p_\eta(x) = \frac{1}{2} e^x p_\xi(\frac{1}{2} e^x) = e^x \cdot e^{-2\cdot\frac{1}{2} e^x} = e^{x-e^x}$$

Note: $\xi$ завжди $>0$. Розглядаємо лише $x>0$.

\section*{№ 9.26}
\begin{mdframed}
    $\xi$ - вип.величина з щільністю $p(x)$
\end{mdframed}

\subsection*{1. $\eta = -5\xi + 4$}

$$F_\eta(x) = P(-5\xi+4 \le x) = P(\xi \ge -\frac{x-4}{5}) = 1 - F(-\frac{x-4}{5})$$
$$p_\eta(x) = \frac{1}{5} p(-\frac{x-4}{5})$$

\subsection*{2. $\eta = \xi^3$}
$$F_\eta(x) = P(\xi^3 \le x) = P(\xi \le \sqrt[3]{x}) = F(x^{1/3})$$
$$p_\eta(x) = \frac{1}{3} x^{-2/3} p(\sqrt[3]{x})$$

\subsection*{3. $\eta = 1/\xi$}

$$F_\eta(x) = P(1/\xi \le x) = P(\xi \ge 1/x) = 1 - P(\xi \le 1/x) = 1-F(1/x)$$
$$p_\eta(x) = +\frac{1}{x^2} p(1/x)$$

\subsection*{4. $\eta =-\xi^4-4\xi^2$}

$$F_\eta(x) = P(-\xi^4-4\xi^2 \le x)$$

$$P(-\xi^4 - 4\xi^2 \le x) = \begin{cases}
    1,& x\ge0 \quad \text{бо ліва частина недодатня}\\
    P(\xi^4 + 4\xi^2 - |x| \ge 0)&, x<0
\end{cases}$$

$$D = 16 + 4|x| = 4(4+|x|)$$
$$\xi_0 = -2 \pm \sqrt{4+|x|}$$
$$\xi^4 + 4\xi^2 - |x| = (\xi^2 - (-2 + \sqrt{4+|x|}))\cdot (\xi^2 - (-2 - \sqrt{4+|x|})) \ge 0$$

$$\xi^2 \notin (-2 - \sqrt{4+|x|}; -2 + \sqrt{4+|x|})$$
Тобто 
$$\xi^2 > -2 + \sqrt{4+|x|}, \quad \text{бо } \xi^2 \ge 0$$
Тобто
$$|\xi| > \sqrt{\sqrt{4+|x|}-2}$$
$$\xi \notin [-\sqrt{\sqrt{4+|x|}-2},\sqrt{\sqrt{4+|x|}-2}]$$

$$F_\eta(x) = \begin{cases}
    1,& x\ge0\\
    1-F(\sqrt{\sqrt{4+|x|}-2}) + F(-\sqrt{\sqrt{4+|x|}-2})&, x<0
\end{cases} = $$
$$= \begin{cases}
    1,& x\ge0\\
    1-F(\sqrt{\sqrt{4-x}-2}) + F(-\sqrt{\sqrt{4-x}-2})&, x<0
\end{cases}$$

$$p_\eta(x) |_{x\ge0} = 1$$
\begin{multline*}
    p_\eta(x) |_{x<0} = -\frac{1}{2\sqrt{\sqrt{4-x}-2}}\cdot \frac{1}{2\sqrt{4-x}} \cdot (-1) p(\sqrt{\sqrt{4-x}-2}) - \\
    - \frac{1}{2\sqrt{\sqrt{4-x}-2}}\cdot \frac{1}{2\sqrt{4-x}} \cdot (-1) p(-\sqrt{\sqrt{4-x}-2}) = \\
= \frac{p(\sqrt{\sqrt{4-x}-2}) + p(\sqrt{\sqrt{4-x}-2})}{4\sqrt{\sqrt{4-x}-2}\sqrt{4-x}}
\end{multline*}

$$p_\eta(x) = \begin{cases}
    \displaystyle \frac{p(\sqrt{\sqrt{4-x}-2}) + p(\sqrt{\sqrt{4-x}-2})}{4\sqrt{\sqrt{4-x}-2}\cdot \sqrt{4-x}}, & x<0\\
    1,& x\ge 0
\end{cases}$$

\subsection*{5. $\eta =|\xi-2|$}

$$F_\eta(x) = P(|\xi-2| < x) = \begin{cases}
    P(\xi \in [2-x;2+x]),& x\ge0\\
    0, x<0
\end{cases}$$

$$P(\xi \in [2-x;2+x]) = F(2+x)-F(2-x)$$
$$p_\eta(x) = p(2+x)+p(2-x)$$

$$p_\eta(x) = \begin{cases}
    p(2+x)+p(2-x),& x\ge0\\
    0, x<0
\end{cases}$$

\section*{№ 9.27}
\begin{mdframed}
    $$S \sim Exp(2) \text{ - площа правильного шестикутника}$$
    $$p_S(x) = 2e^{-2x}\cdot \mathbb{1}(x\ge 0);\quad F_S(x) = \left(1-e^{-2x}\right)\cdot \mathbb{1}(x\ge 0)$$

    $$S = \frac{3\sqrt{3}}{2} a^2 $$
    $$a = \sqrt{\frac{2}{3\sqrt{3}} S} \text{ - сторона шестикутника}$$

    $$p_a(x) - ?$$
\end{mdframed}

$$F_a(x) = P(\sqrt{\frac{2}{3\sqrt{3}} S} < x) = P(\frac{2}{3\sqrt 3} S < x^2) 
= P(S < \frac{3\sqrt 3}{2}x^2 ) = F_S(\frac{3\sqrt 3}{2}x^2)$$
$$p_a(x) = 2x\cdot \frac{3\sqrt 3}{2} \cdot p_S(\frac{3\sqrt 3}{2}x^2) $$
$$p_a(x) = 2\sqrt{3} \; x e^{-3\sqrt 3 x^2} \cdot \mathbb{1}(x\ge 0)$$

\section*{№ 9.28}
\begin{mdframed}
    Знайти щільність розподілу $p_S$ величини
    $$S(\xi,\eta) = \frac{\sqrt 3}{4}\left(\xi^2 + \eta^2\right)$$

    Якщо 
    $$\xi, \eta - \text{ незалежні }$$
    $$\xi \sim U[0,2]; \quad \implies p_\xi(x) = \begin{cases}
        \frac{1}{2},& x\in [0,2]\\
        0, x\notin [0,2]
    \end{cases}$$
    $$\eta = \begin{cases}
        1,& p=1/4\\
        -1,& p=3/4\\
    \end{cases}$$
\end{mdframed}

Можемо виділити повний набір гіпотез: $A=\{\eta=1\}$ та $B=\{\eta=-1\}$.
$$P(A) = 1/4;\quad P(B) = 3/4$$

$$F_S(x) = P(S<x) = P(S<x \;|\; A)P(A) + P(S<x \;|\; B)P(B)$$

$$P(S<x \;|\; A) = P(S<x \;|\; B) = P(\frac{\sqrt 3}{4}\left(\xi^2 + 1^2\right) < x) 
= P(\xi < \sqrt{\frac{4}{\sqrt 3}x - 1}) = F_{\xi}(\sqrt{\frac{4}{\sqrt 3}x - 1})$$

$\eta$ входить в вираз лише під квадратом - знак не важливий. Тому
$$P(S<x \;|\; A) = P(S<x \;|\; B) = P(S<x) = F(x)$$

$$F_S(x) = F_{\xi}(\sqrt{\frac{4}{\sqrt 3}x - 1})$$
$$p_S(x) = \frac{1}{2\sqrt{\frac{4}{\sqrt 3}x - 1}} \cdot \frac{4}{\sqrt 3} \cdot p(\sqrt{\frac{4}{\sqrt 3}x - 1}) $$
$$p_S(x) = \frac{1}{\sqrt{\sqrt{3} x - 3}} \cdot \frac{1}{2} \; \cdot \mathbb{1}\left(\sqrt{\frac{4}{\sqrt 3}x - 1} \in [0,2]\right) $$
$$p_S(x) = \frac{1}{\sqrt{\sqrt{3} x - 3}} \cdot \frac{1}{2} \; \cdot \mathbb{1}\left(x \in [\frac{\sqrt 3}{4},5\frac{\sqrt 3}{4}]\right) $$



\section*{№ 11.18}
\begin{mdframed}
    $\xi$ та $\eta$ - однаково розподілені в.в. з щільністю
    $$p(x) = \frac{1}{2}e^{-|x|}$$

    Знайти щільність розподілу їх суми.
\end{mdframed}

$$p'(x) = \int_{-\infty}^\infty p(y)p(x-y) dy =
 \frac{1}{4} \int_{-\infty}^\infty e^{-|y|}e^{-|x-y|} dy = $$

$$\int_{-\infty}^\infty e^{-|y|}e^{-|x-y|} dy = \begin{cases}
    \int_{-\infty}^x e^{+y}e^{+x-y} dy + \int_{x}^0 e^{+y}e^{-x+y} dy 
    + \int_{0}^{+\infty} e^{-y}e^{-x+y} dy ,& x<0 \\
    \int_{-\infty}^0 e^{+y}e^{+x-y} dy + \int_{0}^x e^{-y}e^{+x-y} dy 
    + \int_{x}^{+\infty} e^{-y}e^{-x+y} dy ,& x\ge 0 
\end{cases}$$

$$\int_{-\infty}^\infty e^{-|y|}e^{-|y-x|} dy = \begin{cases}
    \int_{-\infty}^x e^{x} dy + \int_{x}^0 e^{x} dy 
    + \int_{0}^{+\infty} e^{-2y}e^{x} dy ,& x<0 \\
    \int_{-\infty}^0 e^{2y}e^{-x} dy + \int_{0}^x e^{-x} dy 
    + \int_{x}^{+\infty} e^{-2y}e^{x} dy ,& x\ge 0 
\end{cases}$$

$$\int_{-\infty}^\infty e^{-|y|}e^{-|y-x|} dy = \begin{cases}
    \frac{1}{2}e^{2x}e^{-x} - xe^{x} 
    + \frac{1}{2} \cdot 1 \cdot e^{x} ,& x<0 \\
    \frac{1}{2} \cdot 1 \cdot e^{-x} + xe^{-x}
    + \frac{1}{2}e^{-2x}e^{x} ,& x\ge 0 
\end{cases}$$

$$\int_{-\infty}^\infty e^{-|y|}e^{-|y-x|} dy = \begin{cases}
    e^{x}(-x+1) ,& x<0 \\
    e^{-x} (x+1) ,& x\ge 0
\end{cases}$$

$$p'(x) = \begin{cases}
    \frac{1}{4} e^{x} (-x+1) ,& x<0 \\
    \frac{1}{4} e^{-x} (x+1) ,& x\ge 0
\end{cases}$$

Тобто

$$p'(x) = \frac{1}{4} e^{-|x|}(1+|x|)$$

\begin{mdframed}[backgroundcolor=violet!10]
    При чому:
    $$\zeta = \eta + \xi = 2\xi$$
    $$F_\zeta(x) = P(\zeta \le x) = P(\xi \le \frac{x}{2}) = F(x/2)$$
    $$p_\zeta(x) = \frac{1}{2} p(x/2) = \frac{1}{4} e^{-|x/2|}$$

    $$p'(x) = \begin{cases}
        \frac{1}{4} e^{x/2}, & x<0\\
        \frac{1}{4} e^{-x/2}, & x\ge 0
    \end{cases}$$
\end{mdframed}

% TODO: додати пропущений пункт в 9.26
% TODO: пофіксити 9.26.3 - додати обернену заміну


\end{document}

