% !TEX TS-program = xelatex
% !TEX encoding = UTF-8

\documentclass[11pt, a4paper]{article} % use larger type; default would be 10pt

\usepackage{fontspec} % Font selection for XeLaTeX; see fontspec.pdf for documentation
\defaultfontfeatures{Mapping=tex-text} % to support TeX conventions like ``---''
\usepackage{xunicode} % Unicode support for LaTeX character names (accents, European chars, etc)
\usepackage{xltxtra} % Extra customizations for XeLaTeX
\usepackage{tikz}
\usetikzlibrary{arrows,calc,patterns}

\setmainfont[Ligatures=TeX]{Garamond} % set the main body font (\textrm), assumes Charis SIL is installed
%\setsansfont{Deja Vu Sans}
\setmonofont[Ligatures=TeX]{Fira Code}

% other LaTeX packages.....
\usepackage{fullpage}
\usepackage[top=2cm, bottom=4.5cm, left=2.5cm, right=2.5cm]{geometry}
\usepackage{amsmath,amsthm,amsfonts,amssymb,amscd,systeme}
\usepackage{mathtools}
\usepackage{unicode-math}
\usepackage{cancel}
\geometry{a4paper} 
%\usepackage[parfill]{parskip} % Activate to begin paragraphs with an empty line rather than an indent
\usepackage{fancyhdr}
\usepackage{listings}
\usepackage{graphicx}
\usepackage{hyperref}
\usepackage{multicol}

\renewcommand\lstlistingname{Algorithm}
\renewcommand\lstlistlistingname{Algorithms}
\def\lstlistingautorefname{Alg.}
\lstdefinestyle{mystyle}{
    % backgroundcolor=\color{backcolour},   
    % commentstyle=\color{codegreen},
    % keywordstyle=\color{magenta},
    % numberstyle=\tiny\color{codegray},
    % stringstyle=\color{codepurple},
    basicstyle=\ttfamily\footnotesize,
    breakatwhitespace=false,         
    breaklines=true,                 
    captionpos=b,                    
    keepspaces=true,                 
    numbers=left,                    
    numbersep=5pt,                  
    showspaces=false,                
    showstringspaces=false,
    showtabs=false,                  
    tabsize=2
}
\lstset{style=mystyle}

\newcommand\course{5 - Теорія ймовірності}
\newcommand\hwnumber{ДЗ №14}                   % <-- homework number
\newcommand\idgroup{ФІ-91}                
\newcommand\idname{Михайло Корешков}  

\usepackage[framemethod=TikZ]{mdframed}
\mdfsetup{%
	backgroundcolor = black!5,
}
\mdfdefinestyle{ans}{%
    backgroundcolor = green!5,
    linecolor = green!50,
    linewidth = 1pt,
}

\pagestyle{fancyplain}
\headheight 35pt
\lhead{\idgroup \\ \idname}
\chead{\textbf{\Large \hwnumber}}
\rhead{\course \\ \today}
\lfoot{}
\cfoot{}
\rfoot{\small\thepage}
\headsep 1.5em

\linespread{1.2}

\begin{document}

% 13.14,13.15,13.21; 
% 14.21,14.22,14.25,14.26,14.27

\section*{№ 13.4}
(зробив випадково)
\begin{mdframed}
    $\{\xi_k\}$ - незалежні.
    $P(\xi_k = 1) = p;\quad P(\xi_k = 0) = 1-p$

    Prove:
    $$P(\sum_{k=0}^\infty \xi_n = +\infty) = 1$$
\end{mdframed}

$$\sum_{k=0}^\infty \xi_n = +\infty$$
тоді і тільки тоді, коли нескінченно багато $\xi_k = 1$.
\begin{proof}
    Розглянемо
    $A_n = \{\xi_n = 1\}; A = \limsup_{n\to\infty} A_n$.
    
    1. $$\sum_{n=1}^\infty P(A_n) = \sum_{n=1}^\infty p = +\infty \; (p=const >0)$$
    2. $A_n$ - незалежні, бо $\xi_n$ - незалежні
    Умови леми Бореля-Кантеллі виконані. Маємо, що 
    $$P(\limsup_{n\to\infty} A_n) = 1$$
    Тобто 
    $$P(\sum_{k=0}^\infty \xi_n = +\infty) = 1$$
\end{proof}

\section*{№ 13.14}
Розглядаємо послідовність н.о.р.в.в. $\{\xi_n\}$.
$$P(\xi_n = k) = 1/6,\; k = \overline{1,6}$$

Розглядаємо події $A_n = \{\xi_{6n} = 1, \xi_{6n+1} = 2, ..., \xi_{6n+5} = 6\}$
та послідовність $\{A_n\}$.

"$A_n$ зустрiнеться скiнченну кiлькiсть разiв" = "після якогось n подія $\overline{A_n}$ відбувається завжди".
А це буде подія $A = \liminf_{n\to\infty} \overline{A_n}$

$$P(A) - ?$$

$$P(\liminf_{n\to\infty} \overline{A_n}) = 1 - P(\limsup_{n\to\infty} A_n)$$

$$P(A_n) = \frac{1}{6^6} = const$$
$$\sum{n=1}^\infty P(A_n) = \infty$$
$$A_n - \text{незалежні, бо серії не перетинаються}$$
Умови леми Бореля-Кантеллі виконані. 
$$P(\limsup_{n\to\infty} A_n) = 1$$

\begin{mdframed}[style=ans]
    $$P(A) = 1 - P(\limsup_{n\to\infty} A_n) = 0$$
\end{mdframed}

\section*{№ 13.15}
$$P(\liminf_{n\to\infty} \xi_n = 0) - ?$$

\subsection*{a)}
$$P(\xi_n = 0) = P(\xi_n = 1) = \frac{1}{2}$$

$$P(\liminf_{n\to\infty} \xi_n = 0) = P(\exists N: \forall n>N: \xi_n = 0) =$$
$$= P(\liminf_{n\to\infty} \{\xi_n = 0\})$$

$$P(\xi_n = 1) = 1/2 = const.\quad \sum P(\xi_n = 1) = \infty$$
$\xi_n$ - незалежні. Умови леми Бореля-Кантеллі виконані.
$$P(\limsup_{n\to\infty}\xi_n = 1) = 1$$
$$P(\liminf_{n\to\infty} \xi_n = 0) = 1 - P(\limsup_{n\to\infty}\xi_n = 1) = 0$$

\subsection*{b)}
$$P(\xi_n = 1) = P(\xi_n = \frac{1}{n}) = \frac{1}{n}$$
$$P(\xi_n = 2 ) = 1 - \frac{2}{n}$$

$$P(\liminf_{n\to\infty} \xi_n = 0) = P(\exists N: \forall n>N: \xi_n = \frac{1}{n}) =$$
$$= P(\liminf_{n\to\infty} \{\xi_n = 1/n\})$$
бо інакше буде нескінченно багато інших випадків.

$$A_n = \{\xi_n \ne 1/n\}$$
$$P(A_n) = 1-1/n$$
$$\sum_{n=1}^\infty P(A_n) = \infty, \text{ бо ряд навіть не прямує до 0}$$
$A_n$ - незалежні. Умови леми Бореля-Кантеллі виконані.
$$P(\limsup_{n\to\infty}A_n) = 1$$
$$P(\liminf_{n\to\infty} \{\xi_n = 1/n\}) = 1 - P(\limsup_{n\to\infty}A_n) = 0$$

\subsection*{c)}
$$P(\xi_n = 0) = \frac{1}{2^n},\quad P(\xi_n=1) = 1 - \frac{1}{2^n}$$
$$P(\liminf_{n\to\infty} \xi_n = 0) = P(\liminf_{n\to\infty} \{\xi_n = 0\}) $$

$$\sum_{n=1}^\infty \{\xi_n \ne 0\} = \sum_{n=1}^\infty \left(1-\frac{1}{2^n}\right) = \infty$$
$A_n$ - незалежні
Умови леми Бореля-Кантеллі виконані.
$$P(\limsup_{n\to\infty} \{\xi_n \ne 0\}) = 1$$
$$P(\liminf_{n\to\infty} \xi_n = 0) = 1 - P(\limsup_{n\to\infty} \{\xi_n \ne 0\}) = 0$$


\section*{№ 13.21}

\begin{mdframed}
    $$\{A_n\}_{n\ge 1}, \quad A_n \sim U[0,1]$$
    $$A \in [0,1]$$
    $$d_n = \min\{|A_1-A|, ... , |A_n-A|\}$$
    Довести:
    $$P(d_n \xrightarrow[n \to \infty]{} 0) = 1$$
\end{mdframed}

% $$P(d_n \xrightarrow[n \to \infty]{} 0) = P(\min\{|A_1-A|, ... , |A_n-A|\} \xrightarrow[n \to \infty]{} 0) = $$
% $$= P(\forall \varepsilon>0: \exists N:\; \forall n>N: \min\{|A_1-A|, ... , |A_n-A|\} < \varepsilon) =$$
% $$= P(\forall \varepsilon>0: \exists k: |A_k-A| < \varepsilon)$$

\begin{proof}
    
    Розглянемо 
    $$B_n = |A_n-A|$$
    Нам потрібно довести, що (a.s. = almost surely = майже напевне)
    $$d_n \xrightarrow[n \to \infty]{a.s.} 0$$
    
    За достатньою умовою збіжності необхідно щоб
    $$\forall \varepsilon>0: \sum_{k=1}^\infty P(d_k > \varepsilon) < \infty$$
    
    $$P(d_n > \varepsilon) = P(\forall k=\overline{1,n}: A_k > A + \varepsilon \vee A_k < A - \varepsilon) = $$
    $$= P(A_k > A + \varepsilon \vee A_k < A - \varepsilon)^n = (1-2\varepsilon)^n$$
    
    $$\sum_{k=1}^\infty P(d_k > \varepsilon) = \sum_{k=1}^\infty (1-2\varepsilon)^n = \frac{1-2\varepsilon}{2\varepsilon} < \infty$$
    
    Достатня умова виконана. $$d_n \xrightarrow[n \to \infty]{a.s.} 0$$
\end{proof}


\section*{№ 14.21}
$$\xi_n \sim Exp(n);\quad f_{\xi_n}(x) = ne^{-nx} $$
Prove:
$$\xi_n \xrightarrow[n\to\infty]{L^2} 0$$

\begin{proof}
    $$\xi_n \xrightarrow[n\to\infty]{L^2} 0 \iff$$
    $$\iff M|\xi_n|^2 \xrightarrow[n\to\infty]{}0$$
    
    $$M|\xi_n|^2 = \int_0^\infty x^2 ne^{-nx}dx = -\int_0^\infty x^2 e^{-nx} = - \cancelto{0}{x^2e^{-nx}}|_0^\infty + 2\int_0^\infty xe^{-nx}dx = $$
    $$= -\frac{2}{n} \cancelto{0}{x^2e^{-nx}}|_0^\infty + \frac{2}{n^2}\int_0^\infty e^{-nx} dx = \frac{2}{n^2}$$
    
    $$M|\xi_n|^2 = \frac{2}{n^2} \xrightarrow[n\to\infty]{}0$$
\end{proof}

% 14.22,14.25,14.26,14.27

\section*{№ 14.22}
$$\xi_n \sim N(0,1/n^2)$$
$$f_{\xi_n}(x) = \frac{n^2}{\sqrt{2 \pi}} e^{-\frac{n^2x^2}{2}}$$
Prove:
$$\xi_n \xrightarrow[n\to\infty]{a.s.}0$$

\begin{proof}
    
    Необхідно довести:
    $$\forall \varepsilon>0: \sum_{k=1}^\infty P(\xi_n>\varepsilon) < \infty$$
    З нерівності Чебишова
    $$P(\xi_n > \varepsilon) = P(\xi_n \ge \varepsilon) \le \frac{M\xi_n}{\varepsilon} = 0$$
    
    $$\forall \varepsilon>0: \sum_{k=1}^\infty P(\xi_n>\varepsilon) = 0$$
    Достатні умови збіжності майже напевне виконані
\end{proof}

\section*{№ 14.25}
$$M\xi^2_n = \sigma^2, M\xi_n = 0$$
Довести:
$$\frac{S_n}{n} \xrightarrow[n\to\infty]{P}0,\quad S_n=\sum_{k=1}^n \xi_n$$

\begin{proof}
    З зв'язку між видами збіжності ми знаємо, що для збіжності за ймовірністю достатня $L^p$-збіжність 

    $$M\frac{S_n^2}{n^2} = \frac{M\left(\sum_{k=1}^n \xi_n\right)^2}{n^2} \le \frac{\sum_{k=1}^n M\xi^2_n}{n^2} = \frac{n\sigma^2}{n^2} = \frac{\sigma^2}{n}$$
    $$M\frac{S_n^2}{n^2} \xrightarrow[n\to\infty]{L^2} \frac{\sigma^2}{n} \xrightarrow[n\to\infty]{L^2} 0$$
\end{proof}
Можна було також через закон великих чисел з обмеженості маточікування та дисперсії

\section*{№ 14.26}
Prove:
$$\sum_{k=1}^\infty M\xi^2_k < \infty \implies \left(\xi_n \xrightarrow[n\to\infty]{a.s.} 0\right)$$

\begin{proof}
    Необхідно довести:
    $$\forall \varepsilon>0: \sum_{k=1}^\infty P(\xi_n>\varepsilon) < \infty$$

    З нерівності Чебишова:
    $$P(\xi_n>\varepsilon) \le P(\xi_n^2>\varepsilon^2) \le \frac{M\xi_n^2}{\varepsilon^2}$$
    
    $$\sum_{k=1}^\infty P(\xi_n>\varepsilon) \le \sum_{k=1}^\infty \frac{M\xi_k^2}{\varepsilon^2} < \infty$$
    достатньо
\end{proof}

\section*{№ 14.27}
$\{\xi_n\} - $ незалежні в.в.
Prove:
$$\left(\xi_n \xrightarrow[n\to\infty]{a.s.} 0 \right) \implies \sum_{k=1}^\infty P(|\xi_k|\ge 1) < \infty$$

\begin{proof}
    $$P(\xi_n \to 0) = 1$$
    
    Припустимо, що $$\sum_{k=1}^\infty P(|\xi_k|\ge 1) = \infty$$
    Тоді з незалежності та з леми Бореля-Кантеллі випливає, що 
    $$P(\limsup_{n\to\infty} \{|\xi_n|\ge 1\}) = 1$$
    Тобто
    $$\exists N: \forall n>N: |\xi_n|\ge 1$$
    Але тоді $$\xi_n \cancel{\longrightarrow} 0$$

    Протиріччя.
    Значить, $$\sum_{k=1}^\infty P(|\xi_k|\ge 1) < \infty$$
\end{proof}

\end{document}

