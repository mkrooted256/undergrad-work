% !TEX TS-program = xelatex
% !TEX encoding = UTF-8

\documentclass[11pt, a4paper]{article} % use larger type; default would be 10pt

\usepackage{fontspec} % Font selection for XeLaTeX; see fontspec.pdf for documentation
\defaultfontfeatures{Mapping=tex-text} % to support TeX conventions like ``---''
\usepackage{xunicode} % Unicode support for LaTeX character names (accents, European chars, etc)
\usepackage{xltxtra} % Extra customizations for XeLaTeX
\usepackage{tikz}
\usetikzlibrary{arrows,calc,patterns}

\setmainfont[Ligatures=TeX]{Garamond} % set the main body font (\textrm), assumes Charis SIL is installed
%\setsansfont{Deja Vu Sans}
\setmonofont[Ligatures=TeX]{Fira Code}

% other LaTeX packages.....
\usepackage{fullpage}
\usepackage[top=2cm, bottom=4.5cm, left=2.5cm, right=2.5cm]{geometry}
\usepackage{amsmath,amsthm,amsfonts,amssymb,amscd,systeme}
\usepackage{cancel}
\geometry{a4paper} 
%\usepackage[parfill]{parskip} % Activate to begin paragraphs with an empty line rather than an indent
\usepackage{fancyhdr}
\usepackage{listings}
\usepackage{graphicx}
\usepackage{hyperref}
\usepackage{multicol}

\renewcommand\lstlistingname{Algorithm}
\renewcommand\lstlistlistingname{Algorithms}
\def\lstlistingautorefname{Alg.}
\lstdefinestyle{mystyle}{
    % backgroundcolor=\color{backcolour},   
    % commentstyle=\color{codegreen},
    % keywordstyle=\color{magenta},
    % numberstyle=\tiny\color{codegray},
    % stringstyle=\color{codepurple},
    basicstyle=\ttfamily\footnotesize,
    breakatwhitespace=false,         
    breaklines=true,                 
    captionpos=b,                    
    keepspaces=true,                 
    numbers=left,                    
    numbersep=5pt,                  
    showspaces=false,                
    showstringspaces=false,
    showtabs=false,                  
    tabsize=2
}
\lstset{style=mystyle}

\newcommand\course{5 - Теорія ймовірності}
\newcommand\hwnumber{ДЗ №8}                   % <-- homework number
\newcommand\idgroup{ФІ-91}                
\newcommand\idname{Михайло Корешков}  

\usepackage[framemethod=TikZ]{mdframed}
\mdfsetup{%
	backgroundcolor = black!5,
}
\mdfdefinestyle{ans}{%
    backgroundcolor = green!5,
    linecolor = green!50,
    linewidth = 1pt,
}

\pagestyle{fancyplain}
\headheight 35pt
\lhead{\idgroup \\ \idname}
\chead{\textbf{\Large \hwnumber}}
\rhead{\course \\ \today}
\lfoot{}
\cfoot{}
\rfoot{\small\thepage}
\headsep 1.5em

\linespread{1.2}

\begin{document}

\section*{№ 7.13}
$$\Omega = [0; 1]$$
\subsection*{а)}
$$A = [1/3; 1/2]$$

$$
a([1/3; 1/2]) = \{\varnothing, [1/3; 1/2], [0;1/3)\cup (1/2; 1), [0;1]\}
$$

$$\sigma([1/3;1/2]) = a([1/3; 1/2]) \text{, бо алгебра $a(A)$ є скінченною}$$

\subsection*{б)}
$$B =  [0;1] \cap \mathbb{Q}$$
$$a(B) = \{\varnothing,   [0;1] \cap \mathbb{Q},  [0;1] \setminus \mathbb{Q}, [0;1]\}$$

$$\sigma(B) = a(B) \text{, бо алгебра $a(B)$ є скінченною}$$

\subsection*{в)}
$$a(\{\{0\}, \{1\}\}) = \{\varnothing, \{0\}, (0;1], \{1\}, [0;1), \{0,1\}, (0;1), [0;1]\}$$
$$\sigma(\{\{0\}, \{1\}\}) = a(\{\{0\}, \{1\}\}) \text{, бо алгебра є скінченною}$$

\begin{mdframed}
    До речі, кожна алгебра дійсно має $2^k$ елементів.
\end{mdframed}

\section*{№ 7.14}
\begin{mdframed}
    $$\Omega = \mathbb{R};\quad \mathcal{B} - \text{ Борелева $\sigma$-алгебра на } \mathbb{R}$$
    Довести: $$\mathbb{R} \setminus \mathbb{Q} \in \mathcal{B}$$
\end{mdframed}

Зауважу, що $\mathbb{R} \setminus \mathbb{Q} \in \mathcal{B} \iff \mathbb{Q} \in \mathcal{B}$
Тобто якщо $\mathbb{Q}$ - елемент Борелевої $\sigma$-алгебри, то й $\mathbb{R} \setminus \mathbb{Q}$ також

$$\mathbb{Q} = \{\pm \frac{n}{d}\;|\; n,d \in \mathbb{N}\} \sim \mathbb{N}^2$$
$$|\mathbb{Q}| = \aleph_0$$
$$\mathbb{Q} = \left.\{r_k\}\right._{k=0}^\infty = \bigcup_{k=0}^\infty \{r_k\}$$

Кожна множина $\{r_k\}$ точно в Борелевій алгебрі. Маємо зліченне об'єднання таких множин.
За визначенням $\sigma$-алгебри, 
$$\bigcup_{k=0}^\infty \{r_k\} \in \mathcal{B}$$
$$\mathbb{Q} \in \mathcal{B}$$
$$\mathbb{R} \setminus \mathbb{Q} \in \mathcal{B}$$
\qedsymbol


\section*{№ 7.15}
\begin{mdframed}
    $$\Omega = \mathbb{R}^2;\quad \mathcal{B} - \text{ Борелева $\sigma$-алгебра на } \mathbb{R}^2$$
    Довести: $$(\mathbb{R} \setminus \mathbb{Q}) \times \mathbb{Q} \in \mathcal{B}$$
\end{mdframed}

По-перше, 
$$(\mathbb{R} \setminus \mathbb{Q}) \times \mathbb{Q} = \Omega \setminus \mathbb{Q} \times \mathbb{Q} $$

$$|\mathbb{Q} \times \mathbb{Q}| = \aleph_0 \implies \mathbb{Q} \times \mathbb{Q} = \left.\{(p_k,q_k)\}\right._{k=1}^{\infty}$$
$$\forall k:\; \{(p_k,q_k)\} \in \mathcal{B} \text{ як закрита множина}$$
$$\mathbb{Q} \times \mathbb{Q} = \bigcup_{k=1}^\infty \{(p_k,q_k)\} \in \mathcal{B} \text{ як зліченне об'єднання елементів $\mathcal{B}$} $$

А отже $$(\mathbb{R} \setminus \mathbb{Q}) \times \mathbb{Q} \in \mathcal{B}$$ як доповнення до борелівської множини
\qedsymbol

\begin{mdframed}
    $$\Omega = \mathbb{R}^2;\quad \mathcal{B} - \text{ Борелева $\sigma$-алгебра на } \mathbb{R}^2$$
    Довести: 
    $$A = \{(x,y)\in\mathbb{R}^2 \;|\; \max(\sin(xy), \arctan(y-x)) > 0.1\} \in \mathcal{B}$$
\end{mdframed}

$\arctan(y-x)$ та $\sin(xy)$ обидві неперервні на $\mathbb{R}$.
$(y-x)$, $(xy)$ обидві неперервні на $\mathbb{R}^2$.
$\max(f(x,y),g(x,y))$ - неперервна якщо $f$ та $g$ неперервні.

Отже, $\varphi(x,y) = \max(\sin(xy), \arctan(y-x))$ - неперервна на $\mathbb{R}^2$.

Іншими словами, 
$$\forall U - \text{open in }\mathbb{R}:\; \varphi^{-1}(U) - \text{open in }\mathbb{R}^2$$

Нехай $U = (0.1,\infty)$. $U$ дійсно відкрита  множина в $\mathbb{R}$. \\
Тоді, за неперервністю, $\varphi^{-1}(U)$ - відкрита в $\mathbb{R}^2 \implies \varphi^{-1}(U) \in \mathcal{B}$.

А $\varphi^{-1}(U)$ це як раз $A$. \qedsymbol

\section*{№ 7.17}
\subsection*{a)}
\begin{mdframed}
    $\sigma$-алгебра породжена подіями нульової ймовірності.
\end{mdframed}
Нехай $\Omega$ - простір елементарних подій, $Z$ - сігма-алгебра, що нас цікавить.

$$\forall A \text{ - подія}:\; P(A)=0 \implies A\in Z$$

Розглянемо $\left.\{A_k\}\right._{k=0}^\infty:\; P(A_k)=0$. Очевидно, $A_k \in Z$.

$$P(\bigcup_k A_k) \le \sum_k P(A_k) = 0 \implies \bigcup_k A_k \in Z \; (\text{за субадитивністю ймовірнісної міри})$$ 
Також для довільної $A:\; P(A)=0$
$$P(\overline{A}) = 1-P(A) = 1$$
Тобто 
$$\forall A':\; P(A')=1 \implies A' \in Z$$

Зрозуміло, що для $\{A_k:\; A_k \in Z\}$: $P(\bigcup_k A_k) = \begin{cases}
    1, \exists k:\; P(A_k) = 1\\
    0, \forall k:\; P(A_k) = 0
\end{cases}$. 
Тобто $\bigcup_k A_k \in Z$\\

Також, звичайно, $\Omega \in Z, \varnothing \in Z$.\\

Можемо ствержувати, що $\sigma$-алгебра $Z = \{A \subset \Omega:\; P(A) = 0 \vee P(A) = 1\}$ -
дійсно $\sigma$-алгебра.

Залишається довести, що $Z$ - мінімальна. 
Елементи $X\in Z:\; P(X)=0$ прибирати не можемо, бо тоді $Z$ не буде породжена подіями нульової ймовірності.

Нехай $\exists A:\; A \in Z \;\wedge\; P(A)=1\;\wedge\; Z' = Z \setminus \{A\} $ - також алгебра, що містить всі події нульової ймовірності.
Тоді $\exists A' = \overline{A}:\; P(A')=0$. За визначенням, $A'\in Z'$. Але тоді, за визначенням алгебри, $A = \overline{A'}\in Z'$.
Протиріччя. Тому $Z$ - дійсно мінімальна.

Отже, $Z = \{A \in \Omega:\; P(A) = 0 \vee P(A) = 1\} = \sigma({A \subset \Omega:\; P(A) = 0})$ \qedsymbol.

\subsection*{б)}
Все аналогічно \textbf{а)}. 
$$\sigma(\{A\subset\Omega:\; P(A)=1\}) = Z$$
\pagebreak

\section*{№ 7.18}
\begin{mdframed}
    Нехай $S = \{A_n\}$ - послідовність подій.
    Довести:
    $$\underline{\lim} A_n \in \sigma(S)$$
\end{mdframed}

$$\underline{\lim} A_n = \bigcup_{i=1}^\infty \bigcap_{j=i}^\infty A_n$$

$$\forall j:\; B_j = \bigcap_{j=i}^\infty A_n \in \sigma(S) \text{ як зліченний перетин}$$
$$\bigcup_{i=1}^\infty \bigcap_{j=i}^\infty A_n = \bigcup_{i=1}^\infty B_j \in \sigma(S) \text{ як зліченне об'єднання}$$ 

Отже, $\underline{\lim} A_n = \bigcup_{i=1}^\infty \bigcap_{j=i}^\infty A_n \in \sigma(S)$ \qedsymbol

\section*{№ 7.19}
\begin{mdframed}
    $$S = \left.\{\mathcal{A}_n\}\right._{n=1}^\infty$$
    $$\mathcal{A}_n - \sigma\text{-алгебри},\quad\mathcal{A}_n \subseteq \mathcal{A}_{n+1}$$

    Довести:
    $$\mathcal{A} = \bigcup_{n=1}^\infty \mathcal{A}_n - \text{алгебра}$$
\end{mdframed}

\begin{enumerate}
    \item $\varnothing \in \mathcal{A}$, бо $\forall k:\; \varnothing \in \mathcal{A}_k$
    \item $\Omega \in \mathcal{A}$, бо $\forall k:\; \Omega \in \mathcal{A}_k$
    \item Нехай $A \in \mathcal{A}$. 
    Це означає, $\exists k:\; A \in \mathcal{A}_k$.\\
    З цього, $\overline{A} \in \mathcal{A}_k$. 
    А тому $\overline{A} \in \mathcal{A}$
    \item Нехай $A,B \in \mathcal{A}$. 
    Це означає, $\exists i,j:\; A \in \mathcal{A}_i, B \in \mathcal{A}_j$.\\
    Нехай без зменшення загальності $i < j$. 
    Тоді з того, що послідовність неспадна, випливає
    $A,B \in \mathcal{A}_j$. \\
    Оскільки $\mathcal{A}_j$ - алгебра,
    $A \cap B \in \mathcal{A}_j$.
    А тому $A \cap B \in \mathcal{A}$
\end{enumerate}

Отже, всі властивості алгебри доведено. $\mathcal{A}$ - дійсно алгебра \qedsymbol.


\end{document}

