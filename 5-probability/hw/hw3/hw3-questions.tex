% !TEX TS-program = xelatex
% !TEX encoding = UTF-8

\documentclass[11pt, a4paper]{article} % use larger type; default would be 10pt

\usepackage{fontspec} % Font selection for XeLaTeX; see fontspec.pdf for documentation
\defaultfontfeatures{Mapping=tex-text} % to support TeX conventions like ``---''
\usepackage{xunicode} % Unicode support for LaTeX character names (accents, European chars, etc)
\usepackage{xltxtra} % Extra customizations for XeLaTeX
\usepackage{tikz}

\setmainfont[Ligatures=TeX]{Garamond} % set the main body font (\textrm), assumes Charis SIL is installed
%\setsansfont{Deja Vu Sans}
\setmonofont[Ligatures=TeX]{Fira Code}

% other LaTeX packages.....
\usepackage{fullpage}
\usepackage[top=2cm, bottom=4.5cm, left=2.5cm, right=2.5cm]{geometry}
\usepackage{amsmath,amsthm,amsfonts,amssymb,amscd,systeme}
\geometry{a4paper} 
%\usepackage[parfill]{parskip} % Activate to begin paragraphs with an empty line rather than an indent
\usepackage{fancyhdr}
\usepackage{listings}
\usepackage{graphicx}
\usepackage{hyperref}
\usepackage{multicol}

\renewcommand\lstlistingname{Algorithm}
\renewcommand\lstlistlistingname{Algorithms}
\def\lstlistingautorefname{Alg.}
\lstdefinestyle{mystyle}{
    % backgroundcolor=\color{backcolour},   
    % commentstyle=\color{codegreen},
    % keywordstyle=\color{magenta},
    % numberstyle=\tiny\color{codegray},
    % stringstyle=\color{codepurple},
    basicstyle=\ttfamily\footnotesize,
    breakatwhitespace=false,         
    breaklines=true,                 
    captionpos=b,                    
    keepspaces=true,                 
    numbers=left,                    
    numbersep=5pt,                  
    showspaces=false,                
    showstringspaces=false,
    showtabs=false,                  
    tabsize=2
}
\lstset{style=mystyle}

\newcommand\course{5 - Теорія ймовірності}
\newcommand\hwnumber{Контрольні питання дл заняття 4}                   % <-- homework number
\newcommand\idgroup{ФІ-91}                
\newcommand\idname{Михайло Корешков}  

\usepackage[framemethod=TikZ]{mdframed}
\mdfsetup{%
	backgroundcolor = black!5,
}
\mdfdefinestyle{ans}{%
    backgroundcolor = green!5,
    linecolor = green!50,
    linewidth = 1pt,
}

\pagestyle{fancyplain}
\headheight 35pt
\lhead{\idgroup \\ \idname}
\chead{\textbf{\Large \hwnumber}}
\rhead{\course \\ \today}
\lfoot{}
\cfoot{}
\rfoot{\small\thepage}
\headsep 1.5em

\linespread{1.2}

\begin{document}

\section*{Формула для обчислення геометричної ймовірності}
$$
\begin{tikzpicture}
    \draw plot [smooth cycle] coordinates {(-1,-1)(0,2)(4,3)(5,0)(2,-2)} node at (0,0) {$\Omega$};
    \draw plot [smooth cycle] coordinates {(1,1)(1,2)(2,2)(3.5,0)(2,-1)} node at (2,1) {$A$};
\end{tikzpicture}
$$
Нехай $\Omega$ - простір елементарних подій, $A$ - деяка подія. 
Якщо це частина геометричного ймовірнісного експерименту, то ймовірність
події $A$ обчислюватиметься як відношення "об'єму" $A$ до "об'єму" всього простору $\Omega$.

"Об'єм" у нашому випадку визначатиметься \textbf{мірою} $\mu(A)$.
$$
P(A) = \frac{\mu(A)}{\mu(\Omega)}
$$

Для Декартових випадків, мірою будуть просто довжина / площа / об'єм / гіпероб'єм. 
\begin{mdframed}
    Зауваження: деякі $A$ матимуть міру $0$. 
    Наприклад: точка на прямій, пряма на площині, площина в просторі.
\end{mdframed}


\section*{Принцип відбиття}
Це метод обчислення сприятливих подій у задачах блукання з дискретним часом та двома напрямами руху.

Розглянемо наступну модель. 
Зроблено $x$ кроків, система почала в координаті $y_0$ та закінчила в $y$.
$$\Omega = \{(\varepsilon_1, \varepsilon_2, ..., \varepsilon_x) \;|\; 
\varepsilon_i \in \{-1,1\} \;\wedge\; y_0 + \sum_i \varepsilon_i = y \}$$

Розглянемо шляхи, що перетинають чи дотикаються $OX$. Тобто
$$A = \{\bar{\varepsilon} \in \Omega \;|\; \exists k:\;\sum_{i=1}^k \omega_i = 0\}$$

Позначимо кількість шляхів з $(x_0, y_0)$ в $(x,y)$ як $\quad L_{x0,y0}(x,y)$,\\ 
а кількість таких шляхів, що перетинають $OX$ - $\quad L_{x0,y0}^x(x,y)$

\pagebreak
Можна побудувати взаємооднозначну відповідність між шляхами, що перетинають $OX$
та цими ж шляхами, але відображеними до точки першого дотику до осі.
\begin{itemize}
    \item Червоний шлях - перетинає $OX$
    \item Синій шлях - відповідний відображений
\end{itemize}

Оскільки множини шляхів скінченні, бієкція між ними означає, що 
\begin{mdframed}[style='ans']  
$$
L_{(0,y0)}^x(x,y) = L_{(0,-y0)}(x,y)
$$
Ця формула, власне, і є суттю \textbf{принципу відбиття}
\end{mdframed}

$$
\begin{tikzpicture}
    \draw[step=0.5cm,gray,very thin] (0,-4) grid (10,4);
    \draw[thick, ->] (0,0) -- (10.5,0) node[right] {$x$, кроки};
    \draw[thick, ->] (0,-4) -- (0,4) node[above] {$y$, координата системи};
    \draw[red] (0,3) -- (2,1) -- (3,2) -- (5,0) -- (6.5,-1.5) -- (8,0) -- (10,2);
    \draw[dashed, blue] (0,-3) -- (2,-1) -- (3,-2) -- (5,0) -- (6.5,-1.5) -- (8,0) -- (10,2);
    \fill (10,2) node[right] {$A = (x, y)$} circle (2pt);
    \fill (0,3) node[above left] {$B = (0,y_0)$} circle (2pt);
    \fill[blue] (0,-3) node[above left] {$B' = (0,-y_0)$} circle (2pt);
    \fill[red] (5,0) circle (2pt);
\end{tikzpicture}
$$

\pagebreak
\section*{Математична модель задачі про зустріч}
Два агенти окремо приходять в рандомний час та чекають $t$ відносних одиниць часу.
Яка ймовірність, що вони зустрінуться?

$$\Omega = [0;1]^2$$
$$A = \{(x,y) \in \Omega \;|\; |y-x| < t \} \implies
\begin{cases}
    y < x + t \\
    x < y + t
\end{cases} \implies
\begin{cases}
    y < x + t \\
    y > x - t
\end{cases}
$$

Дану матмодель можна представити наступною геометричною побудовою:

$$
\begin{tikzpicture}
    \fill[green!25] (0,0) -- (1,0) -- (4,3) -- (4,4) -- (3,4) -- (0,1) -- cycle;
    \draw[step=0.5cm,gray,very thin] (0,0) grid (4,4);
    \draw[thick, ->] (0,0) node[below] {$0$} -- (4,0) node[right] {$x$}; 
    \node[below] at (4,0) {$1$};
    \node[left] at (0,4) {$1$};
    \draw[thick, ->] (0,0) -- (0,4) node[above] {$y$};
    \draw (0,1) node[left] {$t$} -- (3,4) node[above] {$y = x + t$};
    \draw (1,0) node[below] {$t$} -- (4,3) node[right] {$y = x - t$};
\end{tikzpicture}
$$

\begin{mdframed}
    
$$
P(A) = \frac{\mu(A)}{\mu(\Omega)} = 
\frac{1^2 - 2 \cdot \frac{1}{2} \cdot t^2}{1^2} = 1 - t^2
$$
\end{mdframed}

\end{document}

