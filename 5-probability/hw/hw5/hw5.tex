% !TEX TS-program = xelatex
% !TEX encoding = UTF-8

\documentclass[11pt, a4paper]{article} % use larger type; default would be 10pt

\usepackage{fontspec} % Font selection for XeLaTeX; see fontspec.pdf for documentation
\defaultfontfeatures{Mapping=tex-text} % to support TeX conventions like ``---''
\usepackage{xunicode} % Unicode support for LaTeX character names (accents, European chars, etc)
\usepackage{xltxtra} % Extra customizations for XeLaTeX
\usepackage{tikz}
\usetikzlibrary{arrows,calc,patterns}

\setmainfont[Ligatures=TeX]{Garamond} % set the main body font (\textrm), assumes Charis SIL is installed
%\setsansfont{Deja Vu Sans}
\setmonofont[Ligatures=TeX]{Fira Code}

% other LaTeX packages.....
\usepackage{fullpage}
\usepackage[top=2cm, bottom=4.5cm, left=2.5cm, right=2.5cm]{geometry}
\usepackage{amsmath,amsthm,amsfonts,amssymb,amscd,systeme}
\usepackage{cancel}
\geometry{a4paper} 
%\usepackage[parfill]{parskip} % Activate to begin paragraphs with an empty line rather than an indent
\usepackage{fancyhdr}
\usepackage{listings}
\usepackage{graphicx}
\usepackage{hyperref}
\usepackage{multicol}

\renewcommand\lstlistingname{Algorithm}
\renewcommand\lstlistlistingname{Algorithms}
\def\lstlistingautorefname{Alg.}
\lstdefinestyle{mystyle}{
    % backgroundcolor=\color{backcolour},   
    % commentstyle=\color{codegreen},
    % keywordstyle=\color{magenta},
    % numberstyle=\tiny\color{codegray},
    % stringstyle=\color{codepurple},
    basicstyle=\ttfamily\footnotesize,
    breakatwhitespace=false,         
    breaklines=true,                 
    captionpos=b,                    
    keepspaces=true,                 
    numbers=left,                    
    numbersep=5pt,                  
    showspaces=false,                
    showstringspaces=false,
    showtabs=false,                  
    tabsize=2
}
\lstset{style=mystyle}

\newcommand\course{5 - Теорія ймовірності}
\newcommand\hwnumber{ДЗ №5}                   % <-- homework number
\newcommand\idgroup{ФІ-91}                
\newcommand\idname{Михайло Корешков}  

\usepackage[framemethod=TikZ]{mdframed}
\mdfsetup{%
	backgroundcolor = black!5,
}
\mdfdefinestyle{ans}{%
    backgroundcolor = green!5,
    linecolor = green!50,
    linewidth = 1pt,
}

\pagestyle{fancyplain}
\headheight 35pt
\lhead{\idgroup \\ \idname}
\chead{\textbf{\Large \hwnumber}}
\rhead{\course \\ \today}
\lfoot{}
\cfoot{}
\rfoot{\small\thepage}
\headsep 1.5em

\linespread{1.2}

\begin{document}

% 15, 16, 18, 23

\section*{№ 5.15}
$$\Omega = \{0 ... 51\}$$
$$M_1 = \{0..12\} \quad ... \quad M_4 = \{38..51\}$$
$M_1$ - червова масть - червона, 
$M_2$ - друга масть - червона, 
$M_3$ - третя масть - чорна, 
$M_4$ - четверта масть - чорна.

Має місце така відповідність:
$$
\begin{array}{c|cccccccc}
\text{card} & 2 & 3 & ... &10 & \text{В} & \text{Д} & \text{К} & \text{Т} \\
n \text{ mod } 13 & 0 & 1 & ... & 8 & 9 & 10 & 11 & 12
\end{array}
$$

\subsection*{a)}
$$A = \{\text{червона карта}\} = M_1 \cup M_2$$
$$P(M_1 | A) = \frac{|M_1 \cap A|}{|A|} \cdot \frac{|\Omega|}{|\Omega|} 
= \frac{13}{26} = \frac{1}{2}$$

\subsection*{b)}

$$B = \{\text{порядок > 10}\} = \{n\;|\; 9 \le n \text{ mod } 13 \le 12\}$$
$$P(B|M_1) = \frac{|B \cap M_1|}{|M_1|} = \frac{4}{13}$$

\subsection*{c)}

$$C = \{\text{карта є Тузом}\} = \{n\;|\; n \text{ mod } = 12\}$$
$$P(C|A) = \frac{|C \cap A|}{|A|} = \frac{2}{26} = \frac{1}{13}$$

% -----------------------------------
\section*{№ 5.16}
$$\Omega = \{(a,b) \;|\; a,b \in \overline{1,6}\}$$
$$H = \{(a,b)\in\Omega \;|\; a+b > 7\} = $$ 
$$= \{(2,6), (3,5), (3,6), (4,4), (4,5), (4,6), (5,3), (5,4), (5,5), (5,6), (6,2), (6,3), (6,4), (6,5) ,(6,6)\}$$

$$|H| = \sum_{i=2}^6 (i-1) = 15$$

\subsection*{a)}
$$A = \{(a,b) \;|\; a = 1\}$$
$$P(H|A) = \frac{|H \cap A|}{|A|} = 0$$

\subsection*{b)}
$$B = \{(a,b) \;|\; a < 5\}$$
$$|B| = 4 \cdot 6 = 24$$
$$H \cap B = \{(2,6), (3,5), (3,6), (4,4), (4,5), (4,6)\}$$
$$P(H|B) = \frac{|H\cap B|}{|B|} = \frac{6}{4 \cdot 6} = \frac{1}{4}$$


% -----------------------------------
\section*{№ 5.17}
$$\Omega = \{HH, HT, TH, TT\}$$
$H$ - head (орел), $T$ - tail (решка)

$$A = \{TT, TH\};\quad B = \{TT, HT\};\quad C = \{TT, HH\}$$

Попарна незалежність:
$$P(A) = P(B) = P(C) = \frac{2}{4} = \frac{1}{2}$$
$$P(AB) = \frac{1}{4};\quad P(A)P(B) = \frac{1}{2} \cdot \frac{1}{2} = \frac{1}{4}$$
$$P(AC) = \frac{1}{4};\quad P(A)P(C) = \frac{1}{2} \cdot \frac{1}{2} = \frac{1}{4}$$
$$P(BC) = \frac{1}{4};\quad P(B)P(C) = \frac{1}{2} \cdot \frac{1}{2} = \frac{1}{4}$$
\textbf{Доведено} попарну незалежність \qedsymbol

Незалежність в сукупності:
$$P(ABC) = \frac{1}{4} \ne P(A)P(B)P(C) = \frac{1}{8}$$
\textbf{Спростовано} незалежність в сукупності

% -----------------------------------
\section*{№ 5.18}
$$\Omega = \{(w,b) \;|\; w+b = 3 \wedge w\le 7 \wedge b \le 3\}$$
$$P(w,b) = \frac{C_7^w \cdot C_3^b}{C_{10}^3}$$
$$A = \{(w,b) \in \Omega \;| b\ge 1 \} = \{(2,1), (1,2), (0,3)\}$$
$$P(A) = P(2,1) + P(1,2) + P(0,3) = \frac{C_7^2 \cdot C_3^1 + C_7^1 C_3^2+C_3^3}{C_{10}^3}
= \frac{21*3 + 21 + 1}{C_{10}^3} = \frac{85}{C_{10}^3}$$
$$B = \{(2,1), (3,0)\};\quad A\cap B = \{(2,1)\}$$
$$P(A\cap B) = \frac{C_7^2 \cdot C_3^1}{C_{10}^3} = \frac{21*3}{C_{10}^3}$$

$$P(B|A) = \frac{21*3}{85} = \frac{63}{85}$$

% -----------------------------------
\section*{№ 5.19}
$$\Omega = [0;1]^2$$
$$B = \{\xi_1 + \xi_2 \le 1\};\quad \mu(B) = \frac{1}{2}$$

\subsection*{a) $A = \{|\xi_1 - \xi_2| < 1\} = \Omega \setminus \{(1,0), (0,1)\}$}
$$\begin{tikzpicture}
    \fill[blue!25] (0,0) -- (0,2) -- (2,0) -- cycle;
    \draw[thin] (0,0) grid[step=1cm] (2,2);
    \draw[thick,->] (0,0) -- (0,2) node[above] {$\xi_2$};
    \draw[thick,->] (0,0) -- (2,0) node[right] {$\xi_1$};
    \node[below] at (2,0) {$1$};
    \node[left] at (0,2) {$1$};
    \draw[blue] (0,2) -- (1,1) node[above right] {$\xi_1 + \xi_2 \le 1$} -- (2,0);
    % ---
    \draw (1.8,-0.2) -- (2.2, 0.2);
    \draw (0.2,2.2) -- (-0.2, 1.8);
    \filldraw[fill=white] (0,2) circle (2pt);
    \filldraw[fill=white] (2,0) circle (2pt);
\end{tikzpicture}$$
$$P(A|B) = \frac{0.5}{0.5} = 1$$

\subsection*{b) $A = \{\xi_1 \cdot \xi_2 < \frac{1}{2}\} = \{\xi_2 < \frac{1}{2\xi_1}\}$}
$$\begin{tikzpicture}
    \filldraw[fill=green!25, draw=black] 
				(1,2) -- plot[%
					scale=2,
					samples=100,
					domain=0.5:1,
				] (\x,{0.5/(\x)}) -- (2,1) -- (2,0) -- (0,0) -- (0,2) -- cycle;
    % \fill[blue!25] (0,0) -- (0,2) -- (2,0) -- cycle;
    \draw[thin] (0,0) grid[step=1cm] (2,2);
    \draw[thick,->] (0,0) -- (0,2) node[above] {$\xi_2$};
    \draw[thick,->] (0,0) -- (2,0) node[right] {$\xi_1$};
    \node[below] at (2,0) {$1$};
    \node[left] at (0,2) {$1$};
    \draw[blue, thick, dashed] (0,2) -- (1,1) -- (2,0);
    % ---
    \fill (2,1) circle (2pt) node[right] {$\frac{1}{2}$};
    \fill (1,2) circle (2pt) node[above] {$\frac{1}{2}$};
\end{tikzpicture}$$
$$A\cap B = B;\quad \mu(A\cap B) = \frac{1}{2}$$
$$P(A|B) = \frac{0.5}{0.5} = 1$$

\subsection*{c) $A = \{max\{\xi_1, \xi_2|\} < 1/2\}$}
$$\begin{tikzpicture}
    \fill[green!25] (0,0) -- (0,1) node[below] {$\frac{1}{2}$} -- (1,1) -- (1,0) -- cycle;
    \draw[thin] (0,0) grid[step=1cm] (2,2);
    \draw[thick,->] (0,0) -- (0,2) node[above] {$\xi_2$};
    \draw[thick,->] (0,0) -- (2,0) node[right] {$\xi_1$};
    \node[below] at (2,0) {$1$};
    \node[left] at (0,2) {$1$};
    \draw[blue, thick, dashed] (0,2) -- (1,1) -- (2,0);
    % ---
\end{tikzpicture}$$
$$P(A \cap B) = \frac{1}{4}$$
$$P(A|B) = \frac{0.25}{0.5} = \frac{1}{2}$$

\subsection*{d) $A = \{\xi_1^2 + \xi_2^2 < 1/4\}$}
$$\begin{tikzpicture}
    \fill[green!25] (1,0) arc (0:90:1) -- (0,0) -- cycle;
    \draw[thin] (0,0) grid[step=1cm] (2,2);
    \draw[thick,->] (0,0) -- (0,2) node[above] {$\xi_2$};
    \draw[thick,->] (0,0) -- (2,0) node[right] {$\xi_1$};
    \node[below] at (2,0) {$1$};
    \node[left] at (0,2) {$1$};
    \draw[blue, thick, dashed] (0,2) -- (1,1) -- (2,0);
    % ---
\end{tikzpicture}$$
$$P(A \cap B) = P(A) = \frac{\pi \frac{1}{2}^2}{4} = \frac{\pi}{16}$$
$$P(A|B) = \frac{\pi}{16}$$


\section*{5.20}
$$\Omega = \{O, P\}^n$$
$$P(\{\omega_i = P\}) = p$$
$$A = \{\omega \in \Omega \;|\; \omega_1 = P\}$$
$$B_k = \{\omega \in \Omega \;|\; \text{рівно k Решок}\}$$
$$P(\omega_1 ... \omega_n) = p^{\#P(\omega)}(1-p)^{\#O(\omega)}$$

Для яких пар $(n,k)$ події $A$ та $B_k$ - незалежні?
$$P(A) = \sum_{\omega \in A}P(\omega) = p$$
$$P(B_k) = C_n^k p^k(1-p)^{n-k} = \frac{n! p^k (1-p)^{n-k}}{k!(n-k)!}
= p^k(1-p)^{n-k} \frac{n!}{k!(n-k)!}$$
$$P(A\cap B_k) = C_n^{k-1}p^k(1-p)^{n-k} = p^k(1-p)^{n-k} \frac{n!}{(k-1)!(n-k+1)!}$$


Незалежність - $P(A)P(B_k) = P(A\cap B_k)$
$$P(A)P(B_k) = p \cdot (p^{k}-p^{n}) \frac{n!}{k!(n-k)!} =  p^k(1-p)^{n-k} \frac{n!}{(k-1)!(n-k+1)!}$$
$$p\frac{n!}{k!(n-k)!} = \frac{n!}{(k-1)!(n-k+1)!}$$
$$p (k-1)!(n-k+1)! = k!(n-k)!$$
$$p \cdot (n-k+1) = k$$
\begin{mdframed}[style=ans]
    $$n = \frac{k}{p}+k-1,\quad \text{При } k\vdots p$$
\end{mdframed}

\section*{№ 5.21}
$$\Omega = \{\pi: {1..10} \to {1..10} \text{ - перестановка}\} \ni \omega$$
$$P(\omega) = \frac{1}{|\Omega|} = \frac{1}{10!}$$

\begin{align*}
    A &= \{\omega(4) = 4\} = \{(x,x,x,4,x,x,x,x,x,x)\}\\ \\
    B &= \{\omega(3) = \omega(2)+2 = \omega(1)+4\} = \\
    &= \{(x,1,x,2,x,3,x,x,x,x), (x,x,x,1,x,2,x,3,x,x), (x,x,x,x,x,1,x,2,x,3)\}
\end{align*}
$$P(B) = 3\cdot \frac{7!}{10!}$$

$$A\cap B = \{(x,x,x,4,x,1,x,2,x,3)\};\quad 
P(A\cap B) = \frac{6!}{10!}$$

\begin{mdframed}[style=ans]
\begin{equation*}
    P(A|B) = \frac{P(A\cap B)}{P(B)} = \frac{6!10!}{10!\cdot 3\cdot 7!} = \frac{6!}{3\cdot 7!} = \frac{1}{21}
\end{equation*}
\end{mdframed}

\section*{№ 5.22}
$$\Omega = \{0,1,...,7\} \ni \omega$$
$$P(\omega) = C_7^\omega (0.1)^\omega 0.9^(7-\omega)$$

\subsection*{a) $A=\{2\}$}
$P(A) = P(2) = C_7^2 (0.1)^2 0.9^5 = 21 \cdot \frac{9^5}{10^7} \approx 0.124$

\subsection*{b) $B=\{0\}$}
$P(B) = P(0) = C_7^0 (0.1)^0 0.9^7 = 1 \cdot \frac{9^7}{10^7} \approx 0.478$

\subsection*{c) $C=\{0,1,2\}$}
$P(C) = P(0)+P(1)+P(2) = \frac{1\cdot 9^7 + 7\cdot 9^6 + 21\cdot 9^5}{10^7} \approx 0.974$

\section*{№ 5.23}
% $$\Omega = \{(i,j)\;|\; i \text{ - де шукали}, j \text{ - де шукали}\}$$
$A_j$ - монета в ящику $j$; $B_i$ - монету \textbf{не} знайшли в ящику $i$.

$$P(A_j) = p_j$$
$$P(B_i|A_j) = 
\begin{cases}
    (1-a_i), & i=j\\
    1, & i \ne j
\end{cases}$$

$$P(B_i|A_j) P(A_j) = P(A_j|B_i) P(B_i)$$
$$P(A_j|B_i) = \frac{P(B_i|A_j) P(A_j)}{P(B_i)}$$

$$P(B_i) = P(B_i|A_i)P(A_i) + P(B_i|\overline{A_i})P(\overline{A_i})$$
$$P(B_i) = (1-a_i)p_i + 1\cdot (1-p_i) = 1 - p_i a_i$$

$$P(A_j|B_i) = \frac{P(B_i|A_j) p_j}{1 - p_i a_i} = \begin{cases}
    \displaystyle\frac{(1-a_i)p_{j(=i)}}{1 - p_i a_i}, & i=j\\
    \\
    \displaystyle\frac{1\cdot p_{j}}{1 - p_i a_i}, & i\ne j
\end{cases}$$

\section*{№ 5.24}
$$P(A|B) = P(B|A)$$
$$P(A\cup B) = 1$$
$$P(A \cap B) > 0$$

$$P(A|B) = \frac{P(A\cap B)}{P(B)} = \frac{P(A \cap B)}{P(A)} = P(B|A)$$
$$\implies P(A) = P(B) = p$$

$$P(A\cup B) = P(A) + P(B) - P(A\cap B) = 2p - P(A\cap B) = 1$$
$$2p = 1 + P(A\cap B) > 1 \implies p > \frac{1}{2}$$

\section*{№ 5.25}
$$P(AB) = P(A)P(B),\quad A \subset B$$

$$A \subset B \implies AB = A$$
$$P(AB) = P(A) = P(A)P(B) \;|\; :P(A)$$

Отже,
$$\left[\begin{array}{cc}
P(B) = 1 &\\
P(A) = 0 &
\end{array}
\right.$$

\section*{№ 5.26}
\begin{align*}
    & P(A_1 | C) P(A_2 | A_1 C) P(A_3 | A_1 A_2 C) ... P(A_n | A_1 A_2 ... A_{n-1} C) = \\ 
    &= \frac{\cancel{P(A_1 C)}}{P(C)} \frac{\cancel{P(A_2 A_1 C)}}{\cancel{P(A_1 C)}} 
    \frac{\cancel{P(A_3 A_2 A_1 C)}}{\cancel{P(A_2 A_1 C)}} ... \frac{P(A_n ... A_1 C)}{\cancel{P(A_{n-1} ... A_n C)}} 
    = \frac{P(A_n ... A_1 C)}{P(C)} = \\
    &= P(A_1 ... A_n | C) \quad \qedsymbol
\end{align*}

\section*{№ 5.27}
$$\left[\begin{array}{c}
    P(A) = 0 \\
    P(A) = 1
\end{array} \right.$$

$$\forall B: \; P(A)=0 \implies$$
$$\left(\begin{cases}
    P(A\cap B) \le P(A) \\
    P(A\cap B) \le P(B)
\end{cases}\right) \implies P(AB) \le 0 \implies P(AB) = 0$$
$$P(AB)=0$$
$$P(A)P(B) = 0\cdot P(B) = 0$$
$$\implies P(AB) = P(B) \; \qedsymbol$$

$$\forall B: \; P(A)=1 \implies$$
$$P(A\cup B) \ge P(A) = 1 $$
$$P(A\cup B) = 1$$
$$P(A\cup B) = P(A) + P(B) - P(A\cap B)$$
$$1 = 1 + P(B) - P(AB)$$
$$P(AB) = P(B)$$
$$P(A)P(B) = P(B)$$
$$\implies P(AB) = P(A)P(B) \; \qedsymbol$$


\end{document}

