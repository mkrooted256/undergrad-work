% !TEX TS-program = xelatex
% !TEX encoding = UTF-8

\documentclass[11pt, a4paper]{article} % use larger type; default would be 10pt

\usepackage{fontspec} % Font selection for XeLaTeX; see fontspec.pdf for documentation
\defaultfontfeatures{Mapping=tex-text} % to support TeX conventions like ``---''
\usepackage{xunicode} % Unicode support for LaTeX character names (accents, European chars, etc)
\usepackage{xltxtra} % Extra customizations for XeLaTeX
\usepackage{tikz}
\usetikzlibrary{arrows,calc,patterns}

\setmainfont[Ligatures=TeX]{Garamond} % set the main body font (\textrm), assumes Charis SIL is installed
%\setsansfont{Deja Vu Sans}
\setmonofont[Ligatures=TeX]{Fira Code}

% other LaTeX packages.....
\usepackage{fullpage}
\usepackage[top=2cm, bottom=4.5cm, left=2.5cm, right=2.5cm]{geometry}
\usepackage{amsmath,amsthm,amsfonts,amssymb,amscd,systeme}
\geometry{a4paper} 
%\usepackage[parfill]{parskip} % Activate to begin paragraphs with an empty line rather than an indent
\usepackage{fancyhdr}
\usepackage{listings}
\usepackage{graphicx}
\usepackage{hyperref}
\usepackage{multicol}

\renewcommand\lstlistingname{Algorithm}
\renewcommand\lstlistlistingname{Algorithms}
\def\lstlistingautorefname{Alg.}
\lstdefinestyle{mystyle}{
    % backgroundcolor=\color{backcolour},   
    % commentstyle=\color{codegreen},
    % keywordstyle=\color{magenta},
    % numberstyle=\tiny\color{codegray},
    % stringstyle=\color{codepurple},
    basicstyle=\ttfamily\footnotesize,
    breakatwhitespace=false,         
    breaklines=true,                 
    captionpos=b,                    
    keepspaces=true,                 
    numbers=left,                    
    numbersep=5pt,                  
    showspaces=false,                
    showstringspaces=false,
    showtabs=false,                  
    tabsize=2
}
\lstset{style=mystyle}

\newcommand\course{5 - Теорія ймовірності}
\newcommand\hwnumber{ДЗ №3}                   % <-- homework number
\newcommand\idgroup{ФІ-91}                
\newcommand\idname{Михайло Корешков}  

\usepackage[framemethod=TikZ]{mdframed}
\mdfsetup{%
	backgroundcolor = black!5,
}
\mdfdefinestyle{ans}{%
    backgroundcolor = green!5,
    linecolor = green!50,
    linewidth = 1pt,
}

\pagestyle{fancyplain}
\headheight 35pt
\lhead{\idgroup \\ \idname}
\chead{\textbf{\Large \hwnumber}}
\rhead{\course \\ \today}
\lfoot{}
\cfoot{}
\rfoot{\small\thepage}
\headsep 1.5em

\linespread{1.2}

\begin{document}

\section*{Незалежність подій}
\begin{itemize}
    \item Події $A,B \subseteq \Omega$ називаються \textbf{незалежними} якщо
    $$P(A\cap B) = P(A)\cdot P(B)$$
    
    \item Події $\{A_i\}_{i\in I}$ називаються \textbf{попарно незалежними} якщо 
    кожна пара подій є незалежною
    
    \item Події $\{A_i\}_{i\in I}$ називаються \textbf{незалежними в сукупності} якщо
    $$\forall J\subseteq I:\; P(\bigcap_{i\in J}A_i) = \prod_{i\in J}P(A_i)$$
\end{itemize}

\begin{mdframed}
    Попарна незалежність $\nRightarrow$ Незалежність в сукупності
\end{mdframed}

\section*{Несумісні події}
Події $A, B$ називаються несумісними якщо $A\cap B = \varnothing$. \\
$A,B \text{ - несумісні} \implies P(A\cap B) = 0$, але не навпаки. 

\pagebreak
\section*{Умовна ймовірність}
Умовна ймовірність події $A$ за умови події $B$ позначається $P(A|B)$.
\begin{mdframed}
    $$P(A|B) = \frac{P(A\cap B)}{P(B)} \text{, при } P(B) > 0$$
\end{mdframed}

Тобто умовна ймовірність показує, наскільки ймовірно, що відбудеться $A$, якщо відомо, \\
що відбулась $B$.

\section*{Повна група гіпотез}
Множина $\{H_i\}_{i=1}^n, \; H_i \subseteq \Omega$ називається 
\textbf{Повною групою гіпотез}, якщо:
\begin{enumerate}
    \item $H_i \cap H_j = \varnothing \quad (i\ne j)$
    \item $\displaystyle\bigcup_{i=i}^n H_i = \Omega$
    \item $P(H_i) > 0$
\end{enumerate}

Тобто, \textbf{Повна група гіпотез} - це розбиття $\Omega$, елементи якого мають ненульову ймовірність.

\section*{Формула повної ймовірності}
\begin{mdframed}
Нехай $\{H_1, ..., H_n\}$ - повна група гіпотез, $A$ - подія. Тоді
$$P(A) = \sum_{i=1}^n P(A|H_i)P(H_i)$$
\end{mdframed}

\subsection*{Застосування формули повної ймовірності}
Використовується коли ми знаємо ймовірність події за різних несумісних гіпотез
та хочемо обчислити її повну (безумовну) ймовірність 

\section*{Формула Байеса}
Нехай $\{H_1, ..., H_n\}$ - повна група гіпотез, $A \subseteq \bigcup_{i=1}^n H_i$ - подія.
\begin{align*}
    P(H_k|A) &= \frac{P(A\cap H_k)}{P(A)} \cdot \frac{P(H_k)}{P(H_k)}
    = \frac{P(H_k\cap A)}{P(H_k)} \cdot \frac{P(H_k)}{P(A)} =\\
    &= \frac{P(A|H_k)P(H_k)}{P(A)}
\end{align*}

\pagebreak
Використаємо формулу повної ймовірності ($P(A) = \sum_i P(A|H_i)P(H_i)$)
і отримаємо формулу Байеса:
\begin{mdframed}
$$    P(H_k|A) = \frac{P(A|H_k)P(H_k)}{\sum_{i=1}^n P(A|H_i)P(H_i)}$$
\end{mdframed}

\subsection*{Застосування формули Байеса}
Використовується коли ми знаємо ймовірності деякої події за умов різних несумісних гіпотез,
а хочемо обчислити ймовірність гіпотези за умови, що така подія відбулась.



\end{document}

