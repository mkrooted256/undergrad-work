% !TEX TS-program = xelatex
% !TEX encoding = UTF-8

\documentclass[11pt, a4paper]{article} % use larger type; default would be 10pt

\usepackage{fontspec} % Font selection for XeLaTeX; see fontspec.pdf for documentation
\defaultfontfeatures{Mapping=tex-text} % to support TeX conventions like ``---''
\usepackage{xunicode} % Unicode support for LaTeX character names (accents, European chars, etc)
\usepackage{xltxtra} % Extra customizations for XeLaTeX
\usepackage{tikz}
\usetikzlibrary{arrows,calc,patterns}

\setmainfont[Ligatures=TeX]{Garamond} % set the main body font (\textrm), assumes Charis SIL is installed
%\setsansfont{Deja Vu Sans}
\setmonofont[Ligatures=TeX]{Fira Code}

% other LaTeX packages.....
\usepackage{fullpage}
\usepackage[top=2cm, bottom=4.5cm, left=2.5cm, right=2.5cm]{geometry}
\usepackage{amsmath,amsthm,amsfonts,amssymb,amscd,systeme}
\usepackage{cancel}
\geometry{a4paper} 
%\usepackage[parfill]{parskip} % Activate to begin paragraphs with an empty line rather than an indent
\usepackage{fancyhdr}
\usepackage{listings}
\usepackage{graphicx}
\usepackage{hyperref}
\usepackage{multicol}

\renewcommand\lstlistingname{Algorithm}
\renewcommand\lstlistlistingname{Algorithms}
\def\lstlistingautorefname{Alg.}
\lstdefinestyle{mystyle}{
    % backgroundcolor=\color{backcolour},   
    % commentstyle=\color{codegreen},
    % keywordstyle=\color{magenta},
    % numberstyle=\tiny\color{codegray},
    % stringstyle=\color{codepurple},
    basicstyle=\ttfamily\footnotesize,
    breakatwhitespace=false,         
    breaklines=true,                 
    captionpos=b,                    
    keepspaces=true,                 
    numbers=left,                    
    numbersep=5pt,                  
    showspaces=false,                
    showstringspaces=false,
    showtabs=false,                  
    tabsize=2
}
\lstset{style=mystyle}

\newcommand\course{5 - Теорія ймовірності}
\newcommand\hwnumber{ДЗ №9}                   % <-- homework number
\newcommand\idgroup{ФІ-91}                
\newcommand\idname{Михайло Корешков}  

\usepackage[framemethod=TikZ]{mdframed}
\mdfsetup{%
	backgroundcolor = black!5,
}
\mdfdefinestyle{ans}{%
    backgroundcolor = green!5,
    linecolor = green!50,
    linewidth = 1pt,
}

\pagestyle{fancyplain}
\headheight 35pt
\lhead{\idgroup \\ \idname}
\chead{\textbf{\Large \hwnumber}}
\rhead{\course \\ \today}
\lfoot{}
\cfoot{}
\rfoot{\small\thepage}
\headsep 1.5em

\linespread{1.2}

\begin{document}

%  8.7, 8.16,8.17,8.18

\section*{№ 8.7}

\begin{mdframed}
    Let $\{\xi_n\;|\;n>0\}$ - послідовність випадкових величин, заданих
    на $(\Omega, \mathcal{F}, P)$\\
    Довести
    $$\inf_n \xi_n, \sup_n \xi_n \;- \text{ випадкові величини}$$
\end{mdframed}

Щоб довести, що $\displaystyle \inf_n \xi_n$ - в.в., достатньо довести, що 
$$
\forall a\in\mathbb R:\; \left(\inf \xi_n\right)^{-1} \bigl((-\infty;a)\bigr) \in \mathcal F \iff
$$
$$
\forall a\in\mathbb R:\; \{\omega : \left(\inf \xi_n\right)(\omega) < a\} \in \mathcal{F}
$$

\begin{mdframed}
    Зауваження
    $$\left(\inf_n \xi_n\right)(\omega) \equiv \inf_n \bigl(\xi_n(\omega)\bigr)$$
\end{mdframed}


$$
\begin{gathered}
    \omega \in \{ \omega : \left(\inf \xi_n\right)(\omega) < a \} \iff \\
    \exists n : \xi_n(\omega) < a \iff \\
    \exists n: \omega \in \{\omega : \xi_n(\omega) < a \} \iff \\
    \omega \in \bigcup_n \{\omega : \xi_n(\omega) < a \} 
\end{gathered}
$$
That is,
$$\{\inf_n \xi_n < a\} = \bigcup_n \{\omega : \xi_n(\omega) < a \}$$
$$
\{\omega : \xi_n(\omega) < a\} \in \mathcal F \text{ за критерієм випадкової величини} 
$$
$$
\{\inf_n \xi_n < a\} \in \mathcal F \text{ як зліченне об'єднання вимірних множин}
$$

\begin{mdframed}[style=ans]
    $\displaystyle \inf_n \xi_n$ - вимірна
\end{mdframed}

Для $\displaystyle \sup_n \xi_n$:
$$
\begin{gathered}
    \{\omega : \left(\sup_n \xi_n\right)(\omega) > a\} = \\
    = \{\omega : \exists n:\; \xi_n(\omega) > a\} = \\
    = \bigcup_n \{\omega : \xi_n(\omega) > a\} \in \mathcal{F} \text{ бо $\xi$ - в.в.}
\end{gathered}
$$
\begin{mdframed}[style=ans]
    $\displaystyle \sup_n \xi_n$ - вимірна
\end{mdframed}
\pagebreak

\section*{№ 8.16}
\begin{mdframed}
    Let $(\Omega, \mathcal F, P)$ - ймовірнісний простір\\
    Чи зобов'язана бути $f$ вимірною, якщо вимірною є:
    \begin{itemize}
        \item \textbf{a)} $\cos f$ 
        \item \textbf{b)} $f^f$ ($f>0$)
    \end{itemize}
\end{mdframed}

\subsection*{a)}

Let 
$$
\Omega = \{w1, w2, w3\}$$
$$
\xi = \begin{cases}
    0,& w1\\
    \pi,& w2\\
    2\pi,& w3    
\end{cases}
$$
$$
\mathcal F = \{\Omega, \varnothing, \{w1,w3\}, \{w2\}\}
$$

$\xi$ - не вимірна, бо $\xi^{-1} (0) = \{w1\} \notin \mathcal F$

Let $f = \cos \circ \xi$. 

$$
\begin{gathered}
    f^{-1}\bigl((-\infty; -1)\bigr) = \varnothing \in \mathcal F\\
    f^{-1}\bigl((-\infty; 1)\bigr) = \{f^{-1}(-1)\} = \{\xi^{-1}(\pi)\} = \{w2\} \in \mathcal F\\
    f^{-1}\bigl((-\infty; +\infty)\bigr) = \{f^{-1}(-1)\} \cup f^(-1)(1) = \{ w1, w2, w3 \} = \Omega \in \mathcal F\\
\end{gathered}
$$

Тобто $\forall a \in \mathbb R : f^{-1}\bigl((-\infty;a)\bigr)$. А отже $f$ - вимірна.

\begin{mdframed}[style=ans]
    Отже, $\xi$ не має бути вимірною щоб $\cos \xi$ була вимірною
\end{mdframed}

\subsection*{b)}

Функція $x^x$ не є ін'єктивною для значень $x\in(0,1)$.
Візьмемо два значення $x$, для яких $x_1^{x_1} = x_2^{x_2}$

$$
\begin{tikzpicture}
    \draw[thick] (0,0) -- (3,0) node[right] {$x$};
    \draw[thick] (0,0) -- (0,5) node[right] {$x^x$};
    \draw[thin] (0,1) node[above left] {$1$} -- (3,1);
    \draw[thin, blue] (0,0.8) node[left] {$0.8$} -- (3,0.8);
    \fill[blue] (0.0946497,0.8) circle (0.06cm);
    \fill[blue] (0.739534,0.8) circle (0.06cm);
    \draw[thin] (0.0946497,0) node[below] {$x_1$} -- (0.0946497,0.8);
    \draw[thin] (0.739534,0) node[below] {$x_2$} -- (0.739534,0.8);
    \draw[thin] (1,0) node[below right] {$1$} -- (1,1);
    \fill[black] (1,1) circle (0.06cm);
    \fill[black] (1,0) circle (0.06cm);

    \draw plot[%
        scale=1,
        samples=100,
        domain=0.01:2.05,
    ] (\x,{(\x)^(\x)});
\end{tikzpicture}
$$



Let $f(x) = x^x$
$$
\begin{gathered}
    x_1 \approx 0.0946497; \\
    x_2 \approx 0.739534; \\
    x_1^{x_1} = x_2^{x_2} = 0.8
\end{gathered}
$$


Let 
$$
\Omega = \{w1, w2, w3\}$$
$$
\xi = \begin{cases}
    x_1,& w1\\
    x_2,& w2\\
    1,& w3    
\end{cases}
$$
$$
\mathcal F = \{\Omega, \varnothing, \{w1,w2\}, \{w3\}\}
$$

$\xi$ - не вимірна, бо $\xi^{-1} (x_1) = \{w1\} \notin \mathcal F$

Let $g = f \circ \xi$. 

$$
\begin{gathered}
    g^{-1}\bigl((-\infty; 0.8)\bigr) = \varnothing \in \mathcal F\\
    g^{-1}\bigl((-\infty; 1)\bigr) = f^{-1}(0.8) = \{\xi^{-1}(x_1), \xi^{-1}(x_2)\} = \{w1,w2\} \in \mathcal F\\
    g^{-1}\bigl((-\infty; +\infty)\bigr) = f^{-1}(0.8) \cup f^(-1)(1) = \{ w1, w2, w3 \} = \Omega \in \mathcal F\\
\end{gathered}
$$

$g = \xi^\xi$ - вимірна, бо $\forall a\in\mathbb R : g^{-1}\bigl((-\infty;a)\bigr) \in \mathcal F$

\begin{mdframed}[style=ans]
    Отже, $\xi^\xi$ може бути вімірною навіть коли $\xi$ - не вимірна
\end{mdframed}


\section*{№ 8.17}
\begin{mdframed}
    Let $(\Omega, \mathcal F, P)$ - measure space.\\
    Let $\{\xi_n : n>0\}$ - sequence of $\mathcal F$-measurable functions.\\
    Prove:
    $$\overline{\lim_n}\;\xi_n - \text{ measurable function}$$
\end{mdframed}

\begin{mdframed}
    Note:
    $$\left(\overline{\lim_n}\; \xi_n\right)(\omega) \equiv \overline{\lim_n}\bigl(\xi_n(\omega)\bigr)$$
\end{mdframed}

\begin{mdframed}
    Note:\\
    Let $\{a_n : n>0\} : a_n \in \mathbb R$. Then
    $$\overline{\lim_n}\; a_n = \inf_n \sup_{k \ge n} a_k$$
\end{mdframed}

So
$$
\begin{gathered}
    \left(\overline{\lim_n}\; \xi_n\right)(\omega) \equiv \overline{\lim_n}\bigl(\xi_n(\omega)\bigr) = \\
    = \inf_n \sup_{k\ge n} \bigl(\xi_k(\omega)\bigr)
\end{gathered}
$$

Let $\displaystyle f_n = \sup_{k\ge n} \xi_k$. 
Раніше доведено, що $\displaystyle \sup_n \xi'_n$ - вимірна. 
Можемо побудувати $\{\xi'_n\}$ так, що перші $n$ елементів дорівнюють $\xi_1$.
Додавання скінченної кількості елементів не впливає на збіжність.\\
Отож, $f_n$ - вимірна.

$$\overline{\lim_n}\;\xi_n = \inf_n f_n$$

Раніше доведено, що $g = \displaystyle \inf_n f_n$ - вимірна.

Тобто $\displaystyle \inf_n f_n = \inf_n \sup_{k \ge n} \xi_k$ - вимірна

\section*{№ 8.18}
\begin{mdframed}
    Let $(\Omega, \mathcal F, P)$ - measure space.\\
    Let $\{\xi_n : n>0\}$ - sequence of $\mathcal F$-measurable functions.\\
    Prove:
    $$A = \{\omega\in\Omega : \{\xi_n(\omega) : n>0\} - \text{монотонна}\} \in \mathcal F$$
\end{mdframed}

Let $\omega \in A$.
$$\{\xi_n(\omega) : n>0\} - \text{монотонна}\}$$

Нехай монотонно зростаюча. 
Якщо спадна, то множимо на $-1$ - отримуємо монотонно зростаючу.

$$\forall n>0: \xi_{n+1}(\omega) \ge \xi_{n}(\omega)$$

Let $B_n = \{\omega : \forall 1 \le k < n : \xi_n(\omega) \ge \xi_k(\omega) \}$
$$B_n = \bigcap_{k=1}^{n-1} \{\omega : \xi_n(\omega) \ge \xi_k(\omega) \}$$

Для фіксованого $k$, $\{\omega : \xi_n(\omega) \ge \xi_k(\omega) \} = \{\omega : \xi_n(\omega) \ge a \} \in \mathcal F$,
бо $\xi$ - вимірна.

Тобто $B_n \in \mathcal F$ як скінченний перетин вимірних множин.

$$A = \bigcap_{n=1}^{\infty} B_n$$

Маємо, що $A \in \mathcal F$ як скінченний перетин вимірних множин.

\begin{mdframed}[style=ans]
    $$\text{Дійсно, } A \in \mathcal F$$
    
\end{mdframed}

\end{document}

