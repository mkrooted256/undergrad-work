% !TEX TS-program = xelatex
% !TEX encoding = UTF-8

\documentclass[11pt, a4paper]{article} % use larger type; default would be 10pt

\usepackage{fontspec} % Font selection for XeLaTeX; see fontspec.pdf for documentation
\defaultfontfeatures{Mapping=tex-text} % to support TeX conventions like ``---''
\usepackage{xunicode} % Unicode support for LaTeX character names (accents, European chars, etc)
\usepackage{xltxtra} % Extra customizations for XeLaTeX
\usepackage{tikz}
\usetikzlibrary{arrows,calc,patterns}

\setmainfont[Ligatures=TeX]{Garamond} % set the main body font (\textrm), assumes Charis SIL is installed
%\setsansfont{Deja Vu Sans}
\setmonofont[Ligatures=TeX]{Fira Code}

% other LaTeX packages.....
\usepackage{fullpage}
\usepackage[top=2cm, bottom=4.5cm, left=2.5cm, right=2.5cm]{geometry}
\usepackage{amsmath,amsthm,amsfonts,amssymb,amscd,systeme}
\usepackage{cancel}
\geometry{a4paper} 
%\usepackage[parfill]{parskip} % Activate to begin paragraphs with an empty line rather than an indent
\usepackage{fancyhdr}
\usepackage{listings}
\usepackage{graphicx}
\usepackage{hyperref}
\usepackage{multicol}

\renewcommand\lstlistingname{Algorithm}
\renewcommand\lstlistlistingname{Algorithms}
\def\lstlistingautorefname{Alg.}
\lstdefinestyle{mystyle}{
    % backgroundcolor=\color{backcolour},   
    % commentstyle=\color{codegreen},
    % keywordstyle=\color{magenta},
    % numberstyle=\tiny\color{codegray},
    % stringstyle=\color{codepurple},
    basicstyle=\ttfamily\footnotesize,
    breakatwhitespace=false,         
    breaklines=true,                 
    captionpos=b,                    
    keepspaces=true,                 
    numbers=left,                    
    numbersep=5pt,                  
    showspaces=false,                
    showstringspaces=false,
    showtabs=false,                  
    tabsize=2
}
\lstset{style=mystyle}

\newcommand\course{5 - Теорія ймовірності}
\newcommand\hwnumber{ДЗ №9}                   % <-- homework number
\newcommand\idgroup{ФІ-91}                
\newcommand\idname{Михайло Корешков}  

\usepackage[framemethod=TikZ]{mdframed}
\mdfsetup{%
	backgroundcolor = black!5,
}
\mdfdefinestyle{ans}{%
    backgroundcolor = green!5,
    linecolor = green!50,
    linewidth = 1pt,
}

\pagestyle{fancyplain}
\headheight 35pt
\lhead{\idgroup \\ \idname}
\chead{\textbf{\Large \hwnumber}}
\rhead{\course \\ \today}
\lfoot{}
\cfoot{}
\rfoot{\small\thepage}
\headsep 1.5em

\linespread{1.2}

\begin{document}

%  9.19-9.22

\section*{№ 9.19}
\newcommand{\nA}{\overline{A}}
\begin{mdframed}
    $$\Omega = \{\nA, A\nA, AA\nA, \cdots \}$$
    $$P(A) = 0.8;\quad A = \{\text{вистріл влучив}\}$$
    
    Let $|\omega| = $ кількість пострілів. Наприклад, $|AA\nA| = 3$.

    Let $\xi(\omega) = |\omega|$

    $$F_{\xi} - ?$$
\end{mdframed}

Let $\omega \in \Omega;$. 
$$P(\omega) = P(A)^{|\omega|-1}\cdot P(\nA)$$


$$P(\xi = k) = P(A)^{k-1}\cdot (1-P(A))$$
Одразу можна побачити зв'язок з Геометричним розподілом. 

$$\xi \sim \operatorname{Geom}(1-P(A)) + 1$$ 

\section*{№ 9.20}
\begin{mdframed}
    $$\begin{gathered}
        P(\xi = -4) = 0.3; \quad P(\xi = -3)=0.2;\\
        P(\xi=1)=p; \quad P(\xi=2) = 0.3
    \end{gathered}$$

    \begin{itemize}
        \item Обчислити $p$
        \item Записати закон розподілу в.в.
        $$\eta = \cos \bigl(\frac{\pi}{2} \cdot (\xi^2 + 2\xi -8)\bigr)$$
    \end{itemize}
\end{mdframed}

Якщо припустити, що в умові описані всі можливі значення $\xi$, то 
$$p = 1 - 0.3 - 0.2 - 0.3 = 0.2$$

$$\eta = \begin{cases}
    \cos \bigl(\frac{\pi}{2} \cdot ((-4)^2 - 2\cdot 4 - 8)\bigr) = \cos 0 = 1, & \xi = -4 \\
    \cos \bigl(\frac{\pi}{2} \cdot ((-3)^2 - 2\cdot 3 - 8)\bigr) = \cos (-\frac{5}{2}\pi) = \cos \frac{pi}{2} =  0, & \xi = -3 \\
    \cos \bigl(\frac{\pi}{2} \cdot ((1)^2 + 2\cdot 1 - 8)\bigr) = \cos (-\frac{5}{2}\pi) = 0, & \xi = 1 \\
    \cos \bigl(\frac{\pi}{2} \cdot ((2)^2 + 2\cdot 2 - 8)\bigr) = \cos 0 = 1, & \xi = 2 \\
\end{cases}$$

Маємо:
$$P(\eta = 1) = P(\xi = -4) + P(\xi = 2) = 0.6$$
$$P(\eta = 0) = P(\xi = -3) + P(\xi = 1) = 0.4$$

\section*{№ 9.21}
\begin{mdframed}
    Визначити, які з наступних функцій є функціями розподілу
\end{mdframed}

\begin{mdframed}[backgroundcolor=blue!25]
    Критерій того, що $F(x)$ - фція розподілу:
    \begin{itemize}
        \item $F(x)$ - неспадна;
        \item $F(x+o) = F(x)$ - неперервність справа;
        \item $F(-\infty) = 0;\quad F(+\infty) = 1$ - нормування;
    \end{itemize}
\end{mdframed}

\subsection*{a)}
\begin{mdframed}
    $$F(x) = \begin{cases}
        0,& x<0\\
        1-\frac{1-e^{-x}}{2},& x\ge 0
    \end{cases}$$
\end{mdframed}

$$F'(x) = \begin{cases}
    -\frac{1}{2}\cdot (-1) \cdot - e^{-x} = -\frac{1}{2}e^{-x} \le 0\\
    0, x<0
\end{cases}$$
Отже, $F$ - незростаюча; А має бути неспадна!

$$F(x+o) = \begin{cases}
    0,& x < 0\\
    1-\frac{1-e^{-(x+o)}}{2} = 1-\frac{1-e^{-x}}{2},& x\ge 0
\end{cases} = F(x)$$
Отже, $F$-неперервна справа;

$$\begin{gathered}
    F(-\infty) = 0;\\
    F(+\infty) = 1 - \frac{1 - o}{2} = \frac{1}{2} \ne 1
\end{gathered}$$

\begin{mdframed}[backgroundcolor=red!25]
    $F$ - не фція розподілу, бо $F(+\infty) = \frac{1}{2} \ne 1 $
    та бо вона спадна.
\end{mdframed}

\subsection*{б)}
\begin{mdframed}
    $$F(x) = \begin{cases}
        0,& x\le0\\
        1-\frac{1-e^{-x}}{2},& x > 0
    \end{cases}$$
\end{mdframed}

Єдина проблема може бути в $x=0$.
$$F(0+o) = 1-\frac{1-e^{-o}}{2} = 1-o = 1 \ne F(0) = 0$$

\begin{mdframed}[backgroundcolor=red!25]
    $F$ - не фція розподілу, бо $F(+\infty) = \frac{1}{2} \ne 1 $, 
    бо вона спадна та бо вона не неперервна справа в $x=0$
\end{mdframed}

\subsection*{в)}
\begin{mdframed}
    $$F(x) = e^{-e^{-x}}$$
\end{mdframed}

$$F(x+0) = F(x), \text{ бо експонента - неперервна фція}$$

$$F'(x) = e^{-x} \cdot (-e^{-x}) \cdot e^{-e^{-x}} > 0$$
Тобто, функція неспадна.

$$F(-\infty) = e^{-e^{+\infty}} = e^{-\infty} = 0$$
$$F(+\infty) = e^{-e^{-\infty}} = e^{-0} = 1$$
Тобто, функція правильно поводить себе на нескінченностях.

\begin{mdframed}[backgroundcolor=green!25]
    $F$ - функція розподілу.
\end{mdframed}

\section*{№9.22}
\begin{mdframed}
    Let $\xi$ - випадкова величина із функцією розподілу
    $$F(x) = \begin{cases} 
        0, & x<0 \\
        \displaystyle\frac{x+[x]}{10},& 0\le x\le 5\\
        1,& x>5
    \end{cases}$$
    Обчислити ймовірності:
\end{mdframed}

\begin{mdframed}
    Зауваження: вважатиму $[x] = \lfloor x \rfloor$, бо підлога неперервна справа.
\end{mdframed}

\subsection*{а)}
$$P(\xi \le 3)$$

За визначенням PDF, $F_\xi(x) = P(\xi \le x)$.
$$P(\xi \le 3) = F(3) = \frac{6}{10} = 0.6$$

\subsection*{б)}
$$P(\xi > 2) = 1-P(\xi \le 2) = 1 - F(2) = \frac{4}{10} = 0.4$$

\subsection*{в)}
$$P(1<\xi\le 3.5) = P(\xi \le 3.5) - P(\xi \le 1) = F(3.5) - F(1) = \frac{6.5}{10} - \frac{2}{10} = 0.45$$

\subsection*{г)}
$$\begin{gathered}
    P(\xi^2 - 3\xi \ge -2) = P((\xi-2)(\xi-1)\ge 0) = P(\xi \le 1 \vee \xi \ge 2) = \\
    = F(1) + (1-F(2-o)) = 0.2 + (1-\frac{2+1}{10}) = 0.2 + 0.7 = \\
    = 0.9
\end{gathered}$$

\subsection*{д)}
$$P(|\xi|=2) = 1-P(|\xi|< 2 \vee \overline{|\xi|\le 2}) = 1-\bigl( (F(2-0)-F(-2)) + 1 - (F(2)-F(2-0)) \bigr) = 1-1 = 0$$

\subsection*{е)}

$$\begin{tikzpicture}
    \draw[thick, ->] (-4,0) -- (4,0) node[right] {$\xi$};
    \draw[thick, ->] (0,-2) -- (0,2) node[above] {$\sin \xi$};
    \draw[red] plot[%
        scale=1,
        samples=100,
        domain=-4:4,
    ] (\x,{2*sin(deg(pi * \x))});
    \draw[green] plot[%
        scale=1,
        samples=50,
        domain=5 / 6 : 2 + 1/6,
    ] (\x,{2*sin(deg(pi * \x))});
    \draw (-4,1) -- (4,1) node[right] {$\sin \xi = 0.5$};
    \draw[thin] (2,-2) -- (2,2) node[above] {$2\pi k$};
    \draw[thin] ({{5/6}},-2)  node[below] {\tiny $2\pi k - \frac{7\pi}{6}$} -- ({{5/6}},2);
    \draw[thin] ({{2+1/6}},-2) node[below] {\tiny $2\pi k + \frac{\pi}{6}$}-- ({{2+1/6}},2);
    \fill ({{5/6}},-2) circle (2pt);
    \fill ({{2+1/6}},-2) circle (2pt);
    \fill (2,2) circle (2pt);
\end{tikzpicture}$$

$$p = P(\sin \xi < 0.5) = P\bigl(\exists k:\; \xi \in (2\pi k - \frac{7\pi}{6};2\pi k + \frac{\pi}{6}) \bigr)$$

$$p_k = P(\xi \in (2\pi k - \frac{7\pi}{6};2\pi k + \frac{\pi}{6})) = F(2\pi k + \frac{\pi}{6} - o) - F(2\pi k - \frac{7\pi}{6})$$
$$p = \sum_{k=-\infty}^{+\infty}p_k$$
$$p = p_{-1} + p_0 + p_1 + p_2, \text{ бо для $x$ до $0$ та після $2\pi \approx 6.24$ розглядати не має менсу}$$

$$p_{-1} = F(-2\pi + \frac{\pi}{6} - o) - F(-2\pi-\frac{7\pi}{6}) \approx F(-5.76) - F(-9.94) = 0$$
$$p_0 = F(\frac{\pi}{6} - o) - F(-\frac{7\pi}{6}) = F(\frac{\pi}{6}) \approx \frac{0.5236 + 0}{10} = 0.05236$$
$$p_1 = F(2\pi + \frac{\pi}{6} - o) - F(2\pi-\frac{7\pi}{6}) \approx F(6.8) - F(2.618) = 1-\frac{2.618+2}{10} = 1 - 0.4618 = 0.5382$$
$$p_2 = F(4\pi + \frac{\pi}{6} - o) - F(4\pi-\frac{7\pi}{6}) = 1-1 = 0$$

$$P(\sin \xi < 0.5) = p_0 + p_1 \approx 0.59 $$

\subsection*{є)}
$$P(|\xi-2.5|\le 1.5) = P(\xi \le 4 \wedge \xi \ge 1) = F(4)-F(1-o) = 0.8 - \frac{1+0}{10} = 0.7$$

\subsection*{ж)}
$$P(\{\xi\} < 0.5)$$

\begin{mdframed}
    Зауваження: вважатиму $\{x\}$ як дробова частина $x$.\\
    Зауваження: дробова частина неперервна справа.
\end{mdframed}

$$p = P(\{\xi\} < 0.5) = \sum_{k=0}^{5} P(\xi < k+0.5 \wedge \xi \ge k) 
 = \sum_{k=0}^{5} (F(k+0.5-o) - F(k-o)) = \sum_{k=0}^{5} p_k$$

$$p_{-1} = F(-0.5-o) - F(-1-o) = 0-0 = 0$$
$$p_0 = F(0.5-o) - F(-o) = 0.05 - 0 = 0.05$$
$$p_k = F(k+0.5-o) - F(k-o) = \frac{1}{10} (k+0.5 + k) - (k + (k-1)) 
= \frac{1}{10} (0.5+1) = 0.15$$
$$p_5 = F(5.5-o) - F(5-o) = 1-0.9 = 0.1$$
$$p_6 = F(6.5-o) - F(6-o) = 1-1 = 0$$


$$p = 0.05 + 0.15 \cdot 4 + 0.1 = 0.15 \cdot 5 = 0.75$$



\end{document}

