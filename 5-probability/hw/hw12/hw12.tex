% !TEX TS-program = xelatex
% !TEX encoding = UTF-8

\documentclass[11pt, a4paper]{article} % use larger type; default would be 10pt

\usepackage{fontspec} % Font selection for XeLaTeX; see fontspec.pdf for documentation
\defaultfontfeatures{Mapping=tex-text} % to support TeX conventions like ``---''
\usepackage{xunicode} % Unicode support for LaTeX character names (accents, European chars, etc)
\usepackage{xltxtra} % Extra customizations for XeLaTeX
\usepackage{tikz}
\usetikzlibrary{arrows,calc,patterns}

\setmainfont[Ligatures=TeX]{Garamond} % set the main body font (\textrm), assumes Charis SIL is installed
%\setsansfont{Deja Vu Sans}
\setmonofont[Ligatures=TeX]{Fira Code}

% other LaTeX packages.....
\usepackage{fullpage}
\usepackage[top=2cm, bottom=4.5cm, left=2.5cm, right=2.5cm]{geometry}
\usepackage{amsmath,amsthm,amsfonts,amssymb,amscd,systeme}
\usepackage{unicode-math}
\usepackage{cancel}
\geometry{a4paper} 
%\usepackage[parfill]{parskip} % Activate to begin paragraphs with an empty line rather than an indent
\usepackage{fancyhdr}
\usepackage{listings}
\usepackage{graphicx}
\usepackage{hyperref}
\usepackage{multicol}

\renewcommand\lstlistingname{Algorithm}
\renewcommand\lstlistlistingname{Algorithms}
\def\lstlistingautorefname{Alg.}
\lstdefinestyle{mystyle}{
    % backgroundcolor=\color{backcolour},   
    % commentstyle=\color{codegreen},
    % keywordstyle=\color{magenta},
    % numberstyle=\tiny\color{codegray},
    % stringstyle=\color{codepurple},
    basicstyle=\ttfamily\footnotesize,
    breakatwhitespace=false,         
    breaklines=true,                 
    captionpos=b,                    
    keepspaces=true,                 
    numbers=left,                    
    numbersep=5pt,                  
    showspaces=false,                
    showstringspaces=false,
    showtabs=false,                  
    tabsize=2
}
\lstset{style=mystyle}

\newcommand\course{5 - Теорія ймовірності}
\newcommand\hwnumber{ДЗ №12}                   % <-- homework number
\newcommand\idgroup{ФІ-91}                
\newcommand\idname{Михайло Корешков}  

\usepackage[framemethod=TikZ]{mdframed}
\mdfsetup{%
	backgroundcolor = black!5,
}
\mdfdefinestyle{ans}{%
    backgroundcolor = green!5,
    linecolor = green!50,
    linewidth = 1pt,
}

\pagestyle{fancyplain}
\headheight 35pt
\lhead{\idgroup \\ \idname}
\chead{\textbf{\Large \hwnumber}}
\rhead{\course \\ \today}
\lfoot{}
\cfoot{}
\rfoot{\small\thepage}
\headsep 1.5em

\linespread{1.2}

\begin{document}

% 11.20-11.26, 11.28

% \section*{№ 10.20}
% \begin{mdframed}
%     $\xi, \eta$ - нехалежні в.в.\\
%     $$M\xi = 1, \quad M\eta = 2, \quad D\xi = 1, \quad D\eta = 4$$
% \end{mdframed}

% \begin{mdframed}[backgroundcolor=violet!15]
%     $$D\xi = M\xi^2 - (M\xi)^2$$
% \end{mdframed}

% \subsection*{a)}
% $$\begin{gathered}
%     M(\xi+\eta+1)^2 = M(\xi^2 + \eta^2 + 1 + 2\xi\eta + 2\xi + 2\eta) = \\
%     = M(\xi^2) + M(\eta^2) + 2M(\xi)M(\eta) + 2M\xi + 2M\eta + 1 = \\
%     = D\xi+(M\xi)^2 + D\eta + (M\eta)^2 + 2M(\xi)M(\eta) + 2M\xi +2M\eta + 1 = \\
%     = 1 + 1 + 4 + 4 + 2\cdot2\cdot1 + 2\cdot 1 + 2\cdot 2 = \\
%     = 20
% \end{gathered} $$
% \begin{mdframed}[style=ans]
%     $$M(\xi+\eta+1)^2 = 20$$
% \end{mdframed}

% \subsection*{b)}
% $$\begin{gathered}
%     D\xi\eta = M(\xi^2\eta^2) - (M\xi\eta)^2 = \\
%     = M(\xi^2) M(\eta^2) - (M\xi)^2(M\eta)^2 = \\
%     = \bigl(D\xi+(M\xi)^2\bigr)\bigl( D\eta + (M\eta)^2\bigr) - (M\xi)^2(M\eta)^2 = \\
%     = (1 + 1)(4 + 4) - 1\cdot 4 = \\
%     = 12
% \end{gathered}$$
% \begin{mdframed}[style=ans]
%     $$D\xi\eta = 12$$
% \end{mdframed}

% \section*{№ 10.21}
% \begin{mdframed}
%     $$\xi_1 ... \xi_n \sim U[0;1]$$
%     $$p(x) = \mathbb{1}(x\in[0;1])$$
%     Всі незалежні в сукупності
% \end{mdframed}

% \subsection*{1) $M(\xi_1 + ... + \xi_n)$}
% $$
% \begin{gathered}
%     M(\xi_1 + ... + \xi_n) = M\xi_1 + ... + M\xi_n = n \cdot \frac{1}{2}
% \end{gathered}
% $$

% \subsection*{2) $D(\xi_1 + ... + \xi_n)$}

% $$M\xi^2 = \int_{[0,1]} x^2 dx = \frac{1^3}{3} - 0 = \frac{1}{3}$$
% $$D\xi = M\xi^2 - (M\xi)^2 = \frac{1}{3} - \frac{1}{4} = \frac{1}{12}$$
% $$
% \begin{gathered}
%     D(\xi_1 + ... + \xi_n) = D\xi_1 + ... + D\xi_n = n \cdot \frac{1}{12}
% \end{gathered}
% $$


% \subsection*{3) $M(\sqrt[n]{\xi_1 \cdot ... \cdot \xi_n})$}

% $$M\sqrt[n]{\xi} = \int_{[0;1]} x^{1/n} dx = \frac{n+1}{n} 1^{\frac{n+1}{n}} - 0 = \frac{n+1}{n} $$

% $$M(\sqrt[n]{\xi_1 \cdot ... \cdot \xi_n}) = $$
% $$= \idotsint_{\mathbb{R}^n} \sqrt[n]{x_1} \cdot ... \cdot \sqrt[n]{x_n} f_\xi(x_1, ..., x_n) dx_1 ... dx_n = $$
% $$= \prod_{i=1}^n \int_{\mathbb R} \sqrt[n]{x_i} f_\xi (x_i) dx_i = $$
% $$= \prod_{i=1}^n M\sqrt[n]{\xi_i} = (M\sqrt[n]{\xi})^n = \left(\frac{n+1}{n}\right)^n $$


% \subsection*{4) $D(\xi_1 \cdot ... \cdot \xi_n)$}

% $$
% \begin{gathered}
%     D(\xi_1 \cdot ... \cdot \xi_n) = M(\prod_{i=1}^n \xi_i)^2 - (M \prod_{i=1}^n \xi_i)^2;
% \end{gathered}
% $$

% $$\begin{gathered}
%     M(\prod_{i=1}^n \xi_i)^2 = M\prod_{i=1}^n \xi_i^2 = \prod_{i=1}^n M\xi_i^2 = \\
% = (M\xi^2)^n = \frac{1}{3^n} = 3^{-n}
% \end{gathered}$$

% $$\begin{gathered}
%     M \prod_{i=1}^n \xi_i = \prod_{i=1}^n M\xi_i = \frac{1}{2^n} = 2^{-n}
% \end{gathered}$$

% \begin{mdframed}[style=ans]
%     $$D(\xi_1 \cdot ... \cdot \xi_n) = 3^{-n} - 2^{-n}$$
% \end{mdframed}
% \pagebreak

% (Тут я помітив, що вирішував задачі з 10, а не 11 розділу)

\section*{11.20}
\begin{mdframed}
    $\xi$ - в.в. з ф-цією розподілу $F$

    $F_{(\xi,\xi)}$ - ?
\end{mdframed}

$$F_{(\xi,xi)}(x,y) = P(\xi \le x \wedge \xi \le y) = P(\xi \le \min(x,y)) = F(\min(x,y))$$

\section*{11.21}
\begin{mdframed}
    $$(\xi_1, \xi_2) - \text{в.в. з щільністю розділу } \; p$$
    $$p(x,y) = \frac{c}{1+x^2+x^2y^2+y^2}$$

    Знайти:
    \begin{enumerate}
        \item $c - ?$
        \item $p_{\xi_1}, p_{\xi_2}$
        \item $P(|\xi_1|\le 1, |\xi_2| \le 1)$
        \item Чи є $\xi_1, \xi_2$ незалежними?
    \end{enumerate}
\end{mdframed}

\subsection*{1.}
З означення щільності розподілу:
$$\int_{\mathbb R^2} \frac{c}{1+x^2+x^2y^2+y^2} dxdy = 1$$
$$c \cdot \iint_{R^2} \frac{1}{1+x^2+x^2y^2+y^2} dxdy = $$
$$c \cdot \int_{R} \frac{\arctan y}{1 + x^2} |_{-\infty}^\infty dx = $$
$$c \cdot \int_{R} \frac{\pi}{1 + x^2} |_{-\infty}^\infty dx = $$
$$\pi c \cdot \arctan x |_{-\infty}^\infty = \pi^2 c = 1$$

\begin{mdframed}[style=ans]
    $$c = \frac{1}{\pi^2}$$
\end{mdframed}

\subsection*{2.}
$$p_{\xi_1}(x) = \int_{-\infty}^\infty p(x,y) dy = $$
$$= \frac{1}{\pi^2} \int_{-\infty}^\infty \frac{1}{1+x^2+x^2y^2+y^2} dy = $$
$$= \frac{1}{\pi^2} \frac{\arctan y}{1 + x^2} |_{y = -\infty}^\infty = \frac{1}{\pi(1+x^2)}$$

$$p_{\xi_1}(x) = \frac{1}{\pi(1+x^2)}$$
$$p_{\xi_2}(y) = \frac{1}{\pi(1+y^2)}$$

\subsection*{3.}
$$P(|\xi_1|\le 1, |\xi_2| \le 1) = \int_{-1}^{1} \int_{-1}^1 \frac{1}{1+x^2+x^2y^2+y^2} dxdy = $$
$$ = \frac{1}{\pi^2} \int_{-1}^{1} \frac{\arctan y}{1 + x^2} |_{y=-1}^1 dx = $$ 
$$ = \frac{1}{\pi^2} \frac{\pi}{2} \int_{-1}^{1} \frac{1}{1 + x^2} dx = \frac{1}{\pi^2} \frac{\pi^2}{4} = $$
$$ = \frac{1}{4}$$

\subsection*{4.}
$$p_{\xi_1}(x)p_{\xi_2}(y) = \frac{1}{\pi(1+x^2)}\frac{1}{\pi(1+y^2)} = $$
$$= \frac{1}{\pi^2} \cdot \frac{1}{1 + x^2 + y^2 + x^2y^2} = p(x,y)$$
Тобто, $\xi_1, \xi_2$ - незалежні \qedsymbol

\section*{№ 11.22}
\begin{mdframed}
    $$\xi,\eta \sim Exp(\alpha), \; \alpha >0$$
    $$\xi, \eta - \;\text{незалежні}$$
    $$\boxed{\xi, \eta \ge 0}$$

    Довести:
    $$\zeta = \frac{\xi}{\xi+\eta} \sim U[0;1]$$
\end{mdframed}

$$p_\xi(x) = p_\eta(x) = p(x) = \alpha e^{-\alpha x} \mathbb{1}_{\ge 0}(x)$$

$$\frac{\partial \zeta}{\partial \xi} = \frac{\eta}{(\xi+\eta)^2}$$
$$\frac{\partial \zeta}{\partial \eta} = -\frac{\xi}{(\xi+\eta)^2}$$

Нехай $\gamma = (\zeta, \xi+\eta)$

$$\begin{cases}
    y_1 = \frac{x_1}{x_1+x_2} > 0\\
    y_2 = x_1+x_2 > 0
\end{cases}$$

Також:
$$\boxed{0 < y_1 \le 1}$$

$$\begin{cases}
    x_1 = y_1 y_2 \\
    x_2 = y_2 - y_1 y_2 = y_2 (1-y_1)
\end{cases}$$

$$
J = \begin{vmatrix}
    \frac{\partial x_1}{\partial y_1} & \frac{\partial x_1}{\partial y_2} \\
    \frac{\partial x_2}{\partial y_1} & \frac{\partial x_2}{\partial y_2} \\
\end{vmatrix} 
= \begin{vmatrix}
    y_2 & y_1 \\
    -y_2 & 1-y_1 \\
\end{vmatrix} = y_2 - y_1y_2 + y_1 y_2 = y_2
$$

$$p_\gamma(y_1,y_2) = p_{(\xi,\eta)} (y_1 y_2, y_2 (1-y_1)) \cdot y_2 \mathbb{1}_{> 0}(x)\mathbb{1}_{\le 1}(x)\mathbb{1}_{> 0}(y)$$

$$p_{(\xi,\eta)} (x y, y (1-x)) = p_\xi(xy) p_\eta(y(1-x)) 
= \alpha e^{-\alpha xy} \cdot \alpha e^{-\alpha y(1-x)} \mathbb{1}_{> 0}(x)\mathbb{1}_{\le 1}(x)\mathbb{1}_{> 0}(y) $$

$$p_\gamma(x,y) = p_{(\zeta,\xi+\eta)}(x,y) = \alpha e^{-\alpha xy} \cdot \alpha e^{-\alpha y(1-x)} \cdot y 
= \alpha^2 ye^{-\alpha y} \mathbb{1}_{> 0}(x)\mathbb{1}_{\le 1}(x)\mathbb{1}_{> 0}(y)$$

В принципі, ми знаємо, що якщо $\xi, \eta$ незалежні, то й $\gamma_1, \gamma_2$ - незалежні.
Тобто має бути $p_\gamma(x,y) = p_\zeta(x) \cdot p_{\xi+\eta}(y)$. 
Тобто, $p_\zeta(x) = const$.

Формально:

$$p_\zeta(x) = \int_{-\infty}^\infty p_\gamma(x,y)\mathbb{1}_{> 0}(x)\mathbb{1}_{\le 1}(x)\mathbb{1}_{> 0}(y) dy 
= \int_{-\infty}^\infty \alpha^2 ye^{-\alpha y}\mathbb{1}_{> 0}(x)\mathbb{1}_{\le 1}(x)\mathbb{1}_{> 0}(y) dy =$$
$$= - \left. (\alpha y + 1) e^{-\alpha y} \right|_{0}^\infty \mathbb{1}_{> 0}(x)\mathbb{1}_{\le 1}(x)
= 1 \cdot \mathbb{1}_{> 0}(x)\mathbb{1}_{\le 1}(x)$$

Дійсно, $\zeta \sim U[0;1]$ \qedsymbol

\section*{№ 11.23}
\begin{mdframed}
    $$F_{\xi_1} = ... = F_{\xi_n} = F$$
    $$\xi_i - \text{ незалежні}$$
    Нехай 
    $$\xi_{(1)} \le \xi_{(2)} \le ... \le \xi_{(n)}$$
\end{mdframed}

\subsection*{a)}
$$\alpha = \xi_{(1)} = \min(\xi_i); \quad F_{\alpha} - ?$$

$$F_\alpha(x) = P(\min(\xi_i) \le x) = 1 - P(\min(\xi_i) > x) = 1 - \prod_i P(\xi_i > x) =$$
$$= 1 - \prod_{i=1}^n (1-F_{\xi_i}(x)) = 1 - (1-F(x))^n $$


\subsection*{b)}
$$\alpha = \xi_{(n)} = \max(\xi_i); \quad F_{\alpha} - ?$$

$$F_\alpha(x) = P(\max(\xi_i) \le x) = \prod_i P(\xi_i \le x) = \prod_i F(x) = F(x)^n$$

\subsection*{c)}
$$\alpha = \xi_{(m)}$$

$$P(\xi_{(m)} \le x) = P(\text{не менше m величин $\xi$ є $\le x$}) = $$
$$= \sum_{k=m}^n C_n^k P(\xi \le x)^k P(\xi > x)^{n-k} = $$
$$= \sum_{k=m}^n C_n^k F(x)^k (1-F(x))^{n-k}$$
\pagebreak

\section*{№11.24}
\begin{mdframed}
    $$\xi_1, \xi_2 \sim N(0,1)$$
    $\xi_i$ - незалежні
    $$p(x) = \frac{1}{\sqrt{2 \pi}} e^{-\frac{x^2}{2}}$$

    $$\eta = \xi_1^2 + \xi_2^2$$
    $$p_\eta - ?$$
\end{mdframed}

Let $\alpha = \xi^2$
$$F_\alpha(x) = P(\xi^2 \le x) = P(|\xi| \le \sqrt{x}) = F(\sqrt{x}) - F(-\sqrt{x})$$
$$p_\alpha(x) = \frac{p(\sqrt x) + p(-\sqrt x)}{2\sqrt{x}} \cdot \mathbb{1}_{\ge 0}(x)$$

$$p_\alpha(x) = \frac{1}{\sqrt{2 \pi}} \cdot \frac{e^{-\frac{|x|}{2}} + e^{-\frac{|x|}{2}}}{2\sqrt{x}} \cdot \mathbb{1}_{\ge 0}(x) = $$
$$= \frac{1}{\sqrt{2 \pi}} \cdot \frac{e^{-\frac{|x|}{2}}}{\sqrt{x}} \cdot \mathbb{1}_{\ge 0}(x)$$

$$p_\eta = p_\alpha \ast p_\alpha$$
$$p_\eta(x) = \int_{-\infty}^\infty p_\alpha(y)p_\alpha(x-y) dy = $$
$$= \frac{1}{2\pi} \int_{-\infty}^\infty \frac{e^{-\frac{|y|}{2}}}{\sqrt{y}} \cdot \frac{e^{-\frac{|x-y|}{2}}}{\sqrt{x-y}} \cdot \mathbb{1}_{\ge 0}(x) \cdot \mathbb{1}_{\ge 0}(y) \cdot \mathbb{1}_{\ge 0}(x-y) = $$
$$=\mathbb{1}_{\ge 0}(x) \frac{1}{2\pi} \int_{0}^\infty \frac{e^{-\frac{y}{2}}}{\sqrt{y}} \cdot \frac{e^{-\frac{x-y}{2}}}{\sqrt{x-y}} \cdot \mathbb{1}_{\ge 0}(x-y) = $$
$$=\mathbb{1}_{\ge 0}(x) \frac{1}{2\pi} \int_{0}^\infty \frac{e^{-\frac{x}{2}}}{\sqrt{y}\sqrt{x-y}} \cdot \mathbb{1}_{\ge 0}(x-y) = $$
$$=\mathbb{1}_{\ge 0}(x) \frac{e^{-\frac{x}{2}}}{2\pi} \int_{0}^\infty \frac{1}{\sqrt{y}\sqrt{x-y}} \cdot \mathbb{1}_{\ge 0}(x-y) = $$
$$=\mathbb{1}_{\ge 0}(x) \frac{e^{-\frac{x}{2}}}{2\pi} \int_{0}^x \frac{1}{\sqrt{y}\sqrt{x-y}} = $$
$$=\mathbb{1}_{\ge 0}(x) \frac{e^{-\frac{x}{2}}}{2\pi} \left.-\arcsin(\frac{x-2y}{x})\right|_{y=0}^x = $$
$$=\mathbb{1}_{\ge 0}(x) \frac{e^{-\frac{x}{2}}}{2\pi} (\arcsin(1) - \arcsin(-1)) = \mathbb{1}_{\ge 0}(x) \frac{e^{-\frac{x}{2}}}{2}$$

Тобто $\eta \sim Exp(-\frac{1}{2})$

\section*{№ 11.25}
\begin{mdframed}
    $$\xi_1, \xi_2 \sim U[0;1]$$
    $\xi_i$ - незалежні

    $$P\{\xi_1 < x \wedge \xi_1 + 2\xi_2 < y\}$$
\end{mdframed}

$$D = \{(u,v) \in R^2 : u < x \wedge u + 2v < y\}$$
$$P = \iint_D f(x)f(y) dxdy = \int_{u=0}^{\min(x,y,1)} \int_{v=0}^{\frac{y-u}{2}} \mathbb{1}_D(x,y) du dv= $$
$$= \int_{u=0}^{\min(x,y,1)} \frac{y-u}{2} du \cdot \mathbb{1}(x,y \ge 0) = $$
$$=\boxed{\left(\frac{\min(x,y,1)y}{2} - \frac{\min(x,y,1)^2}{4}\right) \mathbb{1}(x,y \ge 0)}$$

\section*{№ 11.26}
\begin{mdframed}
    $$p(x,y,z)=\begin{cases}
        \frac{1-\sin x\sin y\sin z}{8\pi^3}, & 0\le x,y,z \le 2\pi\\
        0, \text{otherwise}
    \end{cases}$$
\end{mdframed}

$$p_{xy}(x,y) = \frac{1}{8\pi^3} \int_0^{2\pi} (1-\sin x\sin y\sin z) dz = $$
$$= \frac{1}{8\pi^3} (z+\sin x\sin y\cos z) |_{z=0}^{2\pi} = \frac{1}{8\pi^3} (2\pi + 0) = \frac{1}{4\pi^2}$$

$$p_{x}(x) = \frac{1}{8\pi^3} \int_0^{2\pi}\int_0^{2\pi} (1-\sin x\sin y\sin z) dydz = $$
$$= \frac{1}{8\pi^3} \int_0^{2\pi} (z+\sin x\cos y\sin z) |_{y=0}^{2\pi} dz = \frac{1}{4\pi^2} \int_0^{2\pi} dz = $$
$$= \frac{1}{2\pi}$$

Отже,
$$p_{xy}(x,y) = p_{xz}(x,z) = p_{yz}(y,z) =\frac{1}{4\pi^2} $$
$$p_x(x) = p_y(y) = p_z(z) = \frac{1}{2\pi}$$

$$p_x(x)p_y(y) = p_{xy}(x,y)$$
Аналогічно для інших пар

Проте
$$p_x(x)p_y(y)p_z(z) \ne p(x,y,z)$$
\qedsymbol

\section*{№ 11.28}
$$F_X(x) = F_Y(y) = \begin{cases}
    0,& x<-1\\
    1/6,& x\le -1\\ 
    5/6,& x\le 0\\ 
    1,& x\ge 1\\ 
\end{cases}$$

$$P(X\le 0, Y\le 0) = 1/3$$
$$P(X\le 0)P(Y\le 0) = (5/6)^2 \ne 1/3$$
Тобто, $X$ та $Y$ залежні

$$MX = MY = -1\cdot 1/6 + 0\cdot 4/6 + 1 \cdot 1/6 = 0$$

$$\rho(X,Y) = \frac{1}{\sigma_X \sigma_Y} \cdot \operatorname{cov}(X,Y) = \frac{1}{\sigma_X \sigma_Y} M(X-MX)(Y-MY) = $$
$$=\frac{1}{\sigma_X \sigma_Y} M(XY) = 0 \;\text{Бо ймовірність, що обидва X та Y ненульові - нуль}$$
Тобто, $X$ та $Y$ не корелюють. 

\end{document}

