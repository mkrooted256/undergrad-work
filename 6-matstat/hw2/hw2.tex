% !TEX TS-program = xelatex
% !TEX encoding = UTF-8

\documentclass[11pt, a4paper]{article} % use larger type; default would be 10pt

\usepackage{fontspec} % Font selection for XeLaTeX; see fontspec.pdf for documentation
\defaultfontfeatures{Mapping=tex-text} % to support TeX conventions like ``---''
\usepackage{xunicode} % Unicode support for LaTeX character names (accents, European chars, etc)
\usepackage{xltxtra} % Extra customizations for XeLaTeX
\usepackage{tikz}
\usetikzlibrary{arrows,calc,patterns}

\setmainfont[Ligatures=TeX]{Garamond} % set the main body font (\textrm), assumes Charis SIL is installed
%\setsansfont{Deja Vu Sans}
\setmonofont[Ligatures=TeX]{Fira Code}

% other LaTeX packages.....
\usepackage{fullpage}
\usepackage[top=2cm, bottom=4.5cm, left=2.5cm, right=2.5cm]{geometry}
\usepackage{amsmath,amsthm,amsfonts,amssymb,amscd,systeme}
\usepackage{unicode-math}
\usepackage{cancel}
\geometry{a4paper} 
%\usepackage[parfill]{parskip} % Activate to begin paragraphs with an empty line rather than an indent
\usepackage{fancyhdr}
\usepackage{listings}
\usepackage{graphicx}
\usepackage{hyperref}
\usepackage{multicol}

\renewcommand\lstlistingname{Algorithm}
\renewcommand\lstlistlistingname{Algorithms}
\def\lstlistingautorefname{Alg.}
\lstdefinestyle{mystyle}{
    % backgroundcolor=\color{backcolour},   
    % commentstyle=\color{codegreen},
    % keywordstyle=\color{magenta},
    % numberstyle=\tiny\color{codegray},
    % stringstyle=\color{codepurple},
    basicstyle=\ttfamily\footnotesize,
    breakatwhitespace=false,         
    breaklines=true,                 
    captionpos=b,                    
    keepspaces=true,                 
    numbers=left,                    
    numbersep=5pt,                  
    showspaces=false,                
    showstringspaces=false,
    showtabs=false,                  
    tabsize=2
}
\lstset{style=mystyle}

\newcommand\course{6 - Матстатистика}
\newcommand\hwnumber{ДЗ №1}                   % <-- homework number
\newcommand\idgroup{ФІ-91}                
\newcommand\idname{Михайло Корешков}  

\usepackage[framemethod=TikZ]{mdframed}
\mdfsetup{%
	backgroundcolor = black!5,
}
\mdfdefinestyle{ans}{%
    backgroundcolor = green!5,
    linecolor = green!50,
    linewidth = 1pt,
}

\pagestyle{fancyplain}
\headheight 35pt
\lhead{\idgroup \\ \idname}
\chead{\textbf{\Large \hwnumber}}
\rhead{\course \\ \today}
\lfoot{}
\cfoot{}
\rfoot{\small\thepage}
\headsep 1.5em

\linespread{1.2}

\begin{document}

% 2.17 18 19 20 21 22 23

\section*{№ 2.17}
\begin{mdframed}
    $X_1, ..., X_n \sim U[0,\theta]$

    Перевірити незміщеність, змістовність та знайти meansq відхилення оцінок
\end{mdframed}
\subsection*{a)}
$$T_n = X_{(1)} + X_{(n)}$$

$$F_{(1)}(x) = P(\min(\overline X) \le x) = 1 - \left(1 - F(x)\right)^n = \begin{cases}
    1 - (1-\frac{x}{\theta})^n, & x \in (0, \theta) \\
    0, & x \le 0 \\
    1, & x \ge \theta
\end{cases} $$
$$f_{(1)}(x) = \mathbb{1}[0;\theta] \cdot \frac{n}{\theta} (1-\frac{x}{\theta})^{n-1}$$

$$F_{(n)}(x) = P(\max(\overline X) \le x) = \left(F(x)\right)^n = \begin{cases}
   \frac{x^n}{\theta^n}, & x \in (0, \theta) \\
    0, & x \le 0 \\
    1, & x \ge \theta
\end{cases} $$
$$f_{(n)}(x) = \mathbb{1}[0;\theta] \cdot \frac{nx^{n-1}}{\theta^n}$$


$$MX_{(n)} = \int_0^\theta x \cdot n\frac{x^{n-1}}{\theta^n} dx = \left. \frac{n}{n+1} \frac{x^{n+1}}{\theta^n} \right|_0^\theta = \frac{n}{n+1}\theta$$
\begin{multline*}
    MX_{(1)} = \int_0^\theta x \; d \left( F_{(1)}(x) \right) = 
    \left.x F_{(1)}(x)\right|_{x=0}^\theta - \int_0^\theta F_{(1)}(x) dx
    = \theta  - 0 - \int_0^\theta \left(1 - (1-\frac{x}{\theta})^n\right) dx = \\
    = \theta - \theta - \frac{\theta}{n+1} \left.\left(1-\frac{x}{\theta}\right)^{n+1}\right|_0^\theta = \\
    = - 0 + \frac{\theta}{n+1}
\end{multline*}

Тобто $$MT = MX_{(n)} + MX_{(1)} = \theta \cdot \left(\frac{n}{n+1} + \frac{1}{n+1}\right) = \theta$$
\begin{mdframed}[style=ans]
    $T$ - незміщена для $\theta$
\end{mdframed}

Перевіряємо, чи $T_n \overset{P}{\longrightarrow} \theta$?
Це еквівалентно 
$$\forall \varepsilon>0: P(|T_n - \theta| > \varepsilon) \longrightarrow 0$$

Помічу, що $X_{(0)}, X_{(n)} \in [0;\theta]$. Тобто $T_n \in [0;2\theta]$

$$P(|T_n - \theta| > \varepsilon) = P(|X_{(1)} + X_{(n)} - \theta| > \varepsilon) 
\le $$

$$P(|T_n - \theta| > \varepsilon) \le \frac{D|T_n - \theta|}{\varepsilon^2} \le \frac{DT_n + \theta}{\varepsilon^2}$$

$$MT^2 = M(X_{(1)}+X_{(n)})^2 \ge $$

// TODO


\subsection*{b)}
$$T_n = (n+1) X_{(1)}$$

$$MT_n = (n+1) MX_{(1)} = (n+1) \frac{\theta}{n+1} = \theta$$
\begin{mdframed}[style=ans]
    незміщена
\end{mdframed}

\begin{multline*}
    DX_{(1)} = \int_0^\theta (x-\theta)^2 \cdot \frac{n}{\theta} (1-\frac{x}{\theta})^{n-1} dx
    = - \int_0^\theta n \frac{(x-\theta)^{n+1}}{\theta^n} dx 
    = - \left. \frac{n}{n+2} \frac{(x-\theta)^{n+2}}{\theta^n} \right|_0^\theta  
    = \theta^2\frac{n}{n+2}
\end{multline*}
$$DT_n = \theta^2 \frac{n(n+1)^2}{n+2}$$

$$D|T_n - \theta| \le D|T_n| + D|\theta| = DT_n$$

\section*{2.18}
\begin{mdframed}
    $X_1,...,X_n \sim Exp(1/\sqrt a)$ 

    $T_n = (\overline X)^2$
    Перевірити незміщеність та змістовність $T$ як оцінки для $a$.
\end{mdframed}

$$f_{X_1}(x) = \mathbb{1}(x > 0) \cdot \frac{1}{\sqrt a} e^{-1/\sqrt a}$$
$$MX_1 = \sqrt a, \quad DX_1 = a$$

$$M(\overline X)^2 = ?$$

// TODO


\end{document}

