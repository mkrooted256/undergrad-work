% !TEX TS-program = xelatex
% !TEX encoding = UTF-8

\documentclass[11pt, a4paper]{article} % use larger type; default would be 10pt

\usepackage{fontspec} % Font selection for XeLaTeX; see fontspec.pdf for documentation
\defaultfontfeatures{Mapping=tex-text} % to support TeX conventions like ``---''
\usepackage{xunicode} % Unicode support for LaTeX character names (accents, European chars, etc)
\usepackage{xltxtra} % Extra customizations for XeLaTeX
\usepackage{tikz}
\usetikzlibrary{arrows,calc,patterns}

\setmainfont[Ligatures=TeX]{Garamond} % set the main body font (\textrm), assumes Charis SIL is installed
%\setsansfont{Deja Vu Sans}
\setmonofont[Ligatures=TeX]{Fira Code}

% other LaTeX packages.....
\usepackage{fullpage}
\usepackage[top=2cm, bottom=4.5cm, left=2.5cm, right=2.5cm]{geometry}
\usepackage{amsmath,amsthm,amsfonts,amssymb,amscd,systeme}
\usepackage{unicode-math}
\usepackage{cancel}
\geometry{a4paper} 
%\usepackage[parfill]{parskip} % Activate to begin paragraphs with an empty line rather than an indent
\usepackage{fancyhdr}
\usepackage{listings}
\usepackage{graphicx}
\usepackage{hyperref}
\usepackage{multicol}

\renewcommand\lstlistingname{Algorithm}
\renewcommand\lstlistlistingname{Algorithms}
\def\lstlistingautorefname{Alg.}
\lstdefinestyle{mystyle}{
    % backgroundcolor=\color{backcolour},   
    % commentstyle=\color{codegreen},
    % keywordstyle=\color{magenta},
    % numberstyle=\tiny\color{codegray},
    % stringstyle=\color{codepurple},
    basicstyle=\ttfamily\footnotesize,
    breakatwhitespace=false,         
    breaklines=true,                 
    captionpos=b,                    
    keepspaces=true,                 
    numbers=left,                    
    numbersep=5pt,                  
    showspaces=false,                
    showstringspaces=false,
    showtabs=false,                  
    tabsize=2
}
\lstset{style=mystyle}

\newcommand\course{6 - Матстатистика}
\newcommand\hwnumber{Домашня КР №1}                   % <-- homework number
\newcommand\idgroup{ФІ-91}                
\newcommand\idname{Михайло Корешков}  

\usepackage[framemethod=TikZ]{mdframed}
\mdfsetup{%
	backgroundcolor = black!5,
}
\mdfdefinestyle{ans}{%
    backgroundcolor = green!5,
    linecolor = green!50,
    linewidth = 1pt,
}

\pagestyle{fancyplain}
\headheight 35pt
\lhead{\idgroup \\ \idname}
\chead{\textbf{\Large \hwnumber}}
\rhead{\course \\ \today}
\lfoot{}
\cfoot{}
\rfoot{\small\thepage}
\headsep 1.5em

\linespread{1.2}

\begin{document}

\section*{№1}
\begin{mdframed}
    $X_i$ - час обслуговування $i$-го покупця. $X_1, ..., X_n \sim \xi$ - iid.

    $M\xi = 1.4$, $D\xi = 1$

    $$P(\sum_{i=1}^{100} X_i < 120) - ?$$
\end{mdframed}

Розглянемо в.в. $T_n = \sum_{j=1}^n X_j$.

$$MT_n = n \cdot M\xi = 1.4n = 140, \quad DT_n = n \cdot D\xi = n = 100$$

Використовуючи центральну граничну теорему,
\begin{align*}
P(\sum_{i=1}^{100} X_i < 120) &= P(T_100 < 120) = P(\frac{T_100 - 140}{100} < \frac{120 - 140}{100}) \approx \\
&\approx \Phi_{0,1}(-\frac{20}{100}) = 1-\Phi_{0,1}(0.2) = 0.42
\end{align*}

\begin{mdframed}[style=ans]
    $$P(\sum_{i=1}^{100} X_i < 120) \approx 0.42$$
\end{mdframed}

\section*{№2}
\begin{mdframed}
    $$D\xi = 5, \quad M\xi = m \text{(unknown)}$$
    $$X_i \sim \xi - iid$$
    $$P(|\overline X_n - m| > 1) < 0.05$$

    $$n - ?$$
\end{mdframed}

$$\overline X_n = \frac{1}{n} \sum_{i=1}^n X_i$$
$$M\overline X_n = \frac{1}{n} \cdot n \cdot m = m$$
$$D\overline X_n = \frac{1}{n^2} \cdot n \cdot 5 = \frac{5}{n}$$

$$P(|\overline X_n - m| > 1) = P(\overline X_n - m > 1) + P(\overline X_n - m < -1)$$
$$P(\overline X_n - m > 1) = 1 - P(\frac{n}{5} \cdot (\overline X_n - m) < \frac{n}{5}) \approx 1 - \Phi_{0,1}\left(\frac{n}{5}\right)$$
$$P(\overline X_n - m < -1) = P(\frac{n}{5} \cdot (\overline X_n - m) < -\frac{n}{5}) \approx \Phi_{0,1}\left(-\frac{n}{5}\right) = 1 - \Phi_{0,1}\left(\frac{n}{5}\right)$$
Тут була використана ЦГТ.

$$P(|\overline X_n - m| > 1) = 2\left(1 - \Phi_{0,1}\left(\frac{n}{5}\right)\right) < 0.05$$
$$\Phi_{0,1}\left(\frac{n}{5}\right) > 1-0.025 = 0.975$$
Використовуючи калькулятор оберненої cdf нормального розподілу отримано
$$\frac{n}{5} > 1.96$$
$$n \ge 10 > 9.8$$

\begin{mdframed}[style=ans]
    $$n \ge 10$$
\end{mdframed}

\section*{№3}
\begin{mdframed}
    $$X_1,...,X_n \sim Exp\left(\frac{1}{\theta}\right) - iid$$
    $$\theta^*_1 = \overline X, \quad \theta^*_2 = nX_{(1)}$$

    \begin{itemize}
        \item Переконатися в незміщеності
        \item Визначити, яка оцінка краще в meansqr сенсі
    \end{itemize}
\end{mdframed}

$$f(x) = \frac{1}{\theta}e^{-\frac{1}{\theta}x} \cdot \mathbb{1}(x \ge 0)$$
$$F(x) = 1 - e^{-\frac{x}{\theta}}$$
$$MX_1 = \theta, \quad DX_1 = \theta^2$$

\subsection*{1)}
$$M\theta^*_1 = M\left(\frac{1}{n} (X_1 + ... + X_n)\right) = \frac{1}{n} \cdot n \cdot MX_1 = \theta$$

$$F_{(1)} = P(X_{(1)} < x) = 1 - P(X_{(1)} > x) = 1 - \left(P(X_1 > x)\right)^n = 1 - \left(1 - F(x)\right)^n =$$
$$= 1 - e^{-n\frac{x}{\theta}}$$
$$f_{(1)} = e^{-n\frac{x}{\theta}} \cdot \frac{n}{\theta}$$
\begin{align*}    
MX_{(1)} &= \frac{n}{\theta} \int_0^\infty x e^{-n\frac{x}{\theta}} dx 
=  -\int_0^\infty x d\left(e^{-n\frac{x}{\theta}}\right) = \\
&= 0 + \int_0^\infty e^{-n\frac{x}{\theta}} dx 
= -\frac{\theta}{n} \cdot (0-1) = \frac{\theta}{n}
\end{align*}

$$M\theta^*_2 = M\left(nX_{(1)}\right) = n \cdot \frac{\theta}{n} = \theta$$

\begin{mdframed}[style=ans]
    Незміщеність доведено
\end{mdframed}

\subsection*{2)}
Порівняння оцінок з рівним матсподіванням зводиться до порівняння їх дисперсій.

$$D\theta^*_1 = \frac{1}{n} DX_1 = \frac{\theta^2}{n}$$

\begin{align*}    
    DX_{(1)} &= \frac{n}{\theta} \int_0^\infty (x-\theta)^2 e^{-n\frac{x}{\theta}} d(x-\theta) 
    =  -\int_0^\infty (x-\theta)^2 d\left(e^{-n\frac{x}{\theta}}\right) = \\
    &= 0 + \int_0^\infty (x-\theta) e^{-n\frac{x}{\theta}} d(x-\theta)
    = +\frac{\theta^2}{n} \cdot 1 + \frac{\theta}{n} \int_0^\infty e^{-n\frac{x}{\theta}} dx = \\
    &= \frac{\theta^2}{n} -\frac{\theta^2}{n^2} \cdot (0-1) = \frac{\theta^2}{n} + \frac{\theta^2}{n^2} = \\
    &= \theta^2 \frac{n+1}{n^2}
    \end{align*}

$$D\theta^*_2 = D\left(nX_{(1)}\right) = n^2 DX_{(1)} = \theta^2$$

\begin{mdframed}[style=ans]
    $\theta^*_1$ краща, бо $\frac{n+1}{n^2} < 1$ для $n\ge 2$.
\end{mdframed}

\section*{№4}
\begin{mdframed}
    $$X_1, X_2, X_3 \sim \mathcal N(\theta, \sigma^2) - iid$$
    $\theta$ - unknown, $\sigma^2$ - known.

    Яка оцінка краща в meansqr сенсі?
    \begin{itemize}
        \item $\theta^*_1 = (X_1 + X_2 + X_3)/3$
        \item $\theta^*_2 = (X_1 + 2X_2 + X_3)/4$
    \end{itemize}
\end{mdframed}

$$M\theta^*_1 = \frac{1}{3}(MX_1 + MX_2 + MX_3) = \frac{3}{3}\theta = \theta$$
$$M\theta^*_2 = \frac{1}{4}(MX_1 + 2MX_2 + MX_3) = \frac{4}{4}\theta = \theta$$

Достатньо обчислити дисперсії

$$D\theta^*_1 = \frac{1}{9}(3\cdot DX_1) = \frac{\sigma^2}{3}$$
$$D\theta^*_2 = \frac{1}{16}(\sigma^2 + 4 \sigma^2 + \sigma^2) = \frac{3}{8}\sigma^2$$

$$1/3 = 3/9 < 3/8$$
\begin{mdframed}[style=ans]
    Отже, перша оцінка краща за другу в meansqr сенсі
\end{mdframed}

\section*{№5}
$$X_1, ..., X_n \sim \xi - iid$$
$$f(y) = \frac{2y}{\theta^2} \mathbb{1}(y \in (0;\theta))$$

\subsection*{a) Побудова точкової оцінки}

Почну метод моментів з першого степеню (матсподівання):
$$MX_1 = \int_0^\theta y\frac{2y}{\theta^2} dy = \frac{2}{3\theta^2}\left. y^3 \right|_0^\theta = \frac{2}{3}\theta$$
Тоді $$\theta^* = \frac{3}{2}\overline{X}$$

Незміщеність:
$$M\theta^* = \frac{3}{2} \frac{n}{n} MX_1 = \frac{3}{2} \cdot \frac{2}{3} \theta = \theta$$

Змістовність:
$$\frac{3}{2}\overline{X} \underset{n \to \infty}{\longrightarrow} \frac{3}{2} MX_1 = \theta$$

Meansqr відхилення:
$$M(\theta^*)^2 = \frac{9}{4n^2}M(\sum_{i=1}^n X_i)^2 = \frac{9}{4n^2} M(\sum_{i=1}^n X_i^2 + \sum_{1\le i<j \le n} X_iX_j) $$
$$MX_i^2 = \int_0^\theta y^2\frac{2y}{\theta^2} dy = \frac{2}{4\theta^2}\left. y^4 \right|_0^\theta = \frac{1}{2}\theta^2$$
$$MX_iX_j = \text{ (незалежність) } = MX_i \cdot MX_j = \frac{4}{9}\theta^2$$
Кількість таких доданків у сумі - $n(n-1)$.

$$M(\theta^*)^2 = \frac{9}{4n^2} \left(n \cdot \frac{1}{2}\theta^2 + n(n-1)\frac{4}{9}\theta^2\right) = $$
$$= \theta^2 \left(\frac{9}{8n} + 1 - 1/n\right) = \theta^2 \left(1 + \frac{1}{8n}\right)$$

$$M(\theta^* - \theta)^2 = M(\theta^*)^2 + \theta^2 - 2\theta \cdot M\theta^* = \theta^2 (1 + \frac{1}{8n} + 1 - 2) = \frac{\theta^2}{8n} $$ 

\subsection*{б) ОМП}

\begin{align*}    
L(\vec x, \theta) &= \prod_{i=1}^n \frac{2x_i}{\theta^2} \mathbb{1}(x_i \in (0;\theta)) = \frac{2^n}{\theta^{2n}} \prod_{i=1}^n x_i \mathbb{1}(x_{(1)} > 0,\; x_{(n)}<\theta) = \\
&= \theta^{-2n} \cdot \mathbb{1}(\theta > x_{(n)}) \cdot h(\vec x)
\end{align*}

$\theta^{-2n}$ - спадна функція. Для максимізації необхідно взяти мінімально можливе значення $\theta$ - $x_{(n)}$.

Отже, $$\hat\theta = x_{(n)}$$ - оцінка максимальної правдоподібності для $\theta$.

Незміщеність:

$$F(x) = \int_0^x \frac{2y}{\theta^2} dy = \frac{x^2}{\theta^2}$$
$$P(X_{(n)} < y) = F(y)^n = \left(\frac{x}{\theta}\right)^{2n}, \quad y\in(0;\theta)$$
$$f_{(n)}(x) = \frac{2n}{\theta^{2n}} x^{2n-1} $$
$$Mx_{(n)} = \int_0^\theta x\cdot \frac{2n}{\theta^{2n}} x^{2n-1} dx = \frac{2n\theta^{2n+1}}{\theta^{2n}(2n+1)} = \frac{\theta}{1+1/(2n)}$$

Не незміщена, але асимптотично незміщена.

Змістовність:

$$P(\theta - x_{(n)} > \varepsilon) = P(x_{(n)} < \theta - \varepsilon) = \left(\frac{\theta - \varepsilon}{\theta}\right)^{2n} = \left(1-\frac{\varepsilon}{\theta}\right)^{2n}$$
$$1-\frac{\varepsilon}{\theta}<1 \; \implies \; \forall \varepsilon>0: P(\theta - x_{(n)} > \varepsilon) \underset{n \to \infty}{\longrightarrow} 0$$

\subsection*{в) Байесова оцінка}

Апріорний розподіл:
$$\theta \sim U[1;2]$$
$$f_\theta(t) = 1 \cdot \mathbb{1}(1 \le t \le 2)$$

Вже обчислювали, що 
$$L(\vec y, t) = \frac{2^n}{t^{2n}} \prod_{i=1}^n y_i \mathbb{1}(y_{(1)} > 0,\; y_{(n)}<t)$$

Байесова оцінка:
$$\theta_B = h(\vec x)$$
$$h(\vec y) = M(\theta \;|\; X_1=y_1, ..., X_n=y_n)$$

$$h(\vec y) = \frac{\int_\theta t f_\theta(t) L(\vec y, t) dt}{\int_\theta f_\theta(t) L(\vec y, t) dt} = $$
$$= \frac{\int_1^2 t \frac{2^n}{t^{2n}} \prod_{i=1}^n y_i \mathbb{1}(y_{(1)} > 0,\; y_{(n)}<t) dt}{\int_1^2 \frac{2^n}{t^{2n}} \prod_{i=1}^n y_i \mathbb{1}(y_{(1)} > 0,\; y_{(n)}<t) dt} = $$
$$= \frac{\int_{y_{(n)}}^2 t \frac{2^n}{t^{2n}} \prod_{i=1}^n y_i dt}{\int_{y_{(n)}}^2 \frac{2^n}{t^{2n}} \prod_{i=1}^n y_i dt} = $$
$$= \frac{\int_{y_{(n)}}^2 t^{1-2n} dt}{\int_{y_{(n)}}^2 t^{-2n} dt} = \frac{1-2n}{2-2n} \cdot \frac{2^{2-2n} - y_{(n)}^{2-2n}}{2^{1-2n} - y_{(n)}^{1-2n}} = $$
$$= \left(1-\frac{1}{2-2n}\right) \cdot \frac{2^{2-2n} - y_{(n)}^{2-2n}}{2^{1-2n} - y_{(n)}^{1-2n}}$$

$$\theta_B = \left(1-\frac{1}{2-2n}\right) \cdot \frac{2^{2-2n} - x_{(n)}^{2-2n}}{2^{1-2n} - x_{(n)}^{1-2n}}$$



\end{document}

