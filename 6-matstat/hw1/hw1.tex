% !TEX TS-program = xelatex
% !TEX encoding = UTF-8

\documentclass[11pt, a4paper]{article} % use larger type; default would be 10pt

\usepackage{fontspec} % Font selection for XeLaTeX; see fontspec.pdf for documentation
\defaultfontfeatures{Mapping=tex-text} % to support TeX conventions like ``---''
\usepackage{xunicode} % Unicode support for LaTeX character names (accents, European chars, etc)
\usepackage{xltxtra} % Extra customizations for XeLaTeX
\usepackage{tikz}
\usetikzlibrary{arrows,calc,patterns}

\setmainfont[Ligatures=TeX]{Garamond} % set the main body font (\textrm), assumes Charis SIL is installed
%\setsansfont{Deja Vu Sans}
\setmonofont[Ligatures=TeX]{Fira Code}

% other LaTeX packages.....
\usepackage{fullpage}
\usepackage[top=2cm, bottom=4.5cm, left=2.5cm, right=2.5cm]{geometry}
\usepackage{amsmath,amsthm,amsfonts,amssymb,amscd,systeme}
\usepackage{unicode-math}
\usepackage{cancel}
\geometry{a4paper} 
%\usepackage[parfill]{parskip} % Activate to begin paragraphs with an empty line rather than an indent
\usepackage{fancyhdr}
\usepackage{listings}
\usepackage{graphicx}
\usepackage{hyperref}
\usepackage{multicol}

\renewcommand\lstlistingname{Algorithm}
\renewcommand\lstlistlistingname{Algorithms}
\def\lstlistingautorefname{Alg.}
\lstdefinestyle{mystyle}{
    % backgroundcolor=\color{backcolour},   
    % commentstyle=\color{codegreen},
    % keywordstyle=\color{magenta},
    % numberstyle=\tiny\color{codegray},
    % stringstyle=\color{codepurple},
    basicstyle=\ttfamily\footnotesize,
    breakatwhitespace=false,         
    breaklines=true,                 
    captionpos=b,                    
    keepspaces=true,                 
    numbers=left,                    
    numbersep=5pt,                  
    showspaces=false,                
    showstringspaces=false,
    showtabs=false,                  
    tabsize=2
}
\lstset{style=mystyle}

\newcommand\course{6 - Матстатистика}
\newcommand\hwnumber{ДЗ №1}                   % <-- homework number
\newcommand\idgroup{ФІ-91}                
\newcommand\idname{Михайло Корешков}  

\usepackage[framemethod=TikZ]{mdframed}
\mdfsetup{%
	backgroundcolor = black!5,
}
\mdfdefinestyle{ans}{%
    backgroundcolor = green!5,
    linecolor = green!50,
    linewidth = 1pt,
}

\pagestyle{fancyplain}
\headheight 35pt
\lhead{\idgroup \\ \idname}
\chead{\textbf{\Large \hwnumber}}
\rhead{\course \\ \today}
\lfoot{}
\cfoot{}
\rfoot{\small\thepage}
\headsep 1.5em

\linespread{1.2}

\begin{document}

% !17.12, !17.13, !17.15 !17.16 !17.17, !17.20, 17.22

\section*{№ 17.12}
\begin{mdframed}
    \(\xi \sim \mathcal{N}(0,1)\). Знайти:
    \begin{itemize}
        \item \(P(\xi \in [-3, 1])\)
        \item \(x : P(|\xi| \le x) = 0.9\)
    \end{itemize}
\end{mdframed}

\subsection*{a) \(P(\xi \in [-3, 1])\)}
\begin{align*}
    P(\xi \in [-3, 1]) &= P(\xi \le 1) - P(\xi \le -3) = \Phi_{0,1}(1) - \Phi_{0,1}(-3) = \Phi_{0,1}(1) - 1 + \Phi_{0,1}(3) = \\
    &= 0.841 - 1 + 0.999 = \boxed{0.84}
\end{align*}

\subsection*{b) \(x : P(|\xi| \le x) = 0.9\)}

\[P(|\xi| \le x) = P(\xi \le x) - P(\xi \le -x) = \Phi_{0,1}(x) - 1 + \Phi_{0,1}(x) = 2\Phi_{0,1}(x) - 1 = 0.9\]

Умова: \(2\Phi_{0,1}(x) - 1 \ge 0.9\quad\). \(\Phi_{0,1}(x) \ge \frac{19}{20}\).

Маємо \(\boxed{x = 1.645}\)

Зображення: \[
    \begin{tikzpicture}
        \fill[red!20, domain=-1.65:1.65, smooth, variable=\x] (-1.65,0) -- plot ({\x}, {2 * exp(-\x*\x / 2)}) -- (1.65,0);
        \draw[scale=1, domain=-4:4, smooth, variable=\x, blue] plot ({\x}, {2 * exp(-\x*\x / 2)});
        \draw[->] (-4,0) -- (4,0);
        \draw[->] (0,0) -- (0,2.2);
        \node at (0.5,1) {$0.9$}; 
    \end{tikzpicture}
\]

\section*{№ 17.13}
\begin{mdframed}
    $N = 10^6$ підкидань чесної монети. $S = $ кількість гербів. 
    Оцінити $P(499000 < S < 501000)$ Чебишовим та ЦГТ.
\end{mdframed}

\subsection*{а) ЦГТ}
$S = \sum_{i=1}^N \mathbb{1}(H_i)$, де $H_i = \{$в $i$-му підкиданні випав герб $\}$.
$P(H_i) = 0.5$. $\{\xi_i = \mathbb{1}(H_i)\}$ - i.i.d.

$$\xi_i \sim B(0.5)$$
$$P(499000 < S < 501000) = P(S < 501000) - P(S \le 499000)$$

$$MS = M\left[\sum_{i=1}^N \mathbb{1}(H_i)\right] = \sum_{i=1}^N M\xi_1 = \frac{N}{2}$$
$$DS = D\left[\sum_{i=1}^N \mathbb{1}(H_i)\right] = \sum_{i=1}^N D\xi_1 = \frac{N}{4}$$

Розглянемо $R = \frac{S - N/2}{\sqrt{N/4}}$. $MR = 0, DR = 1$.
$$P(S < 501000) = P(\sqrt{N/4}R + N/2 < 501000) = P(R < \frac{501000 - N/2}{\sqrt{N/4}}) = P(R < 2) \approx \Phi_{0,1}(2) = 0.977$$
$$P(S \le 499000) = P(R < \frac{499000 - N/2}{\sqrt{N/4}}) = P(R \le -2) \approx 1-\Phi_{0,1}(2) = 0.023$$
\begin{mdframed}[style=ans]
    $$P(499000 < S < 501000) \approx 0.977 - 0.023 = 0.954$$
\end{mdframed}

\subsection*{б) Чебишов}
Розглянемо $Q = S - N/2 = S - 500000$.
$$MQ = 0, DQ = DS = \frac{N}{4} = 10^6 / 4$$

$$P(499000 < S < 501000) = 1 - P(|Q| > 10^3)$$
$$ P(|Q| > 10^3) \le \frac{DQ}{10^6} = \frac{10^6 / 4}{10^6} = 0.25$$
$$P(499000 < S < 501000) \ge 1 - 0.25 = 0.75$$


\section*{№ 17.15}
\begin{mdframed}
    При випадковому блуканнi на числовiй осi частинка, стартуючи з точки 0, 
    робить крок вправо або влiво з ймовiрностями 1/2. Оцiнiть ймовiрнiсть того, що
    пiсля сотого кроку частинку вiддалятиме вiд початкової точки не менше, нiж 10
    крокiв.
\end{mdframed}

\[S(n) - \text{положення частинки після n кроку}\]
\[S(0) = 0; \quad P(S(n+1)=S(n)+1) = P(S(n+1)=S(n)-1) = 1/2.\]
\[\boxed{P(|S(100)| \ge 10) = ?}\]

\[S(n) = 2\sum_{i=1}^n b_i - n, \quad \{b_i\} - iid, \; b_i \sim B(1/2), \quad n = 1,2,...\]
(iid тут і надалі - "independant identically distributed")

\[MS(n) = 2n Mb_1 - n = n - n = 0;\quad DS(n) = 4n Db_1 = 4n/4 = n.\]

тут я моделюю випадкове блукання через бернулівські випадкові величини.

\begin{align*}
    P(|S(100)| > 10) &= P(S(100) > 10) + P(S(100) < -10) = \\
    &= 1 - P(S(100) \le 10) + P(S(100) < -10)
\end{align*}
\begin{align*}
    P(S(100) \le 10) &= P(\frac{S(100)}{\sqrt{n}} \le \frac{10}{\sqrt{n}} ) = P(\frac{S(100)}{10} \le 1) \approx \\
    &\approx \Phi_{0,1}(1) = 0.841
\end{align*}
\begin{align*}
    P(S(100) < -10) &= P(\frac{S(100)}{\sqrt{n}} < -\frac{10}{\sqrt{n}} ) = P(\frac{S(100)}{10} < -1) \approx \\
    &\approx \Phi_{0,1}(-1) = 1 - \Phi_{0,1}(1) = 0.159
\end{align*}
\begin{align*}
    P(|S(100)| < 10) &= 1 - P(S(100) \le 10) + P(S(100) < -10) \approx \\
    &\approx \Phi_{0,1}(-1) = 1 - 0.841 + 0.159 = \boxed{0.318}
\end{align*}

\section*{№ 17.16}
\begin{mdframed}
    У компанiї-авiаперевiзнику оцiнили, що в середньому 4\% квиткiв, проданих 
    на авiарейс, \\ залишаються невикористаними. Як наслiдок, компанiя обрала наступну
    стратегiю: продавати 100 квиткiв на лiтак, що має 98 мiсць. 
    Знайдiть ймовiрнiсть того, що кожному пасажиру, що опиниться на борту лiтака, знайдеться мiсце.
\end{mdframed}

\[u_i \sim B(0.04), \quad u_i = \{i-\text{й квиток невикористано}\}, \quad u_i - iid\]
\[N = 100, M = 98\]
\[S = \sum_{i=1}^N u_i - \text{ скільки невикористаних квитків при 100 проданих}\]
\[\boxed{P(S \ge 100-98) = P(S \ge 2) = ?}\]

\[MS = 100 \cdot Mu_1 = 100 \cdot 0.04 = 4\]
\[DS = 100 \cdot Du_1 = 100 \cdot 0.04 \cdot 0.96 = 3.84\]
\begin{align*}
    P(S \ge 2) &= P(\frac{S - 4}{\sqrt{3.84}}\ge \frac{2-4}{\sqrt{3.84}} = -1.02) \approx \\
    &\approx \Phi_{0,1}(-1.02) = 1-\Phi_{0,1}(1.02) = \boxed{0.154}
\end{align*}
\pagebreak

\section*{№ 17.17}
\begin{mdframed}
    Кожен з членiв жюрi, яке складається з непарної кiлькостi осiб, незалежно 
    вiд iнших приймає справедливе рiшення з ймовiрнiстю 0.7. Якою повинна бути 
    чисельнiсть жюрi для того, щоб рiшення, ухвалене бiльшiстю голосiв, було 
    справедливим з ймовiрнiстю не меншою за 0.9?
\end{mdframed}

\(N\) жюрі. \(N\) - непарне. $f_i = \mathbb{1}(i$-те жюрі прийняло справедливе рішення$) \sim B(0.7)$.
\[S_N = \sum_{i=1}^N f_i\]

Нехай рішення справедливе, якщо хоча б половина журі проголосувала справедливо. Тоді
\[\boxed{P(S_N > N/2) \ge 0.9, N-?}\]

\[MS_N = 0.7 N,\quad DS_N = 0.7 \cdot 0.3 \cdot N = 0.21 N\]

\[P(S_N > N/2) = P\left(\frac{S_N - 0.7 N}{\sqrt{0.21 N}} > \frac{N/2 - 0.7 N}{\sqrt{0.21 N}}\right) \approx\]
\[\approx 1-\Phi_{0,1}\left(-\frac{0.2N}{\sqrt{0.21 N}}\right) = \Phi_{0,1}\left(\frac{0.2N}{\sqrt{0.21 N}}\right) \ge 0.9 \]

Маємо: \[\frac{0.2N}{\sqrt{0.21 N}} \ge 1.282\]
Нехай \(\sqrt{N} = t\)
\[t (0.2 t - 1.282 \cdot \sqrt{0.21}) \ge 0\]
\[0.2 t \ge 1.282 \cdot \sqrt{0.21} = 0.587\]
\[\sqrt{N} = t \ge \frac{0.587}{0.2} = 2.94 \implies N \ge 9\]
\begin{mdframed}[style=ans]
    \[N \ge 9\]
\end{mdframed}
\pagebreak

\section*{№ 17.20}
\begin{mdframed}
    Гральний кубик пiдкидається до тих пiр, аж поки сума очок вперше перевищить 700. 
    Оцiнiть ймовiрнiсть того, що для цього знадобиться:
    \begin{itemize}
        \item a) \(n > 210\) підкидань
        \item б) \(n < 190\) підкидань
        \item a) \(180 \le n \le 210\) підкидань
    \end{itemize}
\end{mdframed}

\[\xi_i \sim Uniform\{1,2,3,4,5,6\} - iid\]
\[M\xi_1 = \frac{1}{6}(1+2+3+4+5+6) = \frac{7}{2} = 3.5\]
\[D\xi_1 = \frac{1}{6}\sum_{i=1}^6(i-3.5)^2 = 2.92\]

\[S_n = \sum_{i=1}^n \xi_i\]
\[MS_n = 3.5 n; \quad DS_n = 2.92 n\]

\subsection*{a)}
\(n > 210\) значить, що після 210-го підкидання сума менша за 700. Лише в такому випадку ми підкидаємо далі і вимушені мати n > 210.
\begin{align*}
    P(n > 210) &= P(S_{210} < 700) = P\left(\frac{S_n - MS_n}{\sqrt{DS_n}} < \frac{700 - MS_n}{\sqrt{DS_n}}\right) = \\
    &= P\left(\frac{S_n - 3.5 n}{\sqrt{2.92 n}} < \frac{700 - 3.5 n}{\sqrt{2.92 n}}\right) \approx \Phi_{0,1}\left(\frac{700 - 3.5 n}{\sqrt{2.92 n}}\right) = \\
    &= \Phi_{0,1}\left(\frac{700 - 3.5 \cdot 210}{\sqrt{2.92 \cdot 210}}\right) = \Phi_{0,1}(-1.413) = \\
    &= 1 - \Phi_{0,1}(1.413) = 0.079
\end{align*}
\begin{mdframed}[style=ans]
    \[P(n > 210) = 0.079\]
\end{mdframed}

\subsection*{b)}
Аналогічно а), але перейдемо до оберненої події.
$$P(n < 190) = 1 - P(n \ge 190)$$
\begin{align*}
    P(n \ge 190) &= P(S_{190} < 700) = \\
    &= \Phi_{0,1}\left(\frac{700 - 3.5 \cdot 190}{\sqrt{2.92 \cdot 190}}\right) = \Phi_{0,1}(1.486) = \\
    &= \Phi_{0,1}(1.486) = 0.931
\end{align*}
$$P(n < 190) = 1 - 0.931 = 0.069$$

\begin{mdframed}[style=ans]
    \[P(n < 190) = 0.069\]
\end{mdframed}

\subsection*{c)}
Аналогічно. Але використаємо те, що $P(n\le 210) = P(S_{210} \ge 700)$.
\begin{align*}
    P(180 \le n \le 210) &= P(S_{180}\le 700, S_{210}\ge 700) = 1 - P(\{S_{180}> 700\} \sqcup \{S_{210}< 700\}) = \\
    &= 1 - P(S_{180} > 700) - P(S_{210} < 700)
\end{align*}
Бо сума ніколи не може бути більша 700 на 180-му кроці, а потім менша за 700 на 210-му - ці події несумісні.

\begin{align*}
    P(S_{180}>700) &= P\left(\frac{S_{180} - 3.5 \cdot 180}{\sqrt{2.92 \cdot 180}} > \frac{700 - 3.5 \cdot 180}{\sqrt{2.92 \cdot 180}} \right) \approx \\
    &\approx 1 - \Phi_{0,1}(3.05) = 0.001
\end{align*}
\begin{align*}
    P(S_{210}<700) &= P\left(\frac{S_{210} - 3.5 \cdot 210}{\sqrt{2.92 \cdot 210}} < \frac{700 - 3.5 \cdot 210}{\sqrt{2.92 \cdot 210}} \right) \approx \\
    &\approx \Phi_{0,1}(-1.413) = 0.079
\end{align*}
\[P(180 \le n \le 210) \approx 1 - 0.001 - 0.079 = 0.92\]

\section*{№ 17.21}
\begin{mdframed}
    Соцiологiчне опитування має на метi оцiнити невiдому частку p' суспiльства, 
    що пiдтримує новий законопроект. Для цього планується опитати n людей i частку 
    p тих з них, що схвалюють новий законопроект, обрати в якостi оцiнки для p'. За 
    допомогою центральної граничної теореми визначте, яким повинно бути число 
    опитаних n для того, щоб з ймовiрнiстю 0.95 оцiнка вiдрiзнялася вiд iстинного 
    значення не бiльше, нiж на 0.01
\end{mdframed}

\[\xi_i \sim B(p) - iid\]
\[S_n = \sum_{i=1}^n \xi_i\]
\[p' - \text{ істинна частка},\; p_n = \frac{S_n}{n}\]
\[MS_n = np',\quad DS_n = np'(1-p')\]
\[Mp_n = p',\quad Dp_n = \frac{p'(1-p')}{n} = \sigma^2\]

\begin{align*}
    P(|p_n-p'| < 0.01) &= P\left(-\frac{0.01}{\sigma} < \frac{p_n - p'}{\sigma} < \frac{0.01}{\sigma}\right) \approx \\
    &\approx \Phi_{0,1}\left(\frac{0.01}{\sigma}\right) - \Phi_{0,1}\left(-\frac{0.01}{\sigma}\right) = 2\Phi_{0,1}\left(\frac{0.01}{\sigma}\right) - 1 \ge 0.95
\end{align*}

\[\Phi_{0,1}\left(\frac{0.01}{\sigma}\right) \ge \frac{0.95 + 1}{2} = 0.975\]
\[\frac{0.01}{\sigma} \ge 1.96\]
\[\sigma \le \frac{0.01}{1.96} = 0.0051\]
\[\sigma = \sqrt{\frac{p'(1-p')}{n}} \le 0.0051\]
\[n \ge \frac{p'(1-p')}{0.0051} \le \frac{1/4}{0.0051} \approx 49\]
Тобто
\[\boxed{n \ge 49}\]


\section*{№ 17.22}
\begin{mdframed}
    \[\{X_i\} - iid. \quad X_i \sim U[-3,3]\]
    \[T_n = \frac{1}{n} \sum_{i=1}^n |X_i|\]
    \[\sqrt{n} \cdot (T_n - a) \overset{d}{\longrightarrow} \mathcal{N}(0,b^2) \]
    \[a,b - ?\]
\end{mdframed}

\[M|X_i| = \int_{-3}^3 |x| \cdot \frac{1}{6}dx = \frac{1}{6} \left(\int_0^3 x dx + \int_{0}^{-3} x dx\right) = \]
\[= \frac{1}{6} \left(\frac{9}{2} + \frac{9}{2}\right) = \frac{9}{6} = \frac{3}{2} = 1.5\]

\[M|X_i|^2 = MX_i^2 = \int_{-3}^3 x^2 \cdot \frac{1}{6}dx = \frac{1}{18} \left(3^3 - (-3)^3\right) = 3\]

\[D|X_i| = M|X_i|^2 - \left(M|X_i|\right)^2 = 3 - 2.25 = 0.75\]

\[MT_n = \frac{n}{n}M|X_1| = 1.5; \quad DT_n = \frac{n}{n^2}D|X_1| = \frac{0.75}{n}\]
За ЦГТ:
\[\frac{T_n - MT_n}{\sqrt{DT_n}} \overset{d}{\longrightarrow} \mathcal{N}(0,1)\]
\[\sqrt{n} \frac{T_n - 1.5}{\sqrt{0.75}} \overset{d}{\longrightarrow} \mathcal{N}(0,1)\]
\[\sqrt{n} (T_n - 1.5) \overset{d}{\longrightarrow} \mathcal{N}(0,0.75)\]
А отже 
\[a = MT_n = 1.5,\quad b^2 = DT_n = 0.75\]


\end{document}

