% !TEX TS-program = xelatex
% !TEX encoding = UTF-8

\documentclass[11pt, a4paper]{article} % use larger type; default would be 10pt

\usepackage{fontspec} % Font selection for XeLaTeX; see fontspec.pdf for documentation
\defaultfontfeatures{Mapping=tex-text} % to support TeX conventions like ``---''
\usepackage{xunicode} % Unicode support for LaTeX character names (accents, European chars, etc)
\usepackage{xltxtra} % Extra customizations for XeLaTeX
\usepackage{tikz}
\usetikzlibrary{arrows,calc,patterns}

\setmainfont[Ligatures=TeX]{Garamond} % set the main body font (\textrm), assumes Charis SIL is installed
%\setsansfont{Deja Vu Sans}
\setmonofont[Ligatures=TeX]{Fira Code}

% other LaTeX packages.....
\usepackage{fullpage}
\usepackage[top=2cm, bottom=4.5cm, left=2.5cm, right=2.5cm]{geometry}
\usepackage{amsmath,amsthm,amsfonts,amssymb,amscd,systeme}
\usepackage{cancel}
\geometry{a4paper} 
%\usepackage[parfill]{parskip} % Activate to begin paragraphs with an empty line rather than an indent
\usepackage{fancyhdr}
\usepackage{listings}
\usepackage{graphicx}
\usepackage{hyperref}
\usepackage{multicol}

\renewcommand\lstlistingname{Algorithm}
\renewcommand\lstlistlistingname{Algorithms}
\def\lstlistingautorefname{Alg.}
\lstdefinestyle{mystyle}{
    % backgroundcolor=\color{backcolour},   
    % commentstyle=\color{codegreen},
    % keywordstyle=\color{magenta},
    % numberstyle=\tiny\color{codegray},
    % stringstyle=\color{codepurple},
    basicstyle=\ttfamily\footnotesize,
    breakatwhitespace=false,         
    breaklines=true,                 
    captionpos=b,                    
    keepspaces=true,                 
    numbers=left,                    
    numbersep=5pt,                  
    showspaces=false,                
    showstringspaces=false,
    showtabs=false,                  
    tabsize=2
}
\lstset{style=mystyle}

\newcommand\course{5 - Аналіз алгоритмів}
\newcommand\hwnumber{ДЗ №4}                   % <-- homework number
\newcommand\idgroup{ФІ-91}                
\newcommand\idname{Михайло Корешков}  

\usepackage[framemethod=TikZ]{mdframed}
\mdfsetup{%
	backgroundcolor = black!5,
}
\mdfdefinestyle{ans}{%
    backgroundcolor = green!5,
    linecolor = green!50,
    linewidth = 1pt,
}

\pagestyle{fancyplain}
\headheight 35pt
\lhead{\idgroup \\ \idname}
\chead{\textbf{\Large \hwnumber}}
\rhead{\course \\ \today}
\lfoot{}
\cfoot{}
\rfoot{\small\thepage}
\headsep 1.5em

\linespread{1.2}

\begin{document}
% 2.16 а, б
% 2.17 в
% 2.18
% 2.24 в

\section*{№ 2.16 a)}
\begin{mdframed}
    Яку мультиплiкативну складнiсть має множення двох матриць розмiрами
n × m та m × k? Яку складнiсть має множення квадратних матриць розмiрами n × n?
\end{mdframed}

$$C = AB$$
$$A = {a_{ij}}_{i=\overline{1,n}}^{j=\overline{1,m}}$$
$$B = {b_{ij}}_{i=\overline{1,m}}^{j=\overline{1,k}}$$

\subsection*{Множення брутфорсом}
$$c_{ik} = \sum_{l=1}^m a_{il}b_{lk} \quad (m\text{ множень})$$

\begin{lstlisting}
    for i = 1 to n:
        for j = 1 to k:
            C[i,j] = 0 
            for k = 1 to m:
                C[i,j] += A[i,k] * B[k,j]
\end{lstlisting}

Складність - $T_1 = (\text{кількість комірок в }C)\cdot m = n\cdot m \cdot k$
\begin{mdframed}[style=ans]
    \begin{itemize}
        \item Множення n × m та m × k: $T_1 = nmk$
        \item Множення n × n та n × n: $T'_1 = n^3$
    \end{itemize}
\end{mdframed}

\section*{№ 2.16 б)}
\begin{mdframed}
    Знайдiть мультиплiкативну складнiсть множення симетричних матриць розмiрами
n × n.
\end{mdframed}

Я не знайшов алгоритму швидше, ніж як для множення довільних квадратних матриць.
Єдине що, можна десь зекономити в кешуванні.

$$\begin{pmatrix}
    a_{11} & a_{12} \\ a_{21} & a_{22}
\end{pmatrix}\begin{pmatrix}
    b_{11} & b_{12} \\ b_{21} & b_{22}
\end{pmatrix} = \begin{pmatrix}
    a_{11}b_{11}+\boxed{a_{12}b_{21}} & a_{11}b_{12} + {a_{12}b_{22}}\\
    a_{21}b_{11}+{a_{22}b_{21}} & \boxed{a_{21}b_{12}} + a_{22}b_{22}
\end{pmatrix}$$

Повторні множення:
$$a_{12}b_{21} = a_{21}b_{12}$$

Тобто вони виникають у виразах виду $a_{ij}b_{ji},\; i\ne j$.
Можна зменшити цю кількість множень в половину за допомогою кешування.

В свою чергу, такі добутки з'являються лише в елементах $C$ виду $c_{ii}$.
І взагалі,
$$c_{ii} = \sum_{j=1}^n a_{ij}b_{ji} = a_{ii}b_{ii} + \sum_{j\ne i} a_{ij}b_{ji}$$

$$(n=3)\quad AB = \begin{pmatrix}
    a_{11}b_{11}+\boxed{a_{12}b_{21}}+\boxed{a_{13}b_{31}} & * & *\\
    * & \boxed{a_{21}b_{12}} + a_{22}b_{22} + \boxed{a_{23}b_{32}} & *\\
    * & * & \boxed{a_{31}b_{13}} + \boxed{a_{23}b_{32}} + a_{33}b_{33} \\
\end{pmatrix}$$

Перепишемо алгоритм з попереднього пункту
\begin{lstlisting}
    // ab[] - масив

    for i = 1 to n:
        for j = 1 to n:
            C[i,j] = 0 
            if i==j:
                ab[i,i] = A[i,i]B[i,i]
                for k = 1 to n:
                    if ab[i,k] is not set:
                        ab[i,k] = ab[k,i] = A[i,k]B[k,i]
                    C[i,j] += ab[i,k]
            else:
                for k = 1 to n:
                    C[i,j] += A[i,k] * B[k,j]
\end{lstlisting}

Таких множень всього $n(n-1)$ і половина з них пропускається завдяки кешуванню.

$$T_{sym}(n) = T(n) - n(n-1)/2 = n^3 - n^2 + n$$


\begin{mdframed}[style=ans]
    $$T_{sym}(n) = T(n) - n(n-1)/2 = n^3 - n^2 + n$$
\end{mdframed}

\section*{№ 2.17 в)}
\begin{mdframed}
    Мультиплікативна складність обчислення визначника матриці n x n методом елементарних\\ перетворень (методом Гаусса). 
    Вважайте, що перестановка рядкiв матрицi виконується за знехтовно малий час
\end{mdframed}

Note: \\
\texttt{A[i,:]} це i-й рядок\\
\texttt{A[:,j]} це j-й стовпець


Алгоритм:
\begin{lstlisting}

    // Для максимальної складності вважаємо, що А - повнорангова

    for i = 1 to n-1:
        for j = i+1 to n: 
            // (1)
            coef = A[j, i] / A[i, i]   // 1
            // (2)
            A[j,:] -= coef * A[i,:]    // (n-i), бо перші i елементів рядка = 0
        
    // Зараз матриця A має upper rectangular форму
    // det = просто добутку елемнтів на діагоналі

    det = A[1,1]
    for i = 2 to n:    // *(n-1)
        det *= A[i,i]  // 1
    return det
\end{lstlisting}

Рядок після (1) виконається $$\sum_{i=1}^{n-1} (n-i) = \frac{n(n-1)}{2}$$ разів.\\
У рядку після (2) за весь час роботи виконається $$\sum_{i=1}^{n-1} (n-i)^2 = 1/6 (n - 3 n^2 + 2 n^3)$$ множень

$$T_{det}(n) = \frac{n(n-1)}{2} + 1/6 (n - 3 n^2 + 2 n^3) + n-1 = $$
$$=1/3 (n^3 + 2 n - 3) = O(n^3)$$

\section*{№ 2.18}
\begin{mdframed}
    Яку мультиплiкативну складнiсть в найгiршому випадку має обчислення рангу квадратної матрицi розмiру n × n\\
а) методом обвiдних мiнорiв;\\
б) методом елементарних перетворень?
\end{mdframed}

\subsection*{a) Обвідні мінори}

Note: \\
\texttt{A[a:b, c:d]} - це підматриця матриці A 

\begin{lstlisting}
    // IN: A
    // OUT: rang(A)

    // функція може змінювати B
    function find_minor(&B, r):
        // max (n-r)^2 ітерацій
        for i = r to n:
            for j = r to n:
                // Міняємо місцями рядки та стовпці так, щоб мінор був ліворуч згори
                B = A
                B[r,:] = A[i,:]
                B[:,r] = A[:,j]
                d = det(B[1:r, 1:r])   // T_det(r)
                if d != 0:
                    return True
        return False
    
    for r = 1 to n:
        B = A
        found = find_minor(B, r)  // (n-r)^2 * T_det(r)
        if not found:
            return r-1
        A = B
    return n

\end{lstlisting}

$$T_{rang}(n) = \sum_{r=1}^n (n-r)^2 * T_{det}(r) = $$
$$= \sum_{r=1}^n (n-r)^2 * O(r^3) = \sum_{r=1}^n O(r^5) = $$
$$= O(\sum_{r=1}^n r^5) = O(n^6)$$


\subsection*{б) Gauss elimination}
\begin{lstlisting}
    Виконуємо gauss elimination з задачі 2.17  // T(n) = O(n^3)
    (єдине що, треба пропускати рядки, де з'явилися нулі на початку)
    Дивимося на кількість нульових рядків знизу матриці
\end{lstlisting}

$$T_{rang}'(n) = O(n^3)$$

\section*{№ 2.24 в)}
\begin{mdframed}
    Iнтерполяцiйна формула Лагранжа дозволяє також знаходити значення полiному в
довiльнiй точцi. Знайдiть мультиплiкативну складнiсть такого обчислення;
\end{mdframed}

$$p(x) = \sum_{k=0}^m \lambda_k \prod_{t\ne k} (x-x_t)$$
$$\lambda_k = \frac{y_k}{\displaystyle \prod_{t\ne k}(x_k - x_t)}$$

$$T(m) = \sum_{k=0}^m (1 + (m)) \cdot (m) = \sum_{k=0}^m m(m+1) = m(m+1)^2$$

$$T(m) = m^3 + 2m^2 + m = O(m^3)$$



\end{document}

